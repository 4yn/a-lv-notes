\documentclass[../../main]{subfiles}

\begin{document}

\subsection{Social Stratification}

\subsubsection{Equality for All}

\paragraph{In your society, how far is equality for all a reality? (GCE-A/12)}-

\paragraph{Is equality for all within your country a realistic and desirable aim? (RJC/13)}-

\subsubsection{A Caring Community}

\paragraph{`Education has provided a population that is literate but not compassionate.' How far is this true of your country? (RJC/15)}-

\begin{description}
	\item[Analysis] Question suggests that there is a cause and effect relationship between education and lack of compassion, and though framed as a strong statement it still is to be treated like a extreme position. Reduce the question to ``Have we neglected compassion in our pursuit to educate people,'' not just ``Are there compassionate and educated people in Singapore today?'' (Side note: This essay does not lend itself well to the argument-counterargument format, heres a compromise)
	\item[Sample Thesis] While successful in producing literate people, the education system in Singapore has also tried to inculcate compassion but such measures have not been successful.
	\item[Points]:
		\begin{description}
			\item[Social Priorities] entertain the view that education and literacy is more important than compassion.
			\item[Competition] among schools and among students corroborate the stress to perform academically rather than to be developed emotionally, as well as the fact that the competition between students is held mostly through comparison of tangible achievements like grades and competition placing rather than soft, emotional skills.
		\end{description}
\end{description}

\paragraph{Has competition resulted in a less compassionate and caring society? Discuss this with reference to your country. (RJC/15)}-

\subsubsection{``Success'' and its Process}

\paragraph{To what extent can technology be a solution to social problems? (RJC/15)}-

\paragraph{`Competition breeds success.' TWE is this true? (RJC/14)}-

\paragraph{`Success is determined by one's intelligence.' Discuss. (RJC/12)}-

\end{document}