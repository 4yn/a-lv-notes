\documentclass[../../main]{subfiles}

\begin{document}

\subsection{Government and Activism}

\paragraph{Consider the view that people in your society have unrealistic expectations of their government. (RJC/14)}-

\begin{description}
	\item[Analysis] Expectations of the government typically encompass security, standard of living and services but could be extended to conditions specific to a culture or society. Understand why some expectations develop and also why some may not be met due to limitations, unpopular policy and lose-lose situations.
	\item[For Points] :
		\begin{description}
			\item[Economic Slowdown] is a reality since Singapore is tending towards average growth figures 
			\item[Complaint Culture and Feedback] has gained increasing spotlight due to new platforms for activism, especially online through social media with its massive audience. Internet subcultures (HardwareZone's EDMW, /r/Singapore) also act as echo chambers and may amplify issues out of hand.
			\item[Initiative] from the people themselves is necessary and should be less dependent on the government, especially in areas where the government has little potential for conflict with despite its paternalistic nature. Growing support for civil society organizations and similar initiatives are notable cases.
		\end{description}
	\item[Against Points] :
		\begin{description}
			\item[Mutual Expectations] of the people are also placed by the government, therefore the people are also entitled to their demands of the government.
			\item[Inevitable Liberalism]
		\end{description}
\end{description}

\paragraph{TWE is healthy debate encouraged in your society? (RJC/14)}-

\paragraph{How far is it important for people to be aware of current events in countries other than their own? (GCE-A/14)}

\paragraph{Why should we be concerned with current affairs when most of them will soon be forgotten? (GCE-A/13)}-

\paragraph{How far, in your society, should unpopular views be open to discussion? (GCE-A/13)} See section 1.5 Censorship and Free Speech

\paragraph{Do you agree that the tools of social media have reinvented social activism? (RJC/12)} See section 1.2 New Media

\paragraph{TWE does your country challenge the current state of affairs? (RJC/11)}-

\end{document}