\documentclass[../../main]{subfiles}

\begin{document}

\subsection{Aging Population}

\subsubsection{Approaches to Aging}

\paragraph{TWE is old age considered a burden to your society? (RJC/15)}-

\begin{description}
	\item[Analysis] Need to consider the rest of society's opinion about managing the aging population, in the Singapore context. ``Burden'' implies some extent of heavy, unwanted load that is borne by society. Arguments should examine the impacts of managing an aging society as well as causes and scale (validity) of said impacts, and in extension also consider if there is a current trend of change in society's opinion of aging.
	\item[For Points] :
		\begin{description}
			\item[Financial Burden] in that money is used up when having to care for elderly. With aging comes extra needs, care and resources which can create a financial burden for their families as well as on the government. Singapore's social security system, as a part of the budget involves handouts of \$320 million in its first year, in addition to current policy such as the Pioneer Package which involves \$8 billion in handouts.
			\item[Inefficient Workers] in that elderly are ultimately less physically able, reliable workers and hence are unfit to contribute to the economy in the way they once were.
			\item[Emotional Burden] in that elderly are an extra liability to those who take care of them, draining time and effort off the current workforce.
		\end{description}
	\item[Against Points] :
		\begin{description}
			\item[Financial Gap-filler] where the elderly, with their excess of time and still present ability are able to fill the gaps in the economy which desperately need workers, e.g. cleaner staff, hawkers. Older workers has grown more than three times in the past decade, providing more filling for previous economic gaps. The number of elderly workers in Singapore who earn over \$1k a month has quadrupled in absolute numbers over the past decade as well.
			\item[Experienced Workforce] where elderly are already trained and experienced in their own fields, and are hence less indispensable than perceived. Local business experts praise elderly workers as having commendable work ethic, commitment, ability to mentor juniors and independent (low maintenance).
			\item[Secondary Caretakers] where the elderly population are there to back up families, especially in the Singaporean context where 54\% of married couples are dual-income and two thirds of the elderly population cohabit with their children. Elderly provide an alternate support network to complement daycare centers as well as domestic workers in the matter of childcare.
		\end{description}
\end{description}

\paragraph{Is old age approached with horror in your society? (RJC/12)}-

\paragraph{Is longer life expectancy a blessing or a curse? (RJC/11)}-

\paragraph{`Retirement is a redundant word today.' TWE is this true? (RJC/11)}-

\subsubsection{Tackling an Aging Population}

\paragraph{When a government's finances for social welfare are limited, should they be directed to the young or the old? (GCE-A/15)}-

\paragraph{How far DYA that it is the responsibility of the young to take care of the elderly in your society? (RJC/12)}-

\paragraph{TWE are the young in Singapore favored at the expense of the elderly? (GCE-A/04)}-

\end{document}