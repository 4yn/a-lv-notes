\documentclass[../../main]{subfiles}

\begin{document}

\subsection{New Media}

\subsubsection{Managing the New Media}

\paragraph{`The media needs to exercise more responsibility.' DYA? (RJC/15)}-

\begin{description}
	\item[Analysis] Do note the qualifier `More' as well as have a clear understanding of what the media does, as well as society's expectations of the media, namely accuracy, objectivity (to its most possible extent), timeliness, quality and ethical practice and the implications of proper practice and malpractice (misinformed public, societal insecurity etc). Reduce question to ``Where has the media been irresponsible and where can it improve''
	\item[For Points] :
		\begin{description}
			\item[Biased Reporting] of mainstream media is still rampant and leads to the media misinforming the general public and ultimately leading to harm. Institutions such as Fox News in the USA as well as tabloids such as the British-based Sun tend to err on the side of hyperbole and extravagance to appeal to the baser instincts of the general public, such as accusing Obama of being a Muslim and more recently, the success of Donald Trump's campaign whose rhetoric, though misleading, appeals to many due to the extreme but appealing views he presents. Online institutions with low barriers of entry make the situation worse in that they prioritize appealing content over truthful content, hence leading to the proliferation of ``clickbait'' which spans the whole of social media, including sites like Buzzfeed and Cracked. Misleading information can spread as well - medical studies with the slightest glimmer of potential can end up being extrapolated out of proportion - studies conducted on eating chocolate during pregnancy which had no substantial conclusion was reported to the public that eating chocolate would improve blood flow. A similar fake study was released in 2011 where an institution reported that humans eat more than five arachnids during sleep over a month and it spread like wildfire over the media, only to be revealed a year later that it was all a hoax.
			\item[Cultural Insensitivity] in that the media, in their process of exercising freedom of speech and lesser so to make money, will end up crossing cultural boundaries which anger a population. Charlie Hebdo's comics which enraged Muslim assailants to shoot up their office and the backlash of people screaming ``Cultural Appropriation'' when screened with Coldplay's \textit{Hymm for a Weekend} also come to mind.
			\item[Money and Morals] where the media work against social standards and goals in the pursuit of profit. Especially in the case where gender stereotypes are perpetuated in advertising with the portrayal of women being submissive to men while looking `pretty' work against the progression of cultural standards. 
		\end{description}
	\item[Against Points] :
		\begin{description}
			\item[Self-Censorship] where many media outlets do attempt to maintain a high barrier of entry to publication through putting public-sourced responses through the same rigorous editing and curation process of typical media. The successful maintainance of quality in the Straits Times Forum section, especially the recent backlash against Lee Wei Ling who had retracted her affiliation with the fourth estate due to failure to uphold publishing standards like integrity.
		\end{description}
\end{description}

\subsubsection{Impacts of New Media}

\subparagraph{New Media and Communication} :

\paragraph{To what extent has new media made us poor communicators? (RJC/14)}-

\paragraph{TWE has new media changed the face of human interaction? (RJC/11)}-

\paragraph{To what extent are young people in your society slaves to the mass media? (RJC/12)}-

\subparagraph{New Media and Advocacy} :

\paragraph{To what extent is social media a useful platform for change? (RJC/13)}-

\paragraph{Discuss the impact of new media on social cohesion in your society. (RJC/12)}-

\paragraph{TWE DYA that the media has been a liberating force? (RJC/110)}-

\paragraph{Do you agree that the tools of social media have reinvented social activism? (RJC/12)}-

\paragraph{`Social media has changed the face of politics.' TWE is this true? (RJC/11)}-

\subparagraph{New Media and Morals} :

\paragraph{`New media is a new evil.' Discuss. (RJC/12)}-

\subsubsection{New vs Old Media}

\paragraph{`Books serve little purpose in education as technological developments become more sophisticated.' How far DYA? (GCE-A/15)}-

\paragraph{In the digital age, do newspapers still have a role in your society? (GCE-A/11)}-

\end{document}