\documentclass[../../main]{subfiles}

\begin{document}

\subsection{Privacy}

\begin{description}
	\item[Privacy] is defined as the state where one is not observed or disturbed by other people, but typically with regard to media, the former definition is more important. Mass surveillance seems to be the key action taken by many global governments in identifying criminals and threats before they strike, through mass collection of data and background checks, such as the USA's National Security Agency and its many data collection programs. Especially in the proliferation of the digital age, privacy has become even more queer an issue, where using the internet can enhance privacy through anonymity, but at the same time leaving people vulnerable to surveillance.
	\item[National Security] is where a government takes action to work against the possibility of and current crises, typically demarcated as internal and external threats. In the case of Singapore, external threats are mostly addressed through active diplomacy with many countries including the ASEAN coalition and ASEAN's alliances with many of the world's giants and the maintenance of our own military, whereas internal threats are addressed through the maintenance of the military, initiatives of ``Total Defence'' as well as by policy such as the Internal Security Act where suspected criminals can be detained without trial. Especially worrying is the recent trends in terrorism, with Indonesia and Malaysia being hotspots for ISIS cells as well as the recent incarceration and deportation of more than 20 foreign workers suspected of planning terrorist attacks, as well as global trends such as the attacks in Brussels and Paris. Singapore's Criminal Procedure Act and Telecommunications Act give the government power to request data from internet services, and regularly inquires information from Microsoft, Facebook, Google and Twitter
\end{description}

\paragraph{`Personal privacy and national security cannot coexist.' Comment. (RJC/15)}-

\begin{description}
	\item[Analysis] Consider the when, how and why national security requires privacy to be relinquished and back up with real world cases of threats and counter-threats. Points should range from finding an extent which privacy and security are mutually exclusive to considering the advantages of using other solutions to address national security, such that security can be maintained but without as much involvement of infringing privacy.
	\item[For Points] :
		\begin{description}
			\item[Changing Threats] mean that new national security threats are increasingly hard to spot and new criminals who do not have previous records force governments to resort to mass surveillance and intelligence management.
			\item[Value Systems] in societies which place higher priority on safety than individual liberty are able to relinquish privacy to ensure security.
		\end{description}
	\item[Against Points] :
		\begin{description}
			\item[Passive Surveillance] distances data collection from infringing privacy by arguing that no human eyes are actually prying on data, up until flags are tripped, much like the many systems used in the USA's NSA program, though it is rebutted by anecdotal evidence that employees still often exploit data collected for their own personal gain.
			\item[Distrust] arises when infringing privacy is tapped on as the main method to address national security and sours the relationship between the governed and the governor, eventually weakening national security. Cases such as the worries of local Bangladeshi workers when reports of their coworkers being deported on grounds of suspected terrorism can be used. Economic implications also come into play, where trade deals such as Huawei's expansion into the USA are halted due to their connection with the Chinese Government, as well as AT\&T's delayed expansion into Germany after the scare of the NSA being revealed.
			\item[Inefficacy] of mass surveillance is very real, where too much data means that the important information is buried and goes unnoticed and hence fail to deliver. Consider the intelligence on the 9/11 attacks, where the ``Stellar Wind'' program collected so much data that 9/11 intelligence was ignored.
			\item[Corruption] of the government occurs when it has too much power to infringe upon privacy. Privacy is hence needed for checks and balances against the government, especially in instances of revolution or where political dissidence occurs as a necessary step in the political process. 
		\end{description}
\end{description}

\paragraph{`There is no such thing as privacy today.' Comment. (RJC/15)}-

\paragraph{To what extent have people given up their freedom for comfort? (RJC/14)}-

\paragraph{`Privacy is dead, thanks to new media.' TWE do you think this is detrimental to modern society? (RJC/12)}-

\end{document}