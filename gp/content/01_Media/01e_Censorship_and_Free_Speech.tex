\documentclass[../../main]{subfiles}

\begin{document}

\subsection{Censorship and Free Speech}

\begin{description}
	\item[Censorship] is defined as the suppression of speech, public communication or other forms of media which may be considered harmful, insensitive, politically incorrect or inconvenient as deemed by parties such as governing bodies and media outlets. Key examples of modern-day censorship include the Great Firewall of China and Singapore's Internal Security Act as well as the local system of OB markers
\end{description}

\paragraph{How far, in your society, should unpopular views be open to discussion? (GCE-A/13)}-

\paragraph{With the rise of new media, censorship is needed now more than ever. DYA? (RJC/15)}-

\paragraph{`Freedom of speech should be a privilege, not an entitlement.' How far DYA? (RJC/15)}-

\begin{description}
	\item[Analysis] Need to make the distinction between privilege which suggests that it is provided and can be relinquished and entitlement which suggests that free speech is a right. Reduce the question to ``should we be able to remove peoples rights to free speech''
	\item[For Points] :
		\begin{description}
			\item[Abuse] of free speech is a real possibility where undue libel and attacks of people can have external repercussions
			\item[Harmful Potential] of free speech, especially in multicultural societies where strong beliefs and opinions are more than likely to clash and result in harm, as such having the power to remove freedom of speech enables people to be protected, right to comfort and safety of opinion and thoughts with regard to culture is preserved
		\end{description}
	\item[Against Points]:
		\begin{description}
			\item[Protection of Rights] in that free speech is a guarantor of other rights and freedoms, where having the power to remove freedom of speech compounds into issues of preventing checks and balances on government, as well as checks and balances on the presence of other rights. Having the ability to restrict rights may also open doors to other rights being infringed upon.
		\end{description}
\end{description}

\paragraph{`The media needs to exercise more responsibility.' DYA? (RJC/15)}-

\paragraph{`Censorship is both harmful and futile in today's society.' Comment. (RJC/14)}-

\paragraph{`With the emergence of new media, there is a greater need for censorship.' How true is this of your society? (RJC/13)}-

\paragraph{`We should have the freedom to read and watch what we like.' Comment. (RJC/13)}-	

\end{document}