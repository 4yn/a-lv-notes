\documentclass[../../main]{subfiles}

\begin{document}

\subsection{Dependence and Dangers}

\paragraph{Are we overdependent on digital technology? (RJC/15)}-

\paragraph{Is it foolish to be wary of scientific progress? (RJC/14)}-

\begin{description}
	\item[Analysis] Understand that there is a fine threshold between being wise or foolish when being wary of science. Aim to find that threshold through its discussion in the essay.
	\item[Wariness is Foolish] :
		\begin{description}
			\item[Emotional Burden] from highly emotional groups such as animal rights activists and creationists are cases of excessive wariness due to illogical beliefs, in spite of regulation and evidence.
			\item[Precaution and Urgency] can hinder and delay science too late after the point of usefulness. Bureaucracy and checks can work counter to urgent demands for technologies, such as the delaying of the TKM-Ebola drug in the peak of the ebola outbreak.
		\end{description}
	\item[Wariness is Wise] :
		\begin{description}
			\item[Can of Worms] phenomena where discoveries lead to even more problems mean that caution is healthy.
			\item[Ethical Maturity] develops slower than scientific research and hence there are junctures at which the ethical decision cannot be made just yet.
			\item[Freedom of Private Science] means that private organizations are able to conduct research with a biased motive, whether political or profit-driven. This can lead to undesirable technologies and methodologies, therefore there is a need for wariness to mitigate the potential harms.
		\end{description}
\end{description}

\paragraph{Does technology facilitate crime? (RJC/11)}-

\end{document}