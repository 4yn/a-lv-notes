\documentclass[../../main]{subfiles}

\begin{document}

\subsection{Research}

\paragraph{`We should only fund scientific research which benefits our quality of life.' (RJC/15)}

\begin{description}
	\item[Analysis]
	\item[For Points] :
		\begin{description}
			\item[Pragmatism] This opinion is suitable in situations where funding is scarce, and the most pressing form of research which is necessary is that which directly impacts quality of life rather than esoteric causes.
			\item[Stakeholders] Funding of research ultimately needs to translate into returns, and from the standpoint of private economic entities or from governments, not creating a bar to restrict funding for research results in irresponsible management of funds.
		\end{description}
	\item[Against Points] :
		\begin{description}
			\item[Stepping-Stone Science] where Science is a process which builds upon previous knowledge, and that certain facets of scientific study do not have any clear goal, motive or benefit to quality of life end up being the foundation for other knowledge that can improve quality of life, such as the theory of relativity and its applications in GPS and satellite building and x-rays to its medical applications.
			\item[For the Sake of Knowledge] can also be brought up as a point, in that funding research to purely benefit understanding and knowledge is permissible.
			\item[Practicality] in that creating this barrier to funding forces scientists into working with an end-product in mind and its implications to the rest of the field of science, such as when foreign talent Neal Copeland and Nancy Jenkins emigrated to Houston, Texas to conduct research and Lee Wei Ling criticizing local funding boards citing the fact that Biopolis and Fusionopolis overemphasized economic output.
		\end{description}
\end{description}

\end{document}