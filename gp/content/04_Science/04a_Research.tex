\documentclass[../../main]{subfiles}

\begin{document}

\subsection{Research}

\subsubsection{Motivating Research}

\paragraph{`We should only fund scientific research which benefits our quality of life.' (RJC/15)}-

\begin{description}
	\item[Analysis]
	\item[For Points] :
		\begin{description}
			\item[Pragmatism] This opinion is suitable in situations where funding is scarce, and the most pressing form of research which is necessary is that which directly impacts quality of life rather than esoteric causes.
			\item[Stakeholders] Funding of research ultimately needs to translate into returns, and from the standpoint of private economic entities or from governments, not creating a bar to restrict funding for research results in irresponsible management of funds. A* uses \$5 billion in funding annually for science research and large pharmaceutical companies like GSK and Bayer AG approximate that each drug released to the public ultimately costs \$5 billion to produce.
		\end{description}
	\item[Against Points] :
		\begin{description}
			\item[Stepping-Stone Science] where Science is a process which builds upon previous knowledge, and that certain facets of scientific study do not have any clear goal, motive or benefit to quality of life end up being the foundation for other knowledge that can improve quality of life, such as the theory of relativity and its applications in GPS and satellite building and x-rays to its medical applications.
			\item[For the Sake of Knowledge] can also be brought up as a point, in that funding research to purely benefit understanding and knowledge is permissible.
			\item[Practicality] in that creating this barrier to funding forces scientists into working with an end-product in mind and its implications to the rest of the field of science, such as when foreign talent Neal Copeland and Nancy Jenkins emigrated to Houston, Texas to conduct research and Lee Wei Ling criticizing local funding boards citing the fact that Biopolis and Fusionopolis overemphasized economic output.
		\end{description}
\end{description}

\subsubsection{Limiting Research / Ethical Research}

\paragraph{To what extent is it desirable to place limits on scientific research? (RJC/15)}-

\paragraph{`Unlimited scientific research is the only way to make real scientific progress.' DYA? (RJC/15)}-

\paragraph{TWE can the regulation of scientific or technological developments be justified? (GCE-A/14)}-

\paragraph{TWE should we limit technology's influence on sports? (RJC/14)}-

\paragraph{`Moral Considerations hinder scientific progress.' C. (RJC/12)}-

\paragraph{Should every country have the right to carry out unlimited scientific research? (GCE-A/09)}-

\paragraph{Are there any circumstances in which it would be acceptable to use animals for scientific research? (GCE-A/06)}-

\subsubsection{Niche Research}

\paragraph{Consider the view that advances in gene therapy research have gone too far. (RJC/14)}-

\paragraph{DYA that exploring space should not be a priority in today's world? (RJC/14)}-

\paragraph{Can space research be justified nowadays? (GCE-A/11)}-

\paragraph{`Air travel should be discouraged, not promoted.' TWE do you agree? (GCE-A/08)}-

\end{document}