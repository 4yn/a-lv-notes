\documentclass[../../main]{subfiles}

\begin{document}

\subsection{Question Prompts}

\begin{description}
	\item[DYA] Thesis is a distinct stand
	\item[TWE] Thesis contains qualifying criteria and then determines a slant. This is generally the most demanding question, consider all questions as a \textbf{TWE} question when revising to account for worst case scenario
	\item[Comment] Thesis involves some synthesis / balancing of information to reach a conclusion
	\item[How far...] Similar to \textbf{TWE}
\end{description}

\subsection{Question Qualifiers}

\begin{description}
	\item[Only] indicates a extreme position as well as usually indicating the presence of a criteria when making a decision. Consider other criteria that can satisfy a substitution test with this criteria when looking for points, where such a judgment is suitable as well as the dangers of having such a narrow view (think missed opportunity etc). Conclude with a compromise between the for and against of this opinion.
	\item[More/Less] indicates a change in state, describing the current trend and future trends or supposed better trends to aspire towards. Taking a stand with such a questions requires you to substantiate the fact if the trend exists or not, followed by which course of action should be taken.
	\item[Excess/Lacking] indicates that a large or little trend is extreme to the point of harm or dysfunction, and in taking a stand you need to show the extent of these harms.
	\item[Cause/Effect] indicates a process which will need to be elaborated on when explaining most points, be it identifying the causes which lead to a certain effect or the effects which stem from causes.
	\item[Problem/Solution] to problems need to be considered as a system of cause and effect in addition to considering the nature of the problem and its solution. Many problems and solutions involve some extent of multi-facetedness which is where an essay's balance view will stem from. Always consider the whats, hows and whys of a certain problem before attempting to approach it, especially in questions which have an implied problem and solution set.
\end{description}

\end{document}