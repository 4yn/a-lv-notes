\documentclass[../../main]{subfiles}

\begin{document}

\subsection{Question Triggers}

\begin{description}
	\item[Your Society] implies that points need to pertain to the local, Singaporean situation. Hence, knowledge of the local culture, demographic, current affairs, politics and other matters are crucial to GP, especially once the application question comes into play. Key unique characteristics about Singapore that changes what points are applicable to us include:
		\begin{description}
			\item[Multiculturalism] with multiple local races and religions as well as a open attitude towards inflows of foreigners and foreign talent
			\item[Asian Values] which place much focus on conservatism as well as importance of family-oriented values of loyalty
			\item[Political System] with one largely long-serving and dominant party, described as many as a benevolent dictatorship
		\end{description}
	\item[Globalised] implies that points should be applicable across physical and political boundaries, hence examples should either be applicable to many countries at once or should be sourced from multiple different countries, opposite that of \textbf{Your Society}. Pulling examples which directly involve increased cross-boundary flows of humans and knowledge are best.
	\item[Modern/Today] limits the key scope to the near past. Points need to have some involvement with recent global trends, specifically that of technology, globalization and so on. Points involving the internet, travel, economic growth, global trends are good to tap on.
	\item[Quality of life] has the dictionary definition of the standard of health, comfort and happiness experienced by an individual or a group. Do note that when this phrase appears, distinct and solid links between actions and quality of life should be made.
\end{description}

\end{document}