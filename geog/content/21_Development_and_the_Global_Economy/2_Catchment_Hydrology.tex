\documentclass[../../main]{subfiles}

\begin{document}

\section{Catchment Hydrology}

	Catchment Hydrology is the study of the water cycle on land. Catchments, also known as drainage basins, refer to an area of land which directs its water input into a river and tributaries. Once water enters a catchment, it travels through flows and stores before leaving the basin.

\subsection{Drainage Basin Water Balance}

\subsubsection{Inputs}

	\begin{description}
		\item[Precipitation] is the introduction of water into a catchment system through rain, snow or ice from the atmosphere. Humid tropics experience high precipitation (2500-4000mm in the Amazon), evenly distributed all year round in intense storms of 10cm/hr to 15cm/hr. Arid tropics experience highly variable rainfall with less than 250mm of precipitation a year across a few concentrated storms which typically last no longer than 15 minutes with a high intensity of 5mm/min.
		\item[Ice Melt] is the introduction of water into a catchment due to upstream melt in frozen ice. This is typically due to a change in climatic conditions upstream where seasonal changes raise the temperature in these regions and melt ice, after which the water is contributed to its streams.
	\end{description}

\subsubsection{Stores}

	\begin{description}
		\item[Interception Storage] is where water is captured by the canopy layer (comprising leaves and trees) and the litter layer (comprising dropped leaves and other biological matter). Water is introduced into the store during precipitation which enters these layers before being able to proceed, after which the water in these stores are typically lost to evaporation.
		\item[Biological Storage] is where water is captured and absorbed within plant mass.
		\item[Soil Moisture Storage] is where water is captured in a thin film which sticks to soil particles. Soil moisture is characterized by 3 states. In decreasing water level, right after a storm the soil is flooded and saturated with water, after which the soil loses water to percolation to achieve field capacity where water is held as `capillary water' and finally the remaining water is lost to evaporation and to plants to achieve wilting point as `hygroscopic water'. \\
		The fluctuations of soil moisture storage is generally determined by the climate of the area, where humid tropics generally have soil permanently at field capacity whereas seasonally humid tropics may reach wilting point during dry seasons.
		\item[Groundwater Storage and Aquifers] are the storage of water in rock formations, where water originates from percolation and is typically ejected through baseflow into channels. The water table is the boundary between aquifer and the non-saturated rock above. The level of water table is affected by: \\
			Surface relief where the water table typically follows rises and drops in a surface. \\
			Seasonal fluctuations which affect the total volume of water stored in the aquifer.
		\item[Channel Storage] is the water that is contained within a river channel.
	\end{description}

\subsubsection{Flows}

	\begin{description}
		\item[Interception] is where precipitated water enters interception storage. This is affected by:\\
			Amount of vegetation cover where different amounts of vegetation can absorb from 10-50\% of rainfall. \\
			Intensity of rain where high intensity rainfall shakes water off leaves and reduces interception storage.
		\item[Leaf Drip and Stem Flow] is where interception storage (especially canopy storage) flows downwards to the ground surface.
		\item[Infiltration] is the seeping of water into the soil due to gravity and capillary forces. \\
			The texture of the soil where fine soil and alluvium are less permeable. This is affected by:\\
			The presence of vegetation where roots and fauna tunnel passages for water, organic acids from decaying matter prevent clumping, canopy cover reduces rainsplash action which would otherwise compact the soil and vegetation provides leaf matter which can increase litter interception and retain water to be infiltrated over a longer time. \\
			The intensity of rainfall where a low total precipitation and low intensity will maximize infiltration.
		\item[Percolation] is the downward flow of intercepted water towards the aquifer below.. This is affected by: \\
			Permeability of rock affects the rate of percolation, where primary permeability is the ability of water to pass through pores in the rock and secondary permeability is the ability of water to pass through fractures, joints and bedding planes in rock.
		\item[Throughflow] is where percolated water flows laterally due to a difference in soil permeability above ground, where deeper layers of rock are less permeable due to compression or the presence of clay pans. As such, water is forced laterally and may even return to the surface through return flow.
		\item[Return Flow] is the flow of throughflow back above ground to be converted into overland flow.
		\item[Baseflow] is the flow of water from an aquifer to a channel through the base of a river channel.
		\item[Overland Flow] occurs where water flows above ground towards a channel.
		\begin{description}
			\item[Horton Overland Flow], also known as `infiltration excess overland flow', occurs when the rate of rainfall exceeds the maximum rate at which ground can infiltrate water away, resulting in the excess water flowing elsewhere through channels above ground.
			\item[Saturation Overland Flow] occurs where the water table is raised after prolonged rainfall and causes overland flow to occur. Saturation overland flow first comprises of the return flow which is ejected above ground due to a raised water table, and secondly where rainfall occurs on areas of land which meets the water table (therefore there is 0 infiltration capacity) and is forced to flow above ground.
		\end{description}
	\end{description}

\subsubsection{Output}

	\begin{description}
		\item[River Discharge]  is the flow of water in a river, contributed by direct precipitation (above the river), overland flow (above ground to the river), throughflow (through the side of a river) and baseflow (from below the river). Rivers have variances in temporal distribution:
		\begin{description}
			\item[Perennial Channels] are permanently occupied by flowing water and are typically found in humid tropics, due to a high water table ensuring baseflow year round.
			\item[Ephemeral Channels] are typically dry except for the brief period after a storm and are typically found in arid tropics because little precipitation leads to a low water table which is unable to constantly supply a channel, therefore the water in the channel is primarily obtained from overland flow and throughflow. Water typically infiltrates downwards and also is evaporated downstream, leading to a decrease in river discharge downstream.
			\item[Intermittent Channels] are seasonally occupied with water due to the distinct wet and dry seasons of a seasonal tropic.
		\end{description}
		\item[Evapotranspiration] is the loss of water as a gas, encompassing natural evaporation and plant-assisted transpiration. This is affected by: \\
			Temperature where higher temperatures allow for more evapotranspiration. \\
			Relative humidity which determines how much moisture can be effectively transferred to the air. \\
			Amount of vegetation which determines the rate of transpiration.
	\end{description}

\subsubsection{Water Balance}

	The water balance quantifies the inputs, outputs and stores of a water cycle, where the total input (via precipitation) is translated into evapotranspiration, channel runoff and finally a net gain or loss in storage.

	\[P = E + R \pm S\]

\subsection{Fluvial Processes}

\subsubsection{River Energy}

	When precipitation occurs, the potential energy of water due to its high altitude needs to be expended as it travels throughout the hydrological cycle. High river energy is characterized by a large amount of discharge and a large velocity.

	\begin{description}
		\item[Discharge] \(Q = AV\) \\
			Discharge \(Q\) is the product of cross-sectional area \(A\) and river velocity \(V\). Discharge in humid tropics increase downstream due to increased contribution from tributary channels while discharge in arid tropics decrease downstream due to water loss by infiltration and evaporation. 
		\item[Velocity] \(V = \frac{1.49 R^{\frac{2}{3}} S^{\frac{1}{2}}}{n}\) \\
			Manning's equation for river velocity \(V\) relates velocity to hydraulic radius \(R\), gradient \(S\) and friction coefficient \(n\).
		\item[Channel Slope] \(S = \frac{\text{Height above sea level}}{\text{Channel length}}\) \\
			Steeper slopes have a higher \(S\) value than more gradual slopes and hence have a more concentrated change in potential energy of water. 
		\item[Coefficient of Rougness] \(n\) \\
			Coefficient of roughness increases as channels are rougher, typically ranging from 0.010 to 0.050. This accounts for variances in channel material and amounts of sediment.
		\item[Hydraulic Radius] \(R = \frac{\text{Cross sectional area}}{\text{Wetted Perimeter}}\) \\
			Hydraulic radius accounts for the surface area of the channel, where a larger surface area in contact with water implies greater frictional forces of the channel wall and bed on the fluid and hence reducing velocity. The maximum hydraulic radius of a river is of a semicircular cross-sectional area and one of a larger total volume as well.
	\end{description}

	River energy tends to increase downstream in humid tropics because as the total discharge increases due to tributaries, the hydraulic radius R increases as larger volumes of water are now more efficiently transported. Over long periods of time, the river reaches an equilibrium between its gradient and its sediment load, where now less energy is needed downstream since sediments and surfaces undergo attrition and abrasion and more energy is available to transport load due to a reduction in friction, therefore the river will achieve a concave profile. \\

	Increases in \(R\) and decreases in \(n\) account for the reduction in \(S\) in a graded river. Therefore the river velocity still tends to increase downstream. \\

	Urban planners design drainage channels with high \(R\) and low \(n\) in order to ensure high \(V\) in their drains to drain away water as fast as possible.

\subsubsection{Erosion}

	Erosion is the gradual wear of a surface. Erosion in a channel takes place by 4 main mechanisms.

	\begin{description}
		\item[Hydraulic Action] is the wear of a rock surface from the impact of fluid on the rock.
		\item[Abrasion] is the wear of a rock surface from the impact of moving sediment on the rock.
		\item[Solution] is the dissolution of a rock surface due to chemical reaction with fluid.
		\item[Attrition] is the wear of sediments due to mechanical impact of sediment on sediment.
	\end{description}

\subsubsection{Sediment Transportation}

	Sediment transport is the process where static sediments are put into motion. These sediments are supplied from landslides and sometimes are carried with overland flow into the river during storms. Sediments are eroded when the river energy is sufficient to maintain their motion, otherwise known as when the river velocity meets the critical erosion velocity of the specific sediment size. \\

	Sediments in a river, once eroded, are transported in the river channel by numerous mechanisms.

	\begin{description}
		\item[Traction] is where sediment rolls along the channel bed.
		\item[Saltation] is where sediment bounces along the channel bed.
		\item[Suspension] is where sediment is suspended in turbulent flow.
		\item[Solution] is where sediment is chemically dissolved in fluid.
	\end{description}

	River capacity describes the volume of sediment which a river can transport and grows with the third power of river velocity. River competence describes the largest diameter sediment that can be transported as bed load and grows with the sixth power of river velocity. \\

	Sediment transported downstream tend to increase in volume due to the increased discharge and also decrease in caliber due to attrition and the reduction in gradient which supplies less energy to transport sediment.

\subsubsection{Deposition}

	Deposition is the processes where sediment is taken out of transportation and remains in a static position in the river channel. Deposition takes place when there is a sudden input of load which causes the overloading of the river, such as after a landslide, or where there is a sudden loss of energy such as at a convex bank of a meandering river, at the mouth of a river or when discharge is reduced after a lack of precipitation. \\

	River deposits typically are found on the channel bed (especially at the aftermath of a flood or storm), at the edge of the channel or on a floodplain. \\

\subsubsection{Hjulstrom Curve}

	The Hjulstrom Curve is a plot of the critical erosion velocity (CEV) and the settling velocity  (SV) of sediments against the diameter of a sediment. Both axes are logarithmic in scale. \\

	The graph of CEV is a bitonic graph with a minimum at sediments of 5mm diameter. Smaller sediments are held together by strong electrostatic chemical bonds which require higher velocities to set into motion and are also small enough to be shielded by the layer of laminar flow at the bottom of a channel. Larger sediments are heavier and require more energy to be suspended. \\

	The graph of SV begins at 0.5mm and increases sharply from diameter 0.5mm to 10mm, from which it increases gradually. Sediments smaller than 0.5mm are assumed to be permanently suspended in fluid due to their small mass, and will not settle unless the river experiences a sharp decrease of velocity at a river mouth or a lake. Sediments larger than 10mm require larger amounts of energy to be kept suspended and hence have an increasing SV.

\subsection{Properties of River Channel Morphology}

\subsubsection{Drainage Density}

	The drainage density Dd of a basin is the sum of channel lengths divided by the area of the basin. A more branched basin with many tributaries has a higher Dd than one of the same area with one primary channel. Dd values range from 5km per square km on sandstone to 500km per square km on clay badlands. \\

	Dd is useful to calculate the spacing and concentration of channels in basin and is an indicator of basin runoff which can lead to phenomena like floods. \\

	Dd is affected by:

	\begin{description}
		\item[Long-term changes] , whether a river is yet to achieve equilibrium or as a river is able to undergo headward erosion.
		\item[Rock Type] of the basin which affects infiltration rate.
		\item[Total Precipitation] which determines the amount of water in the channels.
		\item[Vegetation] which further impacts infiltration rate.
	\end{description}

	Dd values may be inaccurate due to seasonal fluctuations of channel lengths in a seasonal tropic and also because of artificially small channel lengths due to the presence of underground limestone channels which cannot be measured sufficiently.

\subsubsection{Graded Rivers and Equilibrium}

	Graded time is in the scale of tens to hundreds of years. Graded time allows for long-term changes in a system to occur. River morphology is the consequence of a graded river, as opposed to a quick-reacting system which responds to floods and precipitation in a matter of days or weeks. \\

	River channels are in a balance of generating energy as water moves towards sea level and expending energy through erosion (vertical and lateral) and sediment transport. In graded time, a river is able to modify its channel slope to create a dynamic equilibrium of the two.

\subsubsection{Arid Tropic Channels}
	
	Channels in arid tropics decrease in discharge downstream due to the significant infiltration of water in the channel. Arid climates also experience low surface infiltration rates which increase its drainage density and lead to the formation of wider rivers. \\

	Channels in arid tropics typically are flat and shallow with coarse sediment, and often form braided rivers. \\

	Channels in arid tropics have increased erosion rates due to a lack of vegetation holding its banks together.

\subsection{Variations in River Channel Morphology}

\subsubsection{Plan Form Variations}

	The plan form of a channel is the top-down view of a channel, which can range from straight to meandering to braided, which generally develop due to variances in discharge and sediment load.

\subsubsection{Meandering Rivers}

	Meandering rivers are channels whose sinuosity ratio exceeds 1 to 1.5. \\

	Meanders form by:

	\begin{enumerate}
		\item Pools with deeper topology and finer sediment and Riffles with higher topology and coarser sediments form in straight channels as a result of varying levels of discharge, leading to a lopsided relief in a river channel.
		\item A thalweg (fastest path of water flow) forms which passes through the pools and is deflected by riffles.
		\item The asymmetrical flow of the thalweg leads to a corkscrew-like helicodial flow which results in high velocities at river banks near to the thalweg and low velocities at banks farther from the thalweg. Helicodial flow moves such that it contacts the concave banks while flowing downward.
		\item High velocity near concave banks lead to increased erosion by hydraulic action, forming a river bluff and eroding the concave bank outwards.
		\item The loss of energy to eroding the concave bank as well as the friction faced when dragged across the channel bed results in a lower velocity at the convex bank which is low enough to deposit sediment, leading to the formation of point bars and extending the convex bank.
		\item Further erosion and deposition leads to the development of a meander.
		\item Extended meanders can possibly erode away the ridge between two meanders, which suddenly reduces the sinuosity ratio and eventually leads to the formation of an oxbow lake.
	\end{enumerate}

\subsubsection{Floodplains}

	Lateral accretion occurs when there is sufficient deposition at a meander's convex bank which eventually leads to the formation of a point bar which merges with the floodplain, therefore extending the floodplain. \\

	Vertical accretion also occurs when meandering rivers are flooded and bankfull discharge onto the floodplain carries sediment which is deposited onto the floodplain to grow the floodplain vertically and also form natural levees around the main channel.

\subsubsection{Braided Rivers}

\subsubsection{Long Profile Variations}

	The long profile of a river channel is the description of its slope with respect to its channel length.

\subsubsection{Headward Erosion}

	Headward erosion is the erosion of a river bank which extends a section of the river upstream. \\

	Headward erosion at springs occur as hydraulic action is concentrated at the point of spring as water forces itself out from underground aquifers onto the surface, leading to a gradual headward recession. This form of headward erosion increases drainage density. \\

	Headward erosion at waterfalls occur as steep vertical drops in a channel lead to concentrated hydraulic action on the channel bed at a specific point, forming a plunge pool which undermines the face of the waterfall, ultimately leading to collapse of rock overhangs and the recession of a waterfall.

\subsubsection{River Rejuvenation}

	River rejuvenation is the process at which a sudden environmental change, usually tectonic or climatic, leads to a increase of river energy and therefore a modification of its long profile.

\end{document}