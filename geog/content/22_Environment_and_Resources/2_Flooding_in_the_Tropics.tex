\documentclass[../../main]{subfiles}

\begin{document}

\section{Flooding in the Tropics}

\subsection{Hydrographs}

	Hydrographs are plots of river discharge at a specific cross-section of a river against time since a storm. 

\subsubsection{Structure of a Hydrograph}

	A hydrograph consists of the following:

	\begin{description}
		\item[Rising Limb] is the section of the hydrograph with a positive gradient, indicating an increase in flow.
		\item[Peak Discharge] is the peak of the hydrograph which indicates the maximum flow after a storm.
		\item[Lag Time] is the time difference between the instant of highest precipitation and peak discharge.
		\item[Receeding Limb] is the section of the hydrograph with a negative gradient, indicating a decrease in flow.
		\item[Stormflow] is the area under the hydrograph which arises due to water arriving from overland flow, direct precipitation and tributaries.
		\item[Baseflow] is the area under the hydrograph which arises due to water entering from the water table.
	\end{description}

\subsubsection{Comparing Hydrographs}

	Hydrographs with steep limbs, short lag time and a high peak are considered to be flashy. Floods which occur as a result of storms with flashy hydrographs hence lead to the term `Flash Floods'. Hydrographs with more gentle limbs, longer lag time and a lower peak are considered to be attenuated.

	Hydrographs can also have varying levels of discharge before and after a storm. Hydrographs that do not have any discharge before and a long time after the storm can be considered as ephemeral or intermittent.

\subsubsection{Factors Influencing Hydrographs}

	\begin{description}
		\item[Location of Storm] where gauging stations further from the source of precipitation tend to have a more attenuated hydrograph and a longer lag time than stations closer to the source of precipitation.
		\item[Vegetation] causes a more attenuated hydrograph due to interception and biological storage reducing the total discharge into rivers and also increases the amount of baseflow due to increased interception rates.
		\item[Type of Precipitation] affects 
		\item[Basin Morphology]
		\item[Basin Geology]
	\end{description}

\subsection{Flooding in the Humid and Arid Tropics}

\subsubsection{Causes of Floods}

	\begin{description}
		\item[Precipitation]
		\item[Snow and Ice Melt]
		\item[Volcanic Action]
		\item[Landslides]
		\item[Dam Failure]
	\end{description}

\subsubsection{Intensifiers of Floods}

	\begin{description}
		\item[Urbanization]
		\item[Deforestation]
		\item[Agriculture]
	\end{description}

\subsubsection{Impacts of Floods}

	Primary Impacts are the direct consequences of a natural disaster. In the case of floods, this encompasses:

	\begin{itemize}
		\item Increased sediment transport, possibly even large objects like cars and debris and also pollutant substances
		\item Deposition of large debris after flood
		\item Increased erosion, possibly damaging man-made structures like bridges
		\item Water inundation and damage
		\item Drowning and crop damage
	\end{itemize}

	Secondary Impacts are the indirect consequences of a natural disaster. In the case of floods, this encompasses:

	\begin{itemize}
		\item Pollution of water supply
		\item Disruption of electricity and gas supply
		\item Disruption in public utilities such as transport and food
	\end{itemize}

	Tertiary Impacts are the long-term consequences of a natural disaster. In the case of floods, this encompasses:

	\begin{itemize}
		\item Change in river morphology
		\item Destruction of wildlife habitat
		\item Destruction of arable land
		\item Poverty and economic disruption, especially tourism
		\item Increase in cost of living, insurance rates
		\item Corruption after managing relief funding.
	\end{itemize}

\subsubsection{Predicting Floods}

	Flood prediction is the statistical study of flood occurrence to obtain a probability and frequency of flooding. \\

	Flood forecasting is the collection of meteorological data as an early detection system of floods. \\

	Hazard mapping is the modeling of floods, from which data obtained is used for urban planning and other strategies.

\subsubsection{Mitigating Floods}

	Structural approaches or `Hard Engineering' involve the construction of physical structures to mitigate floods.

	\begin{description}
		\item[Artificial Levees] involve the construction of concrete barriers at river banks, reinforcing the strength of the bank to prevent erosion as well as increasing the height of the channel, hence increasing the threshold of bankfull discharge by increasing cross sectional area. \\
		\geocase{Mississippi levee breaks and The Great Flood of 1993}{Artificial levee systems in the Mississippi were common in the late 20th century to prevent overflow into housing areas. However, 4 to 7 times the normal precipitation in 1993 was observed and was followed by a massive flood, covering a total of 830 thousand \si{\square\km}, ultimately costing \$20 billion USD in damages. Most of 700 privately built levees broke, on top of numerous levee breaks near cities which threatened housing, commercial and cultural areas.}
		\item[Dams]
		\item[Channelization]
	\end{description}

	Non-structural approaches or `Soft Engineering' involve the use of intangible solutions to mitigating floods, usually through legislation and urban planing.

	\begin{description}
		\item[Floodways]
		\item[Floodplain Zoning]
		\item[Building Codes]
		\item[Buyout Programs]
		\item[Catchment Management]
		\item[Flood Insurance]
	\end{description}

\subsubsection{Flood Response}

\subsubsection{Floods in Arid Tropics}

\end{document}