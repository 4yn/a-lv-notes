\documentclass[../../main]{subfiles}

\begin{document}

\section{Flooding in the Tropics}

\subsection{Hydrographs}

\subsubsection{Structure of a Hydrograph}

\subsubsection{Comparing Hydrographs}

\subsection{Flooding in the Humid and Arid Tropics}

\subsubsection{Causes of Floods}

	\begin{description}
		\item[Precipitation]
		\item[Snow and Ice Melt]
		\item[Volcanic Action]
		\item[Landslides]
		\item[Dam Failure]
	\end{description}

\subsubsection{Intensifiers of Floods}

	\begin{description}
		\item[Urbanization]
		\item[Deforestation]
		\item[Agriculture]
	\end{description}

\subsubsection{Impacts of Floods}

	Primary Impacts are the direct consequences of a natural disaster. In the case of floods, this encompasses:

	\begin{itemize}
		\item Increased sediment transport, possibly even large objects like cars and debris and also pollutant substances
		\item Deposition of large debris after flood
		\item Increased erosion, possibly damaging man-made structures like bridges
		\item Water inundation and damage
		\item Drowning and crop damage
	\end{itemize}

	Secondary Impacts are the indirect consequences of a natural disaster. In the case of floods, this encompasses:

	\begin{itemize}
		\item Pollution of water supply
		\item Disruption of electricity and gas supply
		\item Disruption in public utilities such as transport and food
	\end{itemize}

	Tertiary Impacts are the long-term consequences of a natural disaster. In the case of floods, this encompasses:

	\begin{itemize}
		\item Change in river morphology
		\item Destruction of wildlife habitat
		\item Destruction of arable land
		\item Poverty and economic disruption, especially tourism
		\item Increase in cost of living, insurance rates
		\item Corruption after managing relief funding.
	\end{itemize}

\subsubsection{Predicting Floods}

	Flood prediction is the statistical study of flood occurrence to obtain a probability and frequency of flooding. \\

	Flood forecasting is the collection of meteorological data as an early detection system of floods. \\

	Hazard mapping is the modeling of floods, from which data obtained is used for urban planning and other strategies.

\subsubsection{Mitigating Floods}

	Artifical levees \\

	Dams \\

	Channelization \\

	Non-structural

	\begin{description}
		\item[Floodways]
		\item[Floodplain Zoning]
		\item[Building Codes]
		\item[Buyout Programs]
		\item[Catchment Management]
		\item[Flood Insurance]
	\end{description}

	Flood response

\subsubsection{Floods in Arid Tropics}

\end{document}