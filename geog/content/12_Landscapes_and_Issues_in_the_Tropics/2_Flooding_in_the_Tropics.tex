\documentclass[../../main]{subfiles}

\begin{document}

\section{Flooding in the Tropics}

\subsection{Hydrographs}

	Hydrographs are plots of river discharge at a specific cross-section of a river against time since a storm. 

\subsubsection{Structure of a Hydrograph}

	A hydrograph consists of the following:

	\begin{description}
		\item[Rising Limb] is the section of the hydrograph with a positive gradient, indicating an increase in flow.
		\item[Peak Discharge] is the peak of the hydrograph which indicates the maximum flow after a storm.
		\item[Lag Time] is the time difference between the instant of highest precipitation and peak discharge.
		\item[Receeding Limb] is the section of the hydrograph with a negative gradient, indicating a decrease in flow.
		\item[Stormflow] is the area under the hydrograph which arises due to water arriving from overland flow, direct precipitation and tributaries.
		\item[Baseflow] is the area under the hydrograph which arises due to water entering from the water table.
	\end{description}

\subsubsection{Comparing Hydrographs}

	Hydrographs with steep limbs, short lag time and a high peak are considered to be flashy. Floods which occur as a result of storms with flashy hydrographs hence lead to the term `Flash Floods', typically referring to floods which last no longer than 6 hours. Hydrographs with more gentle limbs, longer lag time and a lower peak are considered to be attenuated.

	Hydrographs can also have varying levels of discharge before and after a storm. Hydrographs that do not have any discharge before and a long time after the storm can be considered as ephemeral or intermittent.

\subsubsection{Factors Influencing Hydrographs}

	\begin{description}
		\item[Location of Storm] where gauging stations further from the source of precipitation tend to have a more attenuated hydrograph and a longer lag time than stations closer to the source of precipitation.
		\item[Vegetation] causes a more attenuated hydrograph due to interception and biological storage reducing the total discharge into rivers and also increases the amount of baseflow due to increased interception rates.
		\item[Type of Precipitation] affects the flashiness of a hydrograph. High intensity rainfall create more flashy hydrohgraphs when comparing between arid and humid tropics. Large amount of rainfall among different humid tropics also lead to flashier hydrographs.
		\item[Type of Climate] affects the amount of water discharge from ice and snow melt, especially in catchment basins which originate from glaciers and mountains. Higher temperatures also increase the amount of water lost to transpiration, leading to more attenuated hydrographs.
		\item[Basin Morphology] affects hydrographs by determining the total amount of discharge and the speed of water flow. Basins with a smaller area catch less water but more rapidly to form flashy hydrographs. Steep slopes increase channel velocity and a high drainage density also increases the speed of which water flows (since overland flow is faster than throughflow), creating flashier hydrographs.
		\item[Basin Geology] affects the permeability and therefore the flows and stores of water in the litosphere. Permeable and dry soil and permeable rocks (chalk, limestone and slake vs clay, slit and granite) allow for increased interception rates and create more attenuated hydrographs, which is compounded when present in deep layers.
		\item[Human Influence] Deforestation negates the influence of vegetation and tends towards flashier hydrographs. Urban development and construction decreases the permeability of ground due to the widespread use of concrete as well as channelization of drainage systems, leading to flashier hydrographs. Poor agricultural activities such as trampling and poor soil maintenance also leads to flashier hydrographs.
	\end{description}

\subsection{Flooding in the Humid and Arid Tropics}

\subsubsection{Causes of Floods}

	\begin{description}
		\item[Precipitation] causes the occurrence of floods by providing the raw increase in input to the catchment basin hydrological cycle. Periods of increased precipitation can occur due to climatic conditions (humid vs arid), temperate seasons, monsoon seasons and tropical cyclones.

		\geocase{Singapore: Orchard Road Floods}{Changes in weather patterns have created periods of high rainfall intensity in the June to July months, despite the conditions from the dry monsoon. Within the Orchard Road area alone, June 2010 had precipitation of more than 100mm while June 2011 had 124mm of precipitation, more than half of which in half an hour. The former was primarily due to a clog in the road's drainage system while the latter was primarily due to sheer amount of rainfall.}

		\item[Snow and Ice Melt] which arise due to seasonal changes in temperature cause the sudden melting of ice and snow storages which are then channeled downstream. This phenomena is commonly observed in mountain ranges where there are larger storages in frozen water, especially the Himalayas.

		\geocase{Bangladesh}{The combined effect of monsoon rains during the months from June to October where 80\% of its annual precipitation is received, on top of occasional cyclones. Additionally, snow melt from Tibetan region and the Himalayas in later months flow from the river Ganges and the river Brahmaputra, together with additional input from river Meghna. Finally, the topology of Bangladesh also has 80\% of its land mass on a floodplain. As a result, 75\% of its land has been flooded at least once before and 17\% of its land is flooded annually.}

		\item[Volcanic Action]

		\item{Nevado Del Ruiz}{Volcanic Action}

		\item[Landslides] can cause a sudden input of sediment into a water body and suddenly displace large amounts of water to create a flood peak.

		\geocase{Italy: Mt Vajont}{The Vajont/Vaiont dam was constructed in the 1960s to meet power generation needs and also as a checkpoint to manage water between downstream channels. However, as the dam was still yet to reach peak volume in 1963 a massive landslide of 260mil cubic meters of matter was injected into the water body within 45 seconds, causing a 50mil cubic meter displacement of water over the dam in a 250m high wave, ultimately leading to the failure of the dam and also the death of almost 2000 Italians.}

		\item[Dam Failure] occurs when constructed dams face leaks and breakages, leading to the sudden outflow of water originally stored upstream of the dam.

		\geocase{Iraq: Mosul Dam}{Mosul Dam is the largest dam in Iraq, located upstream of Mosul. However, it requires 24 machines constnatly pumping concrete to reinforce the dam and has been touted the `most dangerous dam in the world'. Simulations of dam failure estimate a death toll of 500 thousand Iraqis across Mosul and Baghdad. Additional worries arise espeially due to terrorist threats and instability in the region, especially when ISIS took territorial control over the dam in 2014 and halted maintenance.}

	\end{description}

\subsubsection{Intensifiers of Floods}

	\begin{description}
		\item[Urbanization] leads to a decrease in interception rate due to the liberal usage of concrete and similar man-made structures, causing hydrographs to be more flashy due to the larger amount of water in a channel being supplied through overland flow. Additionally, reinforcing of drainage channels and construction of new channels can also speed up drainage due to channelization, leading to a even more flashy hydrograph.

		\item[Deforestation] decreases the vegetation present in a large land area, typically due to urbanization, land clearing efforts or for agriculture. However, this typically leads to the creation of flashier hydrographs due to a decrease of surface area for evapotranspiration and also the decrease in permeability of the soil, increasing the drainage density of a catchment basin and potentially leading to floods and landslides.

		\item[Agriculture]

		\geocase{Malaysia}{Exceptionally strong Northeast monsoons in 2014 caused extensive flooding in multiple areas of Malaysia, the worst in 30 years. Precipitation up to 250mm in 24 hours led to more than 200k displaced across both peninsular and east Malaysia. The prime aggravator of Malaysia's troubled floods was blamed on excessive logging, with more than 10\% of canopy cover being lost within 2 years in states that were flooded, typically to be used to sell timber (\$4.58bn industry) and for oil palm plantations.}

	\end{description}

\subsubsection{Impacts of Floods}

	Primary Impacts are the direct consequences of a natural disaster. In the case of floods, this encompasses:

	\begin{itemize}
		\item Increased sediment transport, possibly even large objects like cars and debris and also pollutant substances
		\item Deposition of large debris after flood
		\item Increased erosion, possibly damaging man-made structures like bridges
		\item Water inundation and damage
		\item Drowning and crop damage
	\end{itemize}

	Secondary Impacts are the indirect consequences of a natural disaster. In the case of floods, this encompasses:

	\begin{itemize}
		\item Pollution of water supply
		\item Disruption of electricity and gas supply
		\item Disruption in public utilities such as transport and food
	\end{itemize}

	Tertiary Impacts are the long-term consequences of a natural disaster. In the case of floods, this encompasses:

	\begin{itemize}
		\item Change in river morphology
		\item Destruction of wildlife habitat
		\item Destruction of arable land
		\item Poverty and economic disruption, especially tourism
		\item Increase in cost of living, insurance rates
		\item Corruption after managing relief funding.
	\end{itemize}

\subsubsection{Predicting Floods}

	Flood prediction is the statistical study of flood occurrence to obtain a probability and frequency of flooding. One such mechanism would be through studying the recurrence interval of floods, which is obtained by sorting years by the year's record daily precipitation and then plotting a best fit graph of record daily precipitation against \(\frac{\text{total years considered}}{\text{placing of this year}}\). Accuracy of this frequency estimation is improved when larger data sets are obtained, optimally in a range of 500 years.

	\geocase{New Orleans: Flood Prediction and Preparation}{After facing Hurricane Katrina in 2005, urban planners in New Orleans had concluded that equipping the city with levees (50 were breached) and pumping stations to account for a category 3 hurricane were insufficient. Current efforts in rebuilding has accounted for the incidence of 100-year and 500-year flood.}

	Flood forecasting is the collection of meteorological data as an early detection system of floods. However, the use of forecasting is slightly less useful for flash floods which are undetectable long before rainfall, especially in arid regions.

	\geocase{Singapore: Orchard Road Floods}{150 storm drain sensors were installed in major canals to be used as a pre-warning system. However, warnings had failed before, such as in the 2011 Orchard Road floods where insufficient predictive technology failed to notify mall operators.}

	Hazard mapping is the modeling of floods, from which data obtained is used for urban planning and other strategies.

\subsubsection{Mitigating Floods}

	Structural approaches or `Hard Engineering' involve the construction of physical structures to mitigate floods.

	\begin{description}
		\item[Artificial Levees] involve the construction of concrete barriers at river banks, reinforcing the strength of the bank to prevent erosion as well as increasing the height of the channel, hence increasing the threshold of bankfull discharge by increasing cross sectional area.

		\geocase{Mississippi: The Great Flood of 1993}{Artificial levee systems in the Mississippi were common in the late 20th century to prevent overflow into housing areas. However, 4 to 7 times the normal precipitation in 1993 was observed and was followed by a massive flood, covering a total of 830 thousand \si{\square\km}, ultimately costing \$20 billion USD in damages. Most of 700 privately built levees broke, on top of numerous levee breaks near cities which threatened housing, commercial and cultural areas.}

		\item[Dams] help to limit the flow of water at crucial cross-sections of a channel system by installing a mechanical barrier to limit water flow downstream, therefore lowering the flood peak. However, dams also lead to ecological detriments as upstream flow (for fish) is blocked, sedimentation at the upstream portion of the dam occurs and continuous maintenance may be needed.

		\geocase{Yangtze China: Three Gorges}{The Three Gorges dam is 2.3km wide and 185 m high and was built due to seasonal floods which would kill hundreds of thousands of Chinese while displacing many more and could potentially flood downstream cities of Nanjing, Wuhan and Shanghai. The dam is built to change the magnitude of 10-year floods to 100-year floods, and was successful in protecting cities against the 2010 South China Floods. Power generation from the dam is also sufficient to supply almost 1 20th of China's energy demands, the same as 20 nuclear plants. However, the 1.2 million displaced due to the inundation of upstream land exacerbated poverty situations in the area. }

		\item[Channelization] is the reinforcement and straightening of drainage systems, natural or man-made, through paving with concrete and other materials. This decreases the friction coefficient in a drainage channel and also eliminates any erosion and sediment transport in a channel, leading to a overall increase in channel velocity. This helps to channel away water as fast as possible from a urbanized area and prevent flooding.

		\geocase{Singapore: Orchard Road Floods}{\$25k was spent on debris-trapping gates in the Stamford Canal while \$26mil was spent on elevating 1.4km of the Orchard Road street in response to the 2010 and 2011 floods.}
	\end{description}

	Non-structural approaches or `Soft Engineering' involve the use of intangible solutions to mitigating floods, usually through legislation and urban planing.

	\begin{description}
		\item[Floodways]

		\geocase{Canada: Red River Floodway}{The Red River of the North is a river system that flows through the Canadian city of Winnipeg. High seasonal rainfall in the northern region, due to the sheer size of its catchment basin, lead to occasional overflows in this river. In order to prevent floods from affecting the city, a man-made river to the east of the city zone was constructed to act as an alternative channel for water to flow by if too much discharge could flow into the city. Recent floods in 2011, 2009 and the record-breaking 1997 flood have effectively protected the city from flooding, in 1997 flood waters reached within inches of city barriers. In total, the floodway has saved Canada an estimated \$75bn in prevented flood damages.}
		
		\item[Floodplain Zoning]
		\item[Building Codes]
		\item[Buyout Programs]
		\item[Catchment Management]
		\item[Flood Insurance]
	\end{description}

\subsubsection{Flood Response}

\subsubsection{Floods in Arid Tropics}

\end{document}