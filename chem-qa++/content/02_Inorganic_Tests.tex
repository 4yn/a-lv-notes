\documentclass[../main]{subfiles}

\begin{document}

\section{Tests for Inorganic Compounds}

	\subsection{Tests for Gases}

	\begin{description}
		\item[\ch{O2}] relights a glowing splint.
		\item[\ch{H2}] extinguishes a lit splint with a `pop' sound.
		\item[\ch{CO2}] forms white ppt of \ch{Ca(OH)2} when bubbled into limewater \ch{Ca(CO3) (aq)}.
		\item[\ch{SO2}] decolorizes acidified purple \ch{KMnO4} filter paper.
		\item[\ch{NO2}] is a brown, pungent gas which (oxidizes?) \ch{FeSO4} solution, turning green solution brown.
		\item[\ch{NH3}] is a pungent gas which turns moist red litmus paper blue.
	\end{description}

	\subsection{Tests and Reactions for Inorganic Compounds, by Identity}

		\subsubsection{Anions}

		\begin{description}
			\item[\ch{NO3-}] produces \ch{NH3} when heated in \ch{NaOH} with \ch{Al (s)}.
			\item[\ch{NO2-}] produces \ch{NH3} when heated in \ch{NaOH} with \ch{Al (s)} and produces \ch{NO2} when reacted with \ch{HCl}.
			\item[\ch{CO3^{2-}}] shows effervescence when reacted with acid to give \ch{CO2}.
			\item[\ch{SO4^{2-}}] produces a white ppt of \ch{BaSO4}, insoluble in acid.
			\item[\ch{SO3^{2-}}] produces a white ppt of \ch{BaSO3}, which dissolves in acid to produce \ch{SO2}.
		\end{description}

		\subsubsection{Halide Anions}

		\begin{description}
			\item[\ch{I-}] produces yellow ppt with \ch{AgNO3}, insoluble in \ch{NH3 (aq)}.
			\item[\ch{Br-}] produces cream ppt with \ch{AgNO3}, insoluble in \ch{NH3 (aq)}.
			\item[\ch{Cl-}] produces white ppt with \ch{AgNO3}, soluble in \ch{NH3 (aq)} due to the formation of the diamminesilver complex.
		\end{description}

		\subsubsection{Cations}

		\begin{tabular}{|r|p{5cm}|p{5cm}|p{5cm}|}
			\hline
			             & Soln       & \ch{NaOH}               & \ch{NH3 (aq)}              \\ \hline
			\multicolumn{4}{|c|}{White ppts}                                        \\ \hline
			\ch{Ba^{2+}} & -          & -                       & -                          \\ \hline
			\ch{Ca^{2+}} & -          & White                   & -                          \\ \hline
			\ch{Zn^{2+}} & -          & White, soluble          & White, soluble             \\ \hline
			\ch{Al^{3+}} & -          & White, soluble          & White                      \\ \hline
			\ch{Mg^{2+}} & -          & White                   & White                      \\ \hline
			\multicolumn{4}{|c|}{Colored ppts}                                      \\ \hline
			\ch{Cu^{2+}} & Blue       & Blue                    & Blue, soluble to deep blue \\ \hline
			\ch{Cr^{3+}} & Green      & Green, soluble to green & Green                      \\ \hline
			\ch{Fe^{2+}} & Pale Green & Dirty Green             & Dirty Green                \\ \hline
			\ch{Fe^{3+}} & Brown      & Brown                   & Brown                      \\ \hline
			\ch{Mn^{2+}} & -          & Off-white               & Off-white                  \\ \hline
		\end{tabular}

		\begin{description}
			\item[\ch{CuCO3}] can be green or blue depending on its concentration.
			\item[\ch{Fe^{2+}} and \ch{Fe^{3+}}] react with \ch{Fe(CN6)^{3-}} to form a deep blue ppt of \ch{Fe4[Fe(CN6)]3} (turnbull's ppt).
			\item[\ch{Fe^{3+}}] reacts with \ch{SCN-} to form a blood red coloration.
		\end{description}

	\subsection{Tests for Inorganic Compounds, by Reactants and Conditions}

\end{document}
