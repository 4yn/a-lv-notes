\documentclass[../main]{subfiles}

\begin{document}

\section{Intro to Organic}

	\scidef{Organic Chemistry}{Organic Chemistry is the study of chemistry with respect to molecules involving Carbon and Hydrogen.}

	\subsection{Carbon in Organic Chemistry}

	Carbon \ch{_6 C} is able to form many compounds because of its properties when having 4 electrons in its valence shell.

	\begin{itemize}
		\item 4 covalent bonds can be formed
		\item All the electrons in a fully bonded Carbon are bond pairs
		\item Single, double and triple bonds can be formed
	\end{itemize}

	Organic compounds are stable and unreactive because:

	\begin{itemize}
		\item \ch{C-H} and \ch{C-C} / \ch{C=C} / \ch{C+C} are strong
		\item Carbon has no low-lying vacant orbitals / cannot expand octet and hence cannot form extra bonds
		\item Skeletal carbon have no lone pairs for reaction
	\end{itemize}

	Carbon causes varied geometries in organic molecules:

	\begin{itemize}
		\item \ch{sp^3} hybridized carbon occurs with 4 single bonds with a tetrahedral structure and a 109.5\degree angle.
		\item \ch{sp^2} hybridized carbon occurs with 3 bonds (2 single and 1 double) or in benzene rings (delocalised \(\pi\) \ch{e- cloud}) with a trigonal planar structure and a 120\degree angle
		\item \ch{sp} hybridized carbon occurs with 2 bonds (two double / 1 single and 1 triple) with a linear structure and a 180\degree angle
	\end{itemize}

	\subsection{Functional Groups}

	\scidef{Functional Group}{A Functional Group comprises an atom or a group of atoms to be bonded to the organic compound which determines its chemical properties.}

	\scidef{Homologous Series}{A Homologous series is a family of compounds with the same functional group.}

	Homologous series usually share chemical and physical properties, and therefore can be prepared by similar processes. Homologous series may differ with a member through the addition of a \ch{CH2} group and its chemical formula follows a general pattern.

	\subsection{Structural Chemical Formula}

	In addition to empirical and molecular formula as discussed in Stoichiometry, organic chemistry now requires the understanding of structural formula due to the large size and convoluted bonding of organic molecules.

	\scidef{Condensed Structural}{Condensed Structural formula describes each \ch{C} in the skeletal chain and what groups are bonded to it. For example, \ch{CH2CH1CH3} for but-1-ene.}
	\scidef{Displayed / Full Structural}{Full Structural formula describes each atom using labels, what bonds form between which atoms  and also the arrangement of groups around carbon (4 groups drawn in orthogonal directions, 3 groups in trigonal).}
	\scidef{Stereochemical}{Stereochemical formula describes each atom using labels and the 3-dimensional distribution of bonded molecules. Solid lines are parallel to the plane of the paper, wedged bonds move out of the paper while hashed bonds move into the paper.}
	\scidef{Skeletal}{Skeletal formula describes primarily all atoms which are not carbon or hydrogen in a compound. Lines representing bonds are drawn in a imaginary hexagonal grid with each unlabeled vertice representing a carbon, functional groups are drawn and finally all vacant electron pairs are implicitly filled with hydrogen. Multiple bonds are drawn as multiple parallel lines.}

	Cyclic carbon are drawn by their skeletal formula with a line from the cycle to whatever other miscellaneous groups it is bonded to, and all C or H are implicit.

	\subsection{Functional Groups and Nomenclature}

	To name an organic compound:

	\begin{enumerate}
		\item Identify the functional group 
		\item Identify the longest C chain which includes the functional group.
		\item Number Cs from one end to other
		\item Identify side-chains, add di/tri prefixes
		\item Order prefixes alphabetically, ignoring di/tri prefixes
		\item Combine prefix, root, multiple bond infixes and suffix.
	\end{enumerate}

	A chain must either not include cyclic C or must all be cyclic. Number of C determines the root. If the root is a cycle, cyclo- is added to the front. Root names are: meth, eth, prop, but, pent, hex, hept, oct, non, dec and so on. \\

	\subsubsection{Functional Groups}

	Functional groups are selected (in increasing priority):
	\begin{itemize}
		\item Carboxylic Acids (-oic acid) \ch{COOH}
		\item Carboxylic Acid Derivatives: Esters ([Name of R]-yl [Root]-ate) \ch{COR}, Acid Halides (-oyl [halide]-ide) \ch{COX}, Amide (-amine) \ch{CONH2}
		\item Nitrile (-nitrile) \ch{C+N}
		\item ``Oxygen groups'' : Aldehyde (-al) \ch{OH} Ketone (-one) \ch{O} , Alcohol (-ol) \ch{-O-H}
		\item Amine (-amine) \ch{NH2}
	\end{itemize}

	Other functional groups not in this list include:

	\subsubsection{Tiebreakers}

	Candidate carbon chains and C numbers are allocated by (in increasing priority):
	\begin{itemize}
		\item Minimizing the number of the functional group
		\item Minimizing the number of the multiple bonds
		\item Minimizing the number of the prefixes
		\item Minimizing the number of the alphabetically first prefix
	\end{itemize}

	Prefix names of functional groups are:
	\begin{itemize}
		\item Bromo-/Chloro-/Iodo- for halide groups
		\item Oxo- for aldehyde and ketone groups
		\item Alkoxy- for esters, Hydroxy- for alcohol groups
		\item Cyano- for nitrile groups
		\item Amino- for amine groups, Nitro- for \ch{NO2}
		\item Phenyl- for benzene ring
		\item Methyl-/Ethyl-/Propyl- for alkane groups
	\end{itemize}

	\subsection{Organic Reactions}

	\subsubsection{Organic Species}

	\scidef{Lewis Acids and Bases}{Lewis Acids are particles which readily accept \ch{e-} pairs. Lewis Bases are particles which readily donate \ch{e-} pairs.}

	\scidef{Electrophile}{An Electrophile is a species of particle which is electron-deficient or readily accepts electrons. This arises when the particle has vacant orbitals or has electrons strongly pulled from it such as a protonic H.}
	\scidef{Nucleophile}{A Nucleophile is a species of particle which is electron-rich or readily donates electrons. This arises when the particle has one or more lone pairs of electrons.}

	Electrophiles are attracted to electron rich sites of molecules while Nucleophiles are attracted to electron deficient sites.

	\subsubsection{Reaction Mechanisms}

	\subsubsection{Types of Reactions}

	\subsection{Isomerism}

	\subsubsection{Constitutional Isomerism}

	\subsubsection{Cis-trans Isomerism}

	\subsubsection{Enantiomerism}

\end{document}

