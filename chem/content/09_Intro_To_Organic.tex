\documentclass[../main]{subfiles}

\begin{document}

\section{Intro to Organic}

	\scidef{Organic Chemistry}{Organic Chemistry is the study of chemistry with respect to molecules involving Carbon and Hydrogen.}

	\subsection{Carbon in Organic Chemistry}

	Carbon \ch{_6 C} is able to form many compounds because of its properties when having 4 electrons in its valence shell.

	\begin{itemize}
		\item 4 covalent bonds can be formed
		\item All the electrons in a fully bonded Carbon are bond pairs
		\item Single, double and triple bonds can be formed
	\end{itemize}

	Organic compounds are stable and unreactive because:

	\begin{itemize}
		\item \ch{C-H} and \ch{C-C} / \ch{C=C} / \ch{C+C} are strong
		\item Carbon has no low-lying vacant orbitals / cannot expand octet and hence cannot form extra bonds
		\item Skeletal carbon have no lone pairs for reaction
	\end{itemize}

	Carbon causes varied geometries in organic molecules:

	\begin{itemize}
		\item \ch{sp^3} hybridized carbon occurs with 4 single bonds with a tetrahedral structure and a 109.5\degree angle.
		\item \ch{sp^2} hybridized carbon occurs with 3 bonds (2 single and 1 double) or in benzene rings (delocalised \(\pi\) \ch{e- cloud}) with a trigonal planar structure and a 120\degree angle
		\item \ch{sp} hybridized carbon occurs with 2 bonds (two double / 1 single and 1 triple) with a linear structure and a 180\degree angle
	\end{itemize}

	\subsection{Functional Groups}

	\scidef{Functional Group}{A Functional Group comprises an atom or a group of atoms to be bonded to the organic compound which determines its chemical properties.}

	\scidef{Homologous Series}{A Homologous series is a family of compounds with the same functional group.}

	Homologous series usually share chemical and physical properties, and therefore can be prepared by similar processes. Homologous series may differ with a member through the addition of a \ch{CH2} group and its chemical formula follows a general pattern.

	\subsection{Structural Chemical Formula}

	In addition to empirical and molecular formula as discussed in Stoichiometry, organic chemistry now requires the understanding of structural formula due to the large size and convoluted bonding of organic molecules.

	\scidef{Condensed Structural}{Condensed Structural formula describes each \ch{C} in the skeletal chain and what groups are bonded to it. For example, \ch{CH2CH1CH3} for but-1-ene.}
	\scidef{Displayed / Full Structural}{Full Structural formula describes each atom using labels, what bonds form between which atoms  and also the arrangement of groups around carbon (4 groups drawn in orthogonal directions, 3 groups in trigonal).}
	\scidef{Stereochemical}{Stereochemical formula describes each atom using labels and the 3-dimensional distribution of bonded molecules. Solid lines are parallel to the plane of the paper, wedged bonds move out of the paper while hashed bonds move into the paper.}
	\scidef{Skeletal}{Skeletal formula describes primarily all atoms which are not carbon or hydrogen in a compound. Lines representing bonds are drawn in a imaginary hexagonal grid with each unlabeled vertice representing a carbon, functional groups are drawn and finally all vacant lone electrons are implicitly filled with hydrogen. Multiple bonds are drawn as multiple parallel lines.}

	Cyclic carbon are drawn by their skeletal formula with a line from the cycle to whatever other miscellaneous groups it is bonded to, and all C or H are implicit.

	\subsection{Functional Groups and Nomenclature}

	To name an organic compound:

	\begin{enumerate}
		\item Identify the functional group 
		\item Identify the longest C chain which includes the functional group.
		\item Number Cs from one end to other
		\item Identify side-chains, add di/tri prefixes
		\item Order prefixes alphabetically, ignoring di/tri prefixes
		\item Combine prefix, root, multiple bond infixes and suffix.
	\end{enumerate}

	A chain must either not include cyclic C or must all be cyclic. Number of C determines the root. If the root is a cycle, cyclo- is added to the front. Root names are: meth, eth, prop, but, pent, hex, hept, oct, non, dec and so on. \\

	\subsubsection{Functional Groups}

	Functional groups are selected (in increasing priority):
	\begin{itemize}
		\item Carboxylic Acids (-oic acid) \ch{COOH}
		\item Carboxylic Acid Derivatives: Esters ([Name of R]-yl [Root]-ate) \ch{COR}, Acyl Halides (-oyl [halide]-ide) \ch{COX}, Amide (-amine) \ch{CONH2}
		\item Nitrile (-nitrile) \ch{C+N}
		\item ``Oxygen groups'' : Aldehyde (-al) \ch{OH} Ketone (-one) \ch{O} , Alcohol (-ol) \ch{-O-H}
		\item Amine (-amine) \ch{NH2}
	\end{itemize}

	Other functional groups not in this list include:

	\begin{itemize}
		\item Amino Acids
		\item Halogenarenes
		\item Phenols
	\end{itemize}

	All these functional groups have been drawn in order below.

	\begin{figure}[H]
    \scalebox{.7}{
		\chemfig{R-[:30](=[2]O)-[:-30]OH}
		\chemfig{R-[:30](=[2]O)-[:-30]O-[:30]R}
		\chemfig{R-[:30](=[2]O)-[:-30]X}
		\chemfig{R-[:30](=[2]O)-[:-30]NH_2}
	}
	\end{figure}
	\begin{figure}[H]
    \scalebox{.7}{
		\chemfig{R~N}
		\chemfig{R-[:30](=[2]O)-[:-30]H}
		\chemfig{R-[:30](=[2]O)-[:-30]R}
		\chemfig{R-[:30]-[:-30]OH}
		\chemfig{R-NH_2}
	}
	\end{figure}
	\begin{figure}[H]
    \scalebox{.7}{
		\chemfig{R-[:30](-[2]NH_2)-[:-30](=[6]O)-[:30]OH}
		\chemfig{[:30]**6(--(-X)----)}
		\chemfig{[:30]**6(--(-OH)----)}
	}
	\end{figure}

	\subsubsection{Tiebreakers}

	Candidate carbon chains and C numbers are allocated by (in increasing priority):
	\begin{itemize}
		\item Minimizing the number of the functional group
		\item Minimizing the sum of the prefixes
		\item Minimizing the number of the alphabetically first prefix
	\end{itemize}

	Prefix names of functional groups are:
	\begin{itemize}
		\item Bromo-/Chloro-/Iodo- for halide groups
		\item Oxo- for aldehyde and ketone groups
		\item Alkoxy- for esters, Hydroxy- for alcohol groups
		\item Cyano- for nitrile groups
		\item Amino- for amine groups, Nitro- for \ch{NO2}
		\item Phenyl- for benzene ring
		\item Methyl-/Ethyl-/Propyl- for alkane groups
	\end{itemize}

	\subsection{Organic Reactions}

	\subsubsection{Organic Species}

	\scidef{Lewis Acids and Bases}{Lewis Acids are particles which readily accept \ch{e-} pairs. Lewis Bases are particles which readily donate \ch{e-} pairs.}

	\scidef{Electrophile}{An Electrophile is a species of particle which is electron-deficient or readily accepts electrons. This arises when the particle has vacant orbitals or has electrons strongly pulled from it such as a protonic H.}
	\scidef{Nucleophile}{A Nucleophile is a species of particle which is electron-rich or readily donates electrons. This arises when the particle has one or more lone pairs of electrons.}
	\scidef{Free Radical}{A Free Radical is an electrically neutral species with an unpaired electron.}

	Electrophiles are attracted to electron rich sites of molecules while Nucleophiles are attracted to electron deficient sites.

	\scidef{Degree of Substitution}{The Degree of Substitution of a atom in an organic molecule is determined by how many carbon it is bonded to. The degree of substitution of a hydrogen atom is the degree of the atom which it is bonded to.}

	\subsubsection{Reaction Mechanisms}

	\scidef{Reaction Mechanism}{A Reaction Mechanism describes the movement of electrons and atoms throughout the process of an organic reaction, hence determining the energetics and kinetics of a reaction.}

	\scidef{Homolytic Fission}{Homolytic Fission involves a covalent bond breaking and donating one electron to both previously bonded atoms, creating free radicals. Homolytic Fission is indicated by drawing `Fish-hook' arrows from the bond to the atoms.}

	\scidef{Heterolytic Fission}{Heterolytic Fission involves a covalent bond breaking and donating both electrons to one specific atom, charging both atoms. Heterolytic Fission is indicated by drawing a double-headed arrow from the bond to one atom. The recepient of the electron pair is the more electronegative atom, forming a carbocation or a carboanion.}

	\scidef{Carbocation}{A Carbocation \ch{C+} is a positively charged species of carbon, usually present in intermediate species of reactions involving heterolytic fission.}

	A mechanism of a reaction is classified as one of Free Radical, Nucleophillic or Electrophillic depending on what species other than the organic molecule is reacted with in the slow step of a mechanism.

	\subsubsection{Types of Reactions}

	\scidef{Addition}{Addition Reactions involve the combination of multiple reactants to form one product, typically involving the breakage of the \(\pi\) bond in a double bond.}

	\scidef{Substitution}{Substitution Reactions involve the combination of multiple reactions to form multiple products.}

	\scidef{Elimination}{Elimination Reactions involve the removal of molecules from one reactant to form \(\pi\) bonds.}

	\scidef{Condensation}{Condensation involves the combination of two large molecules together and the removal of a small molecule (\ch{H2O} or \ch{HCL}).}

	\scidef{Hydrolysis}{Hydrolysis involves the breaking of a bond using water.}

	\scidef{Oxidation}{Oxidation involves the removal of \ch{O}, addition of \ch{H}, loss of \ch{e-} or increase in oxidation number.}

	\scidef{Reduction}{Reduction involves the addition of \ch{O}, removal of \ch{H}, gain of \ch{e-} or decrease in oxidation number.}

	\scidef{Rearrangement}{Rearrangement involves the migration of functional groups within a chain of carbon atoms.}

	\subsubsection{Miscellaneous Terminology}

	\scidef{Electronic Effect}{Compounds are electron-withdrawing or electron-donating if they decrease or increase the electron density in a compound respectively.}

	Electronic effect arises due to a difference in electronegativity in atoms which may affect delocalize electrons or a contribution to the electrons or the orbitals constituting a \(\pi\) electron cloud.

	\scidef{Steric Effect}{Steric Effect, or Steric Hindrance, is the phenomena where the reactivity of a atom is reduced because it is bonded to groups which are large enough to prevent a reaction.}

	\subsection{Isomerism}

	\subsubsection{Cis-trans Isomerism}

	\scidef{Cis-trans Isomers}{Cis-trans Isomers contain the same bonding structure but differ in the arrangement of groups due to a restriction in rotation.}

	Cis-trans isomers occur when there are:

	\begin{itemize}
		\item Restricted rotation due to the presence of a \ch{C=C} double bond or a ring structure
		\item For alkenes, different functional groups attached to each carbon.
		\item For ring, different functional groups attached to different carbon.
	\end{itemize}

	By right cis-trans notation should only be used in cases where there are the same two types of groups on each carbon in a alkene, after which cis-trans is used to desribe how the parent carbon chain passes through the double bond and finally cis-trans is deprecated for the IUPAC E-Z notation which is out of syllabus. \\

	Cis-trans isomers have similar but not identical chemical properties and have differing physical properties, due to the polarization of cis isomers and the better packing ability of trans isomers.

	\subsubsection{Enantiomerism}

	\scidef{Enantiomers}{Enantiomers (ee-nan-shi-oh-merh-s) exists where the mirror image of an atom's structure is non-superimposable with its original structure.}

	Enantiomers have similar physical properties except for a noticeable difference in its optical activity, where different enantiomers rotate polarized light an angle specific to a pair of enantiomers. \\

	When drawing enantiomers, draw two structures separated by a plane of symmetry, using wedges and hashes to indicate bonds out of and into the plane of the paper respectively. Orbitals must be in a tetrahedral shape.

	\scidef{Chiral Carbon}{Chiral Carbons have 4 different groups bonded to it, and tend to indicate the presence of an enantiomer.}

	When identifying chiral carbons, eliminate all carbons without 4 bonds (connected to double or triple bond, two bonds in a skeletal formula etc) to speed up process.

	\scidef{Meso Compounds}{Meso Compounds contain chiral centers but also have planes of symmetry and are hence able to produce superimposable mirror images.}

\end{document}

