\documentclass[../main]{subfiles}

\begin{document}

\section{Halogenoalkanes}

	\subsection{Structure of Halogenoalkanes}

	Halogenoalkanes are derivatives of alkanes compounds with \ch{F, Cl, Br, I} groups. Halogenarenes are derivatives of arenes with benzylic \ch{H} replaced by a \ch{F, Cl, Br, I} group. \\

	Halogenoalkanes can be classified according to the degree of substitution of \ch{C} that the halogen is bonded to.

	\subsection{Physical Properties of Halogenoalkanes}

	Halogenoalkanes contain polar \ch{R-X} bonds which increase their melting and boiling point. For compounds with the same \ch{R} group, the molecule with a larger \ch{X} electron cloud has stronger id-id interactions and hence a larger melting and boiling point. \\

	Though they are polar, halogenoalkanes are still poorly soluble in water. Fluoroalkanes and monochloroalkanes are less dense than water while other \ch{R-X} compounds are denser than water.

	\subsection{Formation of Halogenoalkanes}

	\subsubsection{From Alkanes}

	Mono and polyhalogenated products can be formed from alkanes. Refer to `Free Radical Substitution' in Alkanes.

	\subsubsection{From Alkenes}

	Mono and dihalogenated products can be formed from alkenes. Refer to `Electrophilic Substitution' in Alkenes.

	\subsubsection{From Alcohols}

	Refer to `Nucleophilic Substitution' in Hydroxy Compounds.

	\subsection{Reaction of Halogenoalkanes}

	Halogenoalkanes typically undergo nucleophilic substitution reactions.

	The rate of reaction primarily depends on the nature of the \ch{R-X} bond, with weaker bonds reacting faster despite the change in electronegativity, hence rate of reaction increases in the order \ch{C-Cl},\ch{C-Br} and \ch{C-I}. \\

	\subsubsection{Nucleophilic Substitution}

	\scidef{Nucleophile}{A Nucleophile \ch{Nu} is a particle which has at least one pair of free electrons and can act as an electron donor. Nucleophiles are attracted to electron-poor sites.}

	Note that nucleophiles are not necessarily charged. \\

	Nucleophilic substitution can occur through two mechanisms, characterized by the study of these mechanism's kinetic properties.

	\scidef{\usub{S}{N}2}{The \usub{S}{N}2 mechanism of nucleophilic substitution involves two molecules in its slow (and only) step.}

	\usub{S}{N}2 reactions involve the backside attack of a nucleophile to the halogenoalkane at the opposite side of the \ch{C-X} bond, forming an intermediate molecule both bonded to the nucleophile and \ch{X}, after which \ch{X} is ejected. This reaction preserves the enantiomeric quality of a molecule if the original alkane displays enantiomerism about the \ch{C} with \ch{X} due to the spatial requirements of a backside attack.

	\scidef{\usub{S}{N}1}{The \usub{S}{N}1 mechanism of nucleophilic substitution involves molecules in its slow step.}

	\usub{S}{N}1 reactions involve the spontaneous ejection of the \ch{X} group in its slow step to form a carbocation which will then react with a nucleophile. The later addition of the nucleophile will happen at equal probabilities at either side of the trigonal planar \ch{C+}, hence any enantiomeric products about \ch{C} will be racemic. \\

	Only draw stereochemistry of a reaction if the original molecule displays enantiomerism about the \ch{C} bonded to \ch{X}.

	The mechanism at which a \usub{S}{N} reaction occurs depends primarily on the degree of substitution of the \ch{C} bonded to \ch{X}. \ch{C} with more alkyl groups bonded to it will generally follow the \usub{S}{N}{1} mechanism because it will form a more stable \ch{C+} whereas \ch{C} with less alkyl groups bonded to it will follow the \usub{S}{N}2 mechanism due to the free space around \ch{C} leaving it susceptible to backside attack. \\

	\noindent \textbf{Formation of Alcohol}

	Reagents: \ch{NaOH} \\
	Conditions: Aqueous, heat \\

	A polyhalogenoalkane may react to form alcohol groups, which may then alter the reaction environment to give an Alcoholic medium which can then lead to elimination of \ch{HX}. Refer to Elimination Reactions. \\

	\noindent \textbf{Formation of Amines}

	Reagents: Concentrated \ch{NH3} in excess \\
	Conditions: Alcoholic, heat, sealed tube \\

	If \ch{NH3} is not in excess, a primary amine can act as a nucleophile in later reactions and form a chain reaction to eventually form secondary and tertiary amines and finally a quaternary ammonium salt. \\

	\noindent \textbf{Formation of Nitriles}

	Reagents: \ch{KCN} \\
	Conditions: Alcoholic, heat \\

	In practice, a mixture of alcohol and water is used so that all reactants can be dissolved. Nitriles can be further reduced to form primary amines (\ch{H2}, \ch{Ni} and Heat OR \ch{LiAlH4-}, dry ether and heat) or undergo hydrolysis to form carboxylic acids (\ch{H2SO4 (aq)}, heat) and carboxylate ions (\ch{NaOH (aq)}, heat). \\

	\noindent \textbf{Williamson Ether Synthesis}

	Reagents: Alcohol, \ch{Na} OR \ch{R-O- Na+}\\
	Conditions: heat \\

	\subsubsection{Elimination}

	Though the benzene ring itself is unreactive, any alkyl chains bonded to a benzene ring can undergo oxidation, replacing the alkyl group with a \ch{COOH} group. However, the reaction requires that the benzyllic carbon has at least 1 \ch{H} or 1 \ch{O} atom bonded to it, otherwise a case such as a tert-butyl benzene will not be oxidised due to steric effect of \ch{CH3} groups attached to the benzyllic \ch{C}. \\

	\noindent \textbf{Dehydrohalogenation}

	Reagents: \ch{NaOH} \\
	Conditions: Alcoholic, heat \\

	Dehydrohalogenation can only occur if there is a \ch{H} in some \ch{C} adjacent to the \ch{C-X} group. \\

	\subsection{Halogenoarenes}

	Halogenoarenes are compounds with \ch{X} bonded to benzene rings. \\

	Halogenoarenes burn with a sooty flame and are insoluble in and denser than water. \\

	Refer to `Electrophilic Substitution' in Arenes for methods of preparation. \\

	Halogenoarenes generally do not undergo nucleophilic substitution reactions. Extension of the \(\pi\) \ch{e-} cloud over \ch{C-X} causes the formation of a partial double bond character, the benzene ring physically blocks approaching nucleophiles, the \ch{e-} density of the benzene ring repels approaching nucleophiles and the presence of the \ch{e-} cloud prevents the formation of \ch{C+}. However, reactions may occur under extreme conditions (\SI{150}{\atm}, \SI{350}{\celsius}).\\

	Halogenoarenes can also undergo electrophilic substitution, much like other arenes. However, \ch{X} are electron-withdrawing and electronegative, hence they are deactivating groups and the reactivity of the arene decreases as more \ch{X} groups are added. \ch{X} is a 2,4-directing group.

	\subsection{Distinguishing Test}

	\subsubsection{`4-step Test'}

	\begin{enumerate}
		\item Heat with \ch{NaOH (aq)}
		\item Cool
		\item Acidify with \ch{HNO3}
		\item Add \ch{AgNO3}
	\end{enumerate}

	Positive tests will form a \ch{AgX} precipitate. The identity of \ch{X} can be determined according to the color of the precipitate.

	\subsubsection{Ethanolic \ch{AgNO3}}

	\begin{enumerate}
		\item Heat with ethanolic \ch{AgNO3}
	\end{enumerate}

	Ethanol acts as a nucleophile.

	\subsection{Applications of Halogenoalkanes}
	
\end{document}