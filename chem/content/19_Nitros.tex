\documentclass[../main]{subfiles}

\begin{document}

\section{Organic Nitrogen Compounds}

	\subsection{Structure of Organic Nitrogen Compounds}

	Organic Nitrogen Compounds are organic molecules with \ch{N} atoms present. \\

	Amides are molecules with \ch{N} bonded to an aliphatic group, whereas Amines are molecules with a \ch{C=O} group adjacent to \ch{N}, and are classified as so due to their differing properties.

	\subsection{Physical Properties of Organic Nitrogen Compounds}

	Ammonia, Methylamine and Ethylamine are gases at room temperature, while other aliphatic amines with up to 4 \ch{C} as well as Phenylamine are liquid at room temperature. \\

	Amines typically have higher boiling points than other similar alkanes due to their ability to form \ch{H} bonds, but still have lower boiling points than alcohols as the \ch{N-H\bond{dotted}N} bond is weaker than the \ch{O-H\bond{dotted}O} bond. Boiling points of amines decrease from primary to tertiary species due to their decreasing ability to form \ch{H} bonds and their decreasing dipole moment. \\

	Amines are generally soluble in water as they can form \ch{H} bonds with water molecules. Amines with short carbon chains are also volatile, where warming of solutions containing ammonia or short-chain amides will give a pungent odor. \\

	On the other hand, amides are able to form \ch{N-H\bond{dotted}O} bonds between molecules, hence most primary amides other than methanamide are crystalline solids. Primary and secondary amines have high melting and boiling points whereas tertiary amides have much lower boiling and melting points. Amides regardless of order are able to dissolve in water given short hydrophobic carbon chains. \\

	\subsection{Formation of Organic Nitrogen Compounds}

	\noindent \textbf{Halogenoalkanes to Amines}

	Refer to `Nucleophilic Substitution' in Halogenoalkanes. \\

	\noindent \textbf{Nitriles to Amines}

	Refer to `Nitriles' in Halogenoalkanes and in Carbonyl Compounds. \\

	\noindent \textbf{Amides to Amine}

	Reagents: \ch{LiAlH4} in dry ether \\
	Conditions: Room temperature \\

	\ch{C=O} groups are reduced to \ch{-CH2-}. \\

	\noindent \textbf{Nitrobenzene to Phenylamine}

	Reagents: \ch{Sn} with concentrated \ch{HCl}, then aqueous \ch{NaOH} \\
	Conditions: First step reflux, second step room temperature \\

	Nitrobenzene can be reduced and then neutralized to form phenylamine. \\

	\noindent \textbf{Amides to Amine}

	Refer to `Hydrolysis' in Organic Nitrogen Compounds. \\

	\subsection{Reaction of Organic Nitrogen Compounds}

	Amines typically undergo reactions as either bases or nucleophiles. \\

	\subsubsection{Acid-Base Reactions}

	Amines can react with acids to form amine salts which are non-volatile and hence odorless. Amines can be then regenerated through reacting the salt with base such as \ch{NaOH}. \\

	Various amines have differing degrees of basicity, depending on the nature of its side chains, degree of substitution and their physical state. \\

	Basic strength increases from phenylamine to ammonia to amide. Phenylamine is the least basic species as the lone pair of \ch{e-} in \ch{N} is partially disassociated to the \ch{e-} cloud of benzene. Amides are more basic because it has electron donating \ch{e-} groups which can stabilize its conjugate acid. \\

	Amides are completely neutral because the lone pair of \ch{N} is disassociated across the \ch{C=O} double bond, leaving the \ch{e-} pair unavailable to act as a lewis base. \\

	Basic strength increases from primary to tertiary amides in gaseous state as the number of \ch{e-} donating alkyl groups increases. However, in an aqueous state the presence of large methyl groups in tertiary amines cause it to be less basic than primary and secondary amines. \\

	\subsubsection{Nucleophilic Substitution}

	Refer to `Nucleophilic Substitution' in Halogenoalkanes.

	\subsection{Condensation Reactions}

	Most amines undergo condensation reactions with acid halides to form amides. Tertiary amines do not react as they lack the \ch{H} molecule to be condensed. \\

	Reagents: \ch{ROCl} \\
	Conditions: Anhydrous, room temperature \\
	Observations: White fumes of \ch{HCl} \\

	To form amines, a reaction with acyl halides must occur instead of with carboxylic acids as in the latter case an acid-base reaction will occur instead. \\

	\subsection{Reaction of Phenylamine}

	Phenylamine is a strongly activating group to benzene, moreso than hydroxy groups. \\

	\noindent \textbf{Halogenation}

	Reagents: \ch{X2} (\ch{X} \(\in\) \ch{Cl,Br}) \\
	Conditions:  Aqueous, room temperature\\

	Halogenation of phenylamine in a aqueous substrate forms a polysubstituted product. A tribromophenol is an insoluble white precipitate. \\

	To obtain a monosubstituted product, phenylamine needs to be deactivated before reaction.

	Reagents: \ch{CH3COCl}, then \ch{X2} (\ch{X} \(\in\) \ch{Cl,Br}), then \ch{NaOH} \\
	Conditions: Room temperature, then aqueous and room temperature, then aqueous and heat \\

	\noindent \textbf{Nitration}

	Phenylamine will not undergo nitration. Under acidic conditions, phenylamine is protonated into deactivating phenylammonium which will not react.

	\subsection{Hydrolysis}

	Amide bonds between \ch{R-C=O} and \ch{N-H2} are susceptible to breakdown when heated in acidic or alkaline mediums. Acidic hydrolysis produces carboxylic acids and ammonium whereas basic hydrolysis produces carboxyl ions and amines. \\

	Conditions: Acidic or Alkaline, Heat \\

	To differentiate between amines and amides, gentle heating to liberate \ch{NH3} or other amides can be used as a differentiating test. \\

	\subsection{Amino Acids}

	Amino acids are organic molecules with at least one carboxylic acid group and one amine group. They are classified according to the location of the amine relative to the \ch{COOH} group, with n-amino acid referring to the amine on carbon n from \ch{COOH}.\\

	The most important amino acids are the 20 2-amino acids which make up biological proteins. These amino acids generally display optical activity with the exception of amino-ethanoic acid. \\

	\subsubsection{Zwitterions}

	Due to amino acids having weak acid and weak base groups, amino acids change in makeup depending on the pH of its environment. Zwitterions are the form of amino acids which consist one \ch{RNH3+} and one \ch{COO-} group. This dipolar form is its most stable state at its pH, as compared to if the functional groups were in their neutral state. The pH at which the most number of zwitterions are present is also known as its isoelectric point, written pI.\\ 

	By being able to form zwitterions, amino acids generally form crystalline solids with high melting points, are very soluble and have a moderate acidity and basicity as compared to their pure carboxylic acid and pure amine counterparts. \\

	\subsubsection{Amino Acid Separation}

	Through taking advantage of different amino acid's mass and their differing charges at various pH, amino acids can be separated through paper or gel Electrophoresis. A mixture of amino acids with preferably  different pI are placed in the center of a medium with a known pH, after which a voltage is applied on either side of the paper and the different amino acids will migrate according to:

	\begin{itemize}
		\item Their net charge determines which end of the paper it migrates to 
		\item Their varying masses varies the speed which the molecules move
	\end{itemize}

	\subsection{Peptides and Proteins}

	A peptide bond is a specific name for an amide bond between two amino acids formed via condensation reaction. Dipeptides, tripeptides and so on describe molecules with the according number of amino acid components. Molecules of this form with molar mass smaller than \SI{5000}{\gram\per\mol} are called polypeptides, whereas molecules with molar mass larger than this are considered to be proteins. \\

	Polypeptides and proteins can be broken down via hydrolysis. \\

	\subsubsection{Industrial Hydrolysis}

	\noindent \textbf{Complete Hydrolysis}

	Conditions: 6M \ch{HCl (aq)} OR \ch{NaOH (aq)} at \SI{100}{\celsius} for \SI{20}{\hour} in an evacuated tube \\

	\noindent \textbf{Partial Hydrolysis}

	Conditions: \ch{H2SO4 (aq)} OR \ch{HCl (aq)} OR \ch{NaOH (aq)}, heat for few hours \\

	Acid or base catalyzed hydrolysis of proteins breaks down all amide bonds regardless of their adjacent components, creating a mixture of amino acids in their relative concentrations. If the reaction is conducted in less extreme conditions and given less time to react, a collection of partially hydrolyzed peptides can be obtained, from which the structure of the entire protein may be derived from. \\

	\subsubsection{Enzymatic Hydrolysis}

	Various biological enzymes can selectively hydrolyze amide bonds at specific ends of specific amino acids. Common enzymes used include trypsin, chymotrypsin and pepsin. By using two or more enzymes, analysis of the remaining polypeptides can be used to determine the sequence of amino acids in the protein.

\end{document}