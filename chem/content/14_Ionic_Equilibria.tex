\documentclass[../main]{subfiles}

\begin{document}

\section{Ionic Equilibria}

	\subsection{Ionic Product of Water}

	\ch{H2O <=> H+ + OH-}

	\scidef{Ionic Product of Water \(K_w\)}{The product of concentrations of \ch{H+} and \ch{OH-} are always \(K_w\)=\SI{1e-14}{\mol\squared\per\dm\tothe{6}} at \SI{298}{\K}.}

	As a result of this property of the equilibrium between disassociating \ch{H2O}, the concentration of one ion can be calculated once the concentration of the other is known.

	\subsection{pH and pOH}

	\scidef{p Function}{For some quantity \(X\): \[pX = \log_{10}{X}\] }

	\scidef{pH}{pH is a measure of the concentration of \ch{H+} in a solution, calculated as: \[pH = \log_{10}{\text{[\ch{H+}]}}\]}

	\scidef{pOH}{pH is a measure of the concentration of \ch{OH-} in a solution, calculated as: \[pOH = \log_{10}{\text{[\ch{OH-}]}}\]}

	Note that at \SI{298}{\K}, \(pH + pOH = pK_w = 14 \).

	\subsection{Definitions of Acids and Bases}

	\subsection{Strength of Acids and Bases}

	\subsection{Weak Acids and Bases}



	\subsection{Salt Hydrolysis}

	\subsection{Buffer Solutions}


\end{document}