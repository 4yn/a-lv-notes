\documentclass[../main]{subfiles}

\begin{document}

\section{Chemical Bonding II}

	\subsection{Hybridization}

	Electronic Orbital theory states that electrons fill orbitals of the lowest energy first, arising to a situation of electronic configuration \ch{1s2 2s2 2p2} with two lone electrons. However atoms with this configuration readily form 4 bonds, implying the presence of 4 lone electrons. Additionally, all of these bonds are of same length and require the same amount of energy to break, implying that these are of the same energy level. \\

	\scidef{Hybridization}{Hybridization occurs when multiple orbitals combine to form hybrid orbitals with intermediate properties of the orbitals.}

	Atoms with hybrid orbitals increase the number of lone pairs they have in order to form more bonds, creating a net lower energy state. The hybridization state of a atom is determined by the number of regions of electron density it has. \\

	\ch{sp3} hybridized orbitals are formed by the mixing of 1 s and 3 p orbitals, forming 4 sp3 hybrid orbitals and hence 4 regions of electron density arranged in a tetrahedral orientation.

	\ch{sp2} hybridized orbitals are formed by the mixing of 1 s and 2 p orbitals, forming 3 sp2 hybrid orbitals and hence 3 regions of electron density arranged in a trigonal planar orientation.

	\ch{sp} hybridized orbitals are formed by the mixing of 1 s and 1 p orbitals, forming 2 sp hybrid orbitals and hence 2 regions of electron density arranged in a linear orientation.

	\subsubsection{Drawing Hybridized Orbitals}

	\subsection{Resonance}

	\subsubsection{Resonant Structures}

	\subsubsection{Drawing Resonant Structures}

\end{document}