\documentclass[../main]{subfiles}

\begin{document}

\section{Arenes}

	\subsection{Structure of Benzene}

	A Benzene ring is a cyclic chain of 6 carbon atoms which are all \ch{sp^2} hybridized. Each carbon atom has one remaining electron in an unhybridized \ch{p} orbital, which eventually pool together to form a \ch{\(\pi\)-e-}. This resonance structure ensures that the benzene ring is especially stable, providing its aromatic properties. \\

	Benzene rings have been found to have this special property as all \ch{C-C} bonds have been found to have the same bond length and also because the enthalpy of three \ch{C=C} double bonds result in a energy change significantly larger than that of a Benzene ring. As such, reactions involving benzene tend to maintain its resonance/aromatic structure and rarely convert Benzene into Cyclohexene.

	\subsection{Physical Properties of Benzene}

	Benzene is a colorless liquid at room temperature and pressure. It freezes at \SI{5.5}{\celsius} and melts at \SI{80}{\celsius}. Benzene is insoluble in water and less dense than water.

	\subsection{Aromatic Nomenclature}

	Most benzene derivatives involve their functional group's prefix being added to `benzene'. Four unique names need to be memorized: benzaldehyde, phenol, phenylamine and benzoic acid.

	\chfg{
		\chemfig{!\upbenz([:90]-CHO)}
		\chemfig{!\upbenz([:90]-OH)}
		\chemfig{!\upbenz([:90]-NH3)}
		\chemfig{!\upbenz([:90]-COOH)}
	}

	A benzene ring that does not contain the functional group is a `phenyl' (abbreviated \ch{Ph}) group.

	\chfg{
		\chemfig{!\rightbenz([:0]-R([:90]-)-[:270])}
	}

	\subsection{Methylbenzene}

	\scidef{Methylbenzene}{Methylbenzene is the simplest alkylbenzene, characterized by a methyl group attached to a benzene ring. Due to the presence of these two groups, methylbenzene is able to undergo both reactions involving the benzene ring as well as an alkyl chain.}

	\chfg{
		\chemfig{!\rightbenz([:0]-CH3)}
	}

	\scidef{Benzyl Group}{A \ch{C6H5CH2} group is a `benzyl' group.}

	\chfg{
		\chemfig{!\rightbenz([:0]-CH2-R([:90]-)-[:270])}
	}

	\subsection{Reaction of Arenes}

	Arenes are typically reactive due to the presence of its electron-dense \ch{\(\pi\) e-} cloud. However, benzene rings tend to preserve their aromaticity even after reactions due to its unique stability. Arenes will typically undergo electrophilic substitution reactions. \\

	Side chains of benzene rings are typically more reactive than their counterparts without benzene rings as the \ch{\(\pi\) e-} cloud of the benzene ring can disassociate across \ch{C} atoms in the side chain, making them more reactive. Otherwise, side chains can undergo any reactions as if they lacked its benzene ring substituent.

	\subsubsection{Reduction}

	Reagents: \ch{H2}\\
	Conditions: \ch{Ni} + High T + High P 

	\subsubsection{Electrophilic Substitution}

	\noindent \textbf{Nitration}

	Reagents: Concentrated \ch{HNO3} \\
	Conditions: Concentrated \ch{H2SO4}, \SI{55}{\celsius} in benzene, \SI{30}{\celsius} in methylbenzene \\

	Generation of Electrophile: \ch{HNO3 + H2SO4 <=> H2NO3+ + HSO4-}  \\

	\chfg{
		\chemfig{!\upbenz(-[:90]H)(-[@{a1}:270,,,,draw=none])} \arrow{0}[,0] \+ \chemfig{@{a2}\ch{NO3+}} \arrow \chemfig{**[120,420]6(----@{a3}([:60]-H)([:120]-\ch{NO3})(-[:270,,,,draw=none]+)--)}
		\chemmove[shorten <=4pt,shorten >=2pt]{\draw (a1)..controls +(90:5mm) and +(135:10mm) ..(a2); }
	}

	\chfg{
		\chemfig{**[120,420]6(----(-[@{a1}:60]@{a2}H)([:120]-\ch{NO3})(-[@{a3}:270,,,,draw=none]+)--)} \arrow{0}[,0] \+ \chemfig{@{a4}\lewis{3:,\ch{O-}}-S(=[:90]O)(=[:270]O)-OH} 
		\chemmove[shorten <=2pt,shorten >=2pt]{\draw (a1)..controls +(330:4mm) and +(30:4mm) ..(a3); }
		\chemmove[shorten <=6pt,shorten >=2pt]{\draw (a4)..controls +(135:5mm) and +(300:5mm) ..(a2); }
	}

	\chfg{
		\arrow \chemfig{!\upbenz(-[:90]\ch{NO3})} \arrow{0}[,0] \+ \ch{H2SO4}
	}

	\noindent \textbf{Halogenation}

	Reagents: \ch{X2} (\ch{X} \(\in\) \ch{Cl,Br,I}) \\
	Conditions: Anhydrous, \ch{AlX3} OR \ch{FeX3} or \ch{Fe} \\

	Generation of Electrophile: \ch{Cl-Cl + AlCl3 -> Cl+ + AlCl4-} \\

	\noindent \textbf{Friedel-Craft Alkylation}

	Reagents: \ch{R-Cl} \\
	Conditions: \ch{AlCl3} \\

	Generation of Electrophile: \ch{R-Cl + AlCl3 -> R+ + AlCl4-} \\

	\noindent \textbf{Friedel-Craft Acylation}

	Reagents: \ch{R-COCl} \\
	Conditions: \ch{AlCl3} \\

	Generation of Electrophile: \ch{R-COCl + AlCl3 -> RCO+ + AlCl4-} \\

	\subsubsection{Oxidation}

	Though the benzene ring itself is unreactive, any alkyl chains bonded to a benzene ring can undergo oxidation, replacing the alkyl group with a \ch{COOH} group. However, the reaction requires that the benzyllic carbon has at least 1 \ch{H} or 1 \ch{O} atom bonded to it, otherwise a case such as a tert-butyl benzene will not be oxidised due to steric effect of \ch{CH3} groups attached to the benzylic \ch{C}.

	Reagents: \ch{KMnO4} \\
	Conditions: Warm \\

	\subsection{Activating and Deactivating Groups}

	Functional groups bonded to a benzene ring will distort the electron cloud of the benzene ring through two mechanisms. 
	Groups bonded to benzene rings can donate or withdraw electrons by `Resonance Effect' which arises when there is lone pairs (increasing reactivity) or \(\pi\) bonds (decreasing reactivity) at functional groups adjacent to the benzene. Groups can also affect reactivity through `Inductive Effect' which arises due to electronegativity of attached groups and may supply or withdraw \ch{e-} to the aromatic \(\pi\) \ch{e-} cloud. \\

	The functional groups presently on a benzene ring can determine the locations of where further electrophilic substitution reactions attach their functional groups onto. `Meta'-directing groups prefer addition to the third carbon in the benzene ring whereas `Ortho/Para'-directing groups prefer addition to the second or fourth carbon in the benzene ring. \\

	When there is more than one group on a benzene ring, follow the locations as determined by the strongest activating group on the benzene.
	
\end{document}