\documentclass[../main]{subfiles}

\begin{document}

\section{Stoichiometry}

	\subsection{Particles and Relative Mass}

	\scidef{Proton Number / Atomic Number}{The Proton Number is the number of protons in an atom of that element. This determines the identity of the atom.}
	\scidef{Nucleon Number / Mass Number}{The Nucleon Number is the total number of protons and neutrons in the nucleus of an atom.}
	\scidef{Nuclide}{A Nuclide is a species of atom with a specific proton number and nucleon number, written \(^\text{Atomic Number}_\text{Nucleon Number}\text{X}\)}
	\scidef{Isotope}{Isotopes of an element are atoms with the same proton number but different nucleon numbers}
	\scidef{Relative Isotopic Mass}{Relative Isotopic Mass is the mass of an atom of a specific isotope divided by \(\frac{1}{12}\) the mass of a carbon-12 atom, and is unitless}
	\scidef{Relative Atomic Mass}{Relative Atomic Mass is the weighted average of the masses of naturally occurring species of a specific element, and is unitless. The value is calculated as
	\[
		\text{A}_\text{r} = \frac{\sum \text{Nucleon number} \times \text{Fractional abundance}}{\frac{1}{12}\text{the mass of a carbon-12 atom}}
	\]}
	\scidef{Relative Molecular / Formula Mass}{Relative Molecular Mass is the relative mass of one covalent molecule of a certain substance, obtained as the sum of the \usub{A}{r}s of its constituent atoms. Relative Formula Mass is similar but used for ionic compounds and is calculated using the smallest collection of atoms from which a formula can be made (AKA a formula unit).}

	\subsection{The Mole}

	\scidef{Mole}{A mole of substance is the amount of a substance which contains as many elementary elements (molecules, ions, electrons, atoms, particles etc) as there are  atoms in 12 grams of carbon-12. Alternatively, it is the amount of substance which contains \(6.0 \times 10^{23}\) elementary elements, also known as the Avogadro constant and written as L.}

	\scidef{Molar Mass}{Molar mass is the mass of a mole of substance with units grams per mole.}

	\subsection{Chemical Formulae}

	\scidef{Empirical Formula}{The Empirical Formula of a compound is the simplest ratio of number of atoms of different elements in one molecule.}
	\scidef{Molecular Formula}{The Molecular Formula of a compound is the actual number of atoms of each element in one molecule of the compound.}

	A molecular formula of a substance is always a multiple of its empirical formula. Since Ionic compounds do not exist in single molecules, they do not have a molecular formula.

	\subsection{Stoichiometry}

	\scidef{Stoichiometry}{Stoichiometry is defined as the study of the proportions of which molecules react with each other.}
	\scidef{Stoichiometric Amounts}{Stoichiometric Amounts of a substance are the amounts which undergo reaction.}

	From a balanced equation, one can obtain ratios of moles of reactants and products, masses of reactants and products and volumes of gases evolved.

	\subsubsection{Limiting Reagent}

	When reacting substances, reactants may exceed stoichiometric amounts and not be reacted.

	\scidef{Limiting Reagent}{The Limiting Reagent in a reaction is the reactant which is deficient and consumed completely in a reaction.}

	\subsubsection{Yield}

	\scidef{Theoretical Yield}{The Theoretical Yield of a reaction is the mass of product formed calculated using the chemical equation and the amount of limiting reagent used.}
	\scidef{Actual Yield}{The Actual Yield of a reaction is the mass of product that is actually obtained after reaction.}
	\scidef{Percentage Yield}{The Percentage Yield is the ratio of actual yield to theoretical yield presented in percent.}

	\subsubsection{Volume of Gases}

	Avogadro's hypothesis states that at constant temperature and pressure, any volume of gas will have the same number of molecules.

	\scidef{Molar Volume}{The Molar Volume \usub{V}{m} is the volume taken up by 1 mole of gas at a certain temperature and pressure. Common temperatures and pressures include: \\
		Standard Temperature and Pressure (s.t.p.) at 273\si{K} and 100000\si{Pa} or 1\si{bar} gives \usub{V}{m} = 22.7 \si{dm^3.mol^{-1}}. \\
		Room Temperature and Pressure (r.t.p.) at 293\si{K} and 101325\si{Pa} or 1\si{bar} gives \usub{V}{m} = 24 \si{dm^3.mol^{-1}}.
	}

	\subsection{Concentration}

	\scidef{Solution}{A Solution is a homogeneous mixture of two or more substances, with the more abundant substance being the solvent and the less abundant substance the solute.}
	\scidef{Concentration}{The Concentration of a substance is the amount or mass of substance dissolved per unit of solvent or solution. The molar concentration is written by enclosing the name of substance in square brackets and has units \si{mol.dm^{-3}}. The mass concentration has units \si{g.dm^{-3}}}
	\scidef{Standard Solution}{A Standard Solution is a solution of known constitution and concentration.}

	\subsection{Acid-Base Titration}

	\scidef{Volumetric Analysis}{Volumetric Analysis, otherwise called Titrimetric Analysis, is a category of experiments which involve the precise measurement of volumes of solutions which react, typically involving the reaction of a standard solution (titrant) with a solution of unknown concentration (titre) to obtain the concentration of the unknown solution by adding one solution to another solution until stoichiometric amounts of reactants have reacted.}

	\subsubsection{Acids and Bases}

	\scidef{Arrhenius Acids and Bases}{Arrhenius Acids are substances which increase the concentration of \ch{H+} in a solution while an Arrhenius Base increases the concentration of \ch{OH-} in a solution. Both react to form \ch{H2O}. }

	\scidef{Br\o nsted Acids and Bases}{Br\o nsted-Lowry Acids are substances which donate protons while Br\o nsted-Lowry Bases receive protons.}

	\scidef{Strength of Acid / Base}{The Strength of an Acid or a Base is the extent of which it dissociates in an aqueous solution. Strong acids and bases exist as completely disassociated solutions while weak acids and bases are observed to exist in their complete molecules. The acidity constant, \usub{K}{a} or \usub{pK}{a} of an acid HA is defined by \(\frac{\ch{[H^+][A^-]}}{\ch{[HA]}}\) when the dissociation is in equilibrium.}

	\scidef{Basicity}{The Basicity of an acid is how many \ch{H+} ions it ionizes per molecule.}

	\subsubsection{Titration Curves and Indicators}

	\scidef{Equivalence Point}{The Equivalence Point is said to be reached when an acid-base mixture has undergone complete neutralization and is signified by a region of rapid pH change in the pH-Volume curve.}

	The equivalence point of a acid-base titration depends on whether its acid and base used is strong or weak. Strong acid - weak base reactions have rapid pH change from 3.5 to 6.5 and call for indicators like methyl orange (red-orange-yellow) and screened methyl orange (violet-grey-green) while weak acid - strong base reactions call for thymol blue (yellow-green-blue), phenolphthalein (colorless-pink) and thymolphthalein (colorless-blue). For phenolphthalein and thymolphthalein, the endpoint colors depend on the titrant. Strong-acid strong-base reactions can use all of the above indicators while weak-acid weak-base reactions have no suitable indicator.

	\subsubsection{Back Titration}

	Back titrations are used when the qualities of a substance need to be assessed when they cannot be easily dissolved into a solution, such as solid carbonates. Samples are reacted fully with a standard solution and the standard solution is then titrated against to investigate the change in its concentration to determine the properties of the sample.

	\subsubsection{Doule Indicator Method}

	The double indicator method is used when assessing titrations which have more than one region of rapid pH change. Multiple equivalence points suggest that there are multiple stages to a reaction and hence the amount of titrant used to reach a certain stage can be examined at multiple points and used to infer more data.	\\

	For a titration of a mixture of \ch{Na2CO3} and \ch{NaHCO3}, phenophthalein can be added as an indicator to find the first equivalence point. After reaching that point, methyl orange is then added to find the second equivalence point.

	\subsection{Redox}

	\scidef{Redox}{Redox reactions occur when reduction and oxidation occurs simultaneously.}

	\begin{center} \begin{tabular}{|c|c|c|} \hline
	&	Oxidation 	& 	Reduction 	\\ \hline
	\ch{O2} & 	+ & 	- \\ \hline
	\ch{H} 	&	- & + \\ \hline
	\ch{e-} &	- & + \\ \hline
	Oxidation Number &	+ & - \\ \hline
	\end{tabular} \end{center}

	

\end{document}