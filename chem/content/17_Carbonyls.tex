\documentclass[../main]{subfiles}

\begin{document}

\section{Carbonyl Compounds}

	\subsection{Structure of Carbonyl Compounds}

	Carbonyl compounds are organic molecules with a \ch{C=O} group. \\

	Aldehydes are compounds with a \ch{R-CO-H} (more commonly written \ch{R-CHO}) structure while ketones are compounds with a \ch{R-CO-R} structure. \\

	\subsection{Physical Properties of Carbonyl Compounds}

	Aldehydes and ketones have higher boiling points than similar alkene chains, but have lower boiling points than similar alcohols, typically due to the presence of the \ch{C=O} bond with its permanent dipole but the lack of corresponding \ch{H} bonding between carbonyl molecules. \\

	Lone pairs of \ch{e-} on \ch{O} allow carbonyl compounds to form \ch{H} bonds with water, giving them good solubility in water so long as the carbon chain is small in number (less than 5\ch{C}). Carbonyls such as propanone (common name `acetone') are used as industrial solvents.

	\subsection{Formation of Carbonyl Compounds}

	\noindent \textbf{Oxidation of Alcohols to Aldehydes and Ketones}

	Refer to `Oxidation' in Hydroxy Compounds. \\ 

	\subsection{Reaction of Carbonyl Compounds}

	\subsubsection{Reduction}

	Aldehydes and ketones can be reduced to primary alcohols and secondary alcohols respectively.

	Reagents: \ch{H2}, Ni \\
	Conditions: Heat \\

	Reagents: \ch{LiAlH4} or \ch{NaBH4}\\
	Conditions: Room temperature \\

	\ch{NaBH4} is a source of \ch{H-} and reduces aldehydes and ketones. \\

	Use \ch{LiAlH4} when avoiding reduction of alkene groups and \ch{NaBH4} when specifically reducing carbonyl compounds. \\

	\subsubsection{Oxidation}

	Aldehydes can be completely oxidized to carboxylic acids. \\

	Reagents: Acidic \ch{KMnO4} or Acidic \ch{K2Cr2O7} \\
	Conditions: Heat \\

	\subsubsection{Nucleophilic Addition}

	Carbonyl compounds contain a \ch{C=O} structure. Due to the electronegativity of \ch{O} as compared to \ch{C}, \ch{C} holds a partial positive charge which is then able to attract electron-rich nucleophiles. Compounds are added across the unsaturated \ch{C=O} bond, hence leaving carbonyl compounds susceptible to Nucleophilic Addition. \\

	Carbonyl compounds with alkyl groups or aromatic groups bonded to the \ch{C=O} structure decrease susceptibility to nucleophilic addition due to electronic effect reducing the intensity of the partial positive charge in the former and due to resonance effect in the latter. \\

	\noindent \textbf{Addition of Nitriles / Formation of Cyanohydrins}

	Reagents: \ch{HCN}, trace \ch{KCN} OR \ch{HCN}, trace \ch{KOH} OR \ch{KCN}, \ch{H2SO4} \\
	Conditions: \textbf{Room temperature} \\

	Nucleophilic addition of \ch{C+N} is a endothermic process. Heating the reaction will in fact slow the rate of reaction. \\

	Generation of Electrophile: \ch{KCN <=>  K+ + CN-}, OR \\
	Generation of Electrophile: \ch{HCN + KOH <=>  H2O + K+ + CN-} \\

	\chfg{
		\subscheme[-90]{
			\subscheme{
				\chemfig{-[:30]@{a1}\chembelow{C}{\delta+}(=[:90]\chemabove{O}{\delta-})-[:-30]H} \arrow{0}[,0] \+ \chemfig{@{a2}\lewis{4:,\ch{CN-}}} \arrow{->[slow]} \chemfig{-(-[:-90]H)(-[:90]\lewis{0:,\ch{O-}})-CN}
				\chemmove[shorten <=2pt,shorten >=2pt]{\draw (a2)..controls +(120:6mm) and +(45:6mm) ..(a1); }
			}
			\arrow{0}[,0.2]
			\subscheme{
				\chemfig{-(-[:-90]H)(-[:90]@{a3}\lewis{0:,\ch{O-}})-CN} \arrow{0}[,0] \+ \chemfig{@{a4}H-[@{a5}]@{a6}CN}
				\chemmove[shorten <=6pt,shorten >=2pt]{\draw (a3)..controls +(0:5mm) and +(90:10mm) ..(a4); }
				\chemmove[shorten <=2pt,shorten >=2pt]{\draw (a5)..controls +(90:5mm) and +(90:5mm) ..(a6); }
			}
			\arrow{->[fast]}[,0.85]
			\subscheme{
				\chemfig{-(-[:-90]H)(-[:90]OH)-CN} \arrow{0}[,0] \+ \chemfig{\lewis{4:,\ch{CN-}}}
			}
		}
	}

	Study of this reaction's kinetics determines that the rate of reaction is determined by the concentration of carbonyl and \ch{CN-}. As a result, for higher rates of reaction there needs to be trace amounts of \ch{KCN-} or \ch{KOH} to catalyze reactions. \\

	Note that in the case of \ch{KCN}, \ch{H2SO4} the reaction consumes \ch{H2SO4} rather than using it as a catalyst. \ch{H2SO4} needs to be added in comparable amounts rather than in trace amounts. \\

	Nucleophilic addition of \ch{CN} produces a structure known as a cyanohydrin,  a \ch{CN-} adjacent to \ch{C-OH}. Cyanohydrins, similar to nitrile compounds in halogenoalkanes, can be further hydrolyzed in acidic or basic conditions and can also be reduced to form primary amines. Carboxylic acid, carboxylate and primary amines adjacent to a \ch{C-OH} suggest that the product may have been obtained through a cyanohydrin. \\

	Carbonyl compounds are typically \usup{sp}{3} hybridized with respect to \ch{C} in \ch{C=O}. The trigonal planar geometry of \ch{C=O} containing molecules mean that nucleophiles have equal chance of attacking from either plane, eventually forming a racemic mixture which is optically inactive.

	\subsubsection{Characteristic Reactions}

	Carbonyl compounds can be better differentiated through the use of several chemical tests, typically involving either condensation or oxidation reactions. \\

	\noindent \textbf{Condensation and 2,4-DNPH} \\

	Aldehydes and Ketones both react with 2,4-dinitro phenyl hydrazine (2,4-DNPH) to form 2,4-dinitro phenyl hydrazones which are orange precipitates.

	\chfg{
		\subscheme[-90]{
			\subscheme{
				\chemfig{[:30]**6(--(-NO2)--(-[:120,,,1]NO2)-(-[:180]N(-[:90]H)-[:180]H2N)-)} \arrow{0}[,0] \+ \chemfig{R=O}
			}
			\arrow{->}[,0.85]
			\subscheme{
				\chembelow{\chemfig{[:30]**6(--(-NO2)--(-[:120,,,1]NO2)-(-[:180]N(-[:90]H)=[:180]R)-)}}{orange~ppt} \arrow{0}[,0] \+ \chemfig{\ch{H2O}}
			}
		}
	}

	\vspace{12pt}

	2,4-DNPH does not react with carboxylic acids as the interference of the \ch{C-OH} renders lone pairs on \ch{C=O} unable to undergo condensation. \\

	\noindent \textbf{Oxidation using Tollen's Reagent} \\

	Aldehydes are oxidized by Tollen's reagent to form a silver mirror. \\

	\ch{R=O + 3 OH- + 2 [Ag(NH3)2]+ -> ROO- + 2 Ag + 4 NH3 + 2 H2O} \\

	Reagents: Tollens' Reagent \ch{[Ag(NH3)2]+}, \ch{NaOH} \\
	Conditions: Heat \\
	Observation: Silver mirror formed \\

	\noindent \textbf{Oxidation using Fehling's Solution} \\

	Aliphatic aldehydes are oxidized by Fehling's solution to form a silver mirror. \\

	\ch{R=O + 5 OH- + 2 Cu2+ -> ROO- + Cu2O + 3 H2O} \\

	Reagents: Fehling's solution \ch{Cu2+ (aq)}, \ch{NaOH} \\
	Conditions: Heat \\
	Observation: Reddish-brown ppt formed \\

	\noindent \textbf{2,4-DNPH / Tollens / Fehlings} \\

	Using these three tests sequentially allows for identification of carbonyl compounds and further differentiation between ketones, aldehydes and aromatic aldehydes. \\

	\subsubsection{Iodoform Formation}

	Carbonyl compounds with a \ch{R-CO-CH3} structure form yellow ppt of \ch{CHI3} when heated with \ch{I2} and \ch{NaOH}. As a result, 2-carbonyls and ethaldehyde can be identified through this test.

	Reagents: \ch{Na}, \ch{I2} \\
	Observation: Yellow ppt \ch{(CHI3)} \\

	\chfg{
		\subscheme[-90]{
			\subscheme{\chemfig{R-C(-[:90]CH3)=O} \+ \ch{3 I2} \arrow \chemfig{R-C(-[:90]CI3)=O} \+ \ch{3 HI}}
			\arrow{0}[,0.2]
			\subscheme{\chemfig{R-C(-[:90]CI3)=O} \+ \ch{OH-} \arrow \chemfig{R-C(-[:90]\ch{O-})=O} \+ \ch{CHI3}}
		}
	} \\

	\subsection{Aromatic Carbonyls}

	Carbonyl groups on a benzene ring are deactivating due to the presence of the electron deficient \ch{C=O} structure, in addition to the fact that these compounds are typically strengthened due to resonance structures. \\

	
\end{document}