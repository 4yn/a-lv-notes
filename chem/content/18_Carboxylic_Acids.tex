\documentclass[../main]{subfiles}

\begin{document}

\section{Carboxylic Acids}

	\subsection{Structure of Carboxylic Acids}

	Carboxylic Acids are organic molecules with the \ch{COOH} group.

	\subsection{Physical Properties of Carboxylic Acids}

	

	\subsection{Formation of Hydroxy Compounds}

	\noindent \textbf{Alcohols / Aldehydes to Carboxylic Acids}

	Refer to `Oxidation' in Alcohols and Carbonyls. \\

	\noindent \textbf{Nitriles to Carboxylic Acids}

	Refer to `Nitriles' in Halogenoalkanes. \\ 

	\subsection{Reaction of Carboxylic Acids}

	\subsubsection{Acid-Base Reactions}

	Carboxylic Acids are the most acidic organic functional group in the syllabus. The \ch{C-OH} group contributes acidic character where its \ch{H} can be easily lost, which is then intensified by the \ch{C=O} group due to it withdrawing \ch{e-} from \ch{C}. Additionally, the conjugate base of a carboxylic acid, also known as the carboxylate group \ch{COO-} is highly stable due to it having two equivalent resonance structures. \\

	As a result, carboxylic acids are able to react with not only \ch{Na (s)} and \ch{NaOH}, but can also react with carbonates to produce salt and \ch{CO2 (g)}, and hence are more acidic than \ch{C-OH} and \ch{Ph-OH}. \\

	\subsubsection{Reduction}

	Carboxylic Acids are able to be reduced back to primary alcohols.

	Reagents: \ch{H2}, Ni \\
	Conditions: Heat \\

	Reagents: \ch{LiAlH4} or \ch{NaBH4}\\
	Conditions: Room temperature \\

	\subsubsection{Acyl Chloride Formation}

	Carboxylic Acids can be reacted to form another acidic species known as the Acyl Chloride \ch{COCl}. Acyl Chlorides are more reactive than carboxylic acids and are hence more often used for ester formation and amide formation.

	Reagents: \ch{PCl3} (1:3 Ratio) \\
	Conditions:  Room temperature \\

	Reagents: \ch{PCl5} (1:1 Ratio) \\
	Conditions:  Room temperature \\
	Observation: White fumes formed \ch{(HCl)} \\

	Reagents: \ch{SOCl2} (1:1 Ratio) \\
	Conditions:  Room temperature \\
	Observation: White fumes formed \ch{(HCl)} \\

	\subsubsection{Condensation}

	In the presence of other nucleophiles, an \ch{R-OH} is a suitable leaving group and can make way for nucleophiles such as \ch{X}. \\

	After being treated with \ch{Na} in anhydrous conditions, an alkoxide \ch{R-O-} is obtained which can then act as a nucleophile, allowing it to take place in reactions such as ether synthesis and acylation. \\

	Reagents: \ch{ROH} \\
	Conditions:  Concentrated \ch{H2SO4}, \SI{60}{\celsius} \\

	Reagents: Acyl Chloride and \ch{ROH} \\
	Conditions:  Room temperature \\
	Observation: White fumes formed \ch{(HCl)} \\
	
\end{document}