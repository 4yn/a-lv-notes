\documentclass[../main]{subfiles}

\begin{document}

\section{Carboxylic Acids}

	\subsection{Structure of Carboxylic Acids}

	Carboxylic Acids (combination between carbonyl and hydroxyl groups) are organic molecules with the \ch{COOH} group.

	\subsection{Physical Properties of Carboxylic Acids}

	Carboxylic acids have higher boiling points than similar alkanes as they are able to form intermolecular hydrogen bonds. Carboxylic acids also have higher boilings points than similar alcohols as the \ch{O-H} bond is more polar due to the presence of electron withdrawing \ch{C=O}.\\

	Due to the presence of intermolecular hydrogen bonds, carboxylic acids tend to dimerize when in a gaseous state. \\

	Carboxylic acids with four or less carbon atoms are soluble in water due to the formation of hydrogen bonds with \ch{COOH} as well as the formation of ion-dipole interactions with \ch{COO-}. Longer chains have hydrophobic carbon chains which reduce their solubility.

	\subsection{Formation of Hydroxy Compounds}

	\noindent \textbf{Alcohols / Aldehydes to Carboxylic Acids}

	Refer to `Oxidation' in Alcohols and Carbonyls. \\

	\noindent \textbf{Nitriles to Carboxylic Acids}

	Refer to `Nucleophilic Substitution' in Halogenoalkanes and `Nucleophilic Addition' in Carbonyl Compounds. \\ 

	\noindent \textbf{Alkenes to Carboxylic Acids}

	Refer to `Oxidation' in Alkenes. \\ 

	\noindent \textbf{Phenyls to Carboxylic Acids}

	Refer to `Oxidation' in Arenes. \\ 

	\subsection{Reaction of Carboxylic Acids}

	\subsubsection{Acid-Base Reactions}

	Carboxylic Acids are the most acidic organic functional group in the syllabus. The \ch{C-OH} group contributes acidic character where its \ch{H} can be easily lost, which is then intensified by the \ch{C=O} group due to it withdrawing \ch{e-} from \ch{C}. Additionally, the conjugate base of a carboxylic acid, also known as the carboxylate group \ch{COO-} is highly stable due to it having two equivalent resonance structures. \\

	As a result, carboxylic acids are able to react with not only \ch{Na (s)} and \ch{NaOH}, but can also react with carbonates to produce salt and \ch{CO2 (g)}, and hence are more acidic than \ch{C-OH} and \ch{Ph-OH}. \\

	\subsubsection{Reduction}

	Carboxylic Acids are able to be reduced back to primary alcohols.

	Reagents: \ch{H2}, Ni \\
	Conditions: Heat \\

	Reagents: \ch{LiAlH4} in dry ether \\
	Conditions: Room temperature \\

	\subsubsection{Acyl Chloride Formation}

	Carboxylic Acids can be reacted to form another acidic species known as the Acyl Chloride \ch{COCl}. Acyl Chlorides are more reactive than carboxylic acids and are hence more often used for ester formation and amide formation.

	Reagents: \ch{PCl3} (1:3 Ratio) \\
	Conditions:  Room temperature \\

	Reagents: \ch{PCl5} (1:1 Ratio) \\
	Conditions:  Room temperature \\
	Observation: White fumes formed \ch{(HCl)} \\

	Reagents: \ch{SOCl2} (1:1 Ratio) \\
	Conditions:  Room temperature \\
	Observation: White fumes formed \ch{(HCl)} \\

	\subsubsection{Condensation}

	In the presence of other nucleophiles, an \ch{R-OH} is a suitable leaving group and can make way for nucleophiles such as \ch{X}. \\

	After being treated with \ch{Na} in anhydrous conditions, an alkoxide \ch{R-O-} is obtained which can then act as a nucleophile, allowing it to take place in reactions such as ether synthesis and acylation. \\

	Reagents: \ch{ROH}, NO Phenol \\
	Conditions:  Concentrated \ch{H2SO4}, \SI{60}{\celsius} \\

	Reagents: Acyl Chloride and \ch{ROH} \\
	Conditions:  Room temperature \\
	Observation: White fumes formed \ch{(HCl)} \\

	\subsection{Acyl Chlorides and Esters}

	Acyl chlorides are organic compounds with a \ch{R-COCl} structure. Esters are compounds which have \ch{ROO-R'} strucures. \\

	Carboxylic acid derivatives such as these typically have lower boiling points than the original carboxylic acids due to their inability to form \ch{H} bonds with other molecules. \\

	Easters are generally insoluble in polar solvents due to long hydrophobic chains while acyl chlorides are soluble in polar solvents as it readily hydrolyzes to its carboxylic acid form. Both compounds are readily soluble in organic compounds.

	\subsection{Reactions of Acyl Chlorides and Esters}

	\noindent \textbf{Hydrolysis}

	Reactants: \ch{H2O} \\
	Observation: White fumes formed \ch{(HCl)} \\

	On solution with water, acyl chlorides will readily react with \ch{H2O} to produce \ch{ROOH} and white fumes of \ch{HCl}. The end solution is usually very acidic due to the presence of a large amount of dissolved \ch{HCl}. \\

	Reactants: \ch{HCl} OR \ch{NaOH} \\
	Conditions: Heat under reflux \\

	Ester bonds can be broken in acidic and basic medium, forming carboxylic acids and alcohols in their respective protonated or deprotonated states. Note that if ester bonds are formed between a carboxylic acid and phenol, additional reactant is required to protonate the phenoxide after initial reaction. \\

	\noindent \textbf{Condensation}

	The presence of the \ch{C=O} structure in acyl chlorides renders the \ch{C} positive and hence nucleophiles tend to be attracted towards it to undergo condensation reactions. \\

	Reactive acyl chlorides can form \ch{RC-R'} bonds very rapidly due to the \ch{COCl} group. Acyl chlorides react with alcohols, phenols (but preferably the more nucleophilic phenoxide), primary and secondary amides to form esters. \\

	Reactants: \ch{ROH} OR \ch{Ph-O-} OR \ch{NH3} OR \ch{RNH2} OR \ch{RR'NH2} \\
	Conditions: Room temperature \\

	This reaction proceeds much more rapidly as compared to ester formation with \ch{RCOOH} due to the higher reactivity of \ch{RCOCl}. Phenolic esters can only be formed through the use of acyl chlorides.
	
\end{document}