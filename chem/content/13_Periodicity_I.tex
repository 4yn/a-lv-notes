\documentclass[../main]{subfiles}

\begin{document}

\section{Periodicity I}

	\subsection{Period 3}

	\begin{center} \begin{tabular}{l|l|l|l|l|l|l|l}
		& & & & & & & \\
		\ch{11Na} & \ch{12Mg} & \ch{13Al} & \ch{14Si} & \ch{15P} & \ch{16O} & \ch{17Cl} & \ch{18Ar} \\
		& & & & & & & \\
	\end{tabular} \end{center}

	Period 3 comprises the elements of \ch{Na} to \ch{Ar}.

	\subsection{Atomic and Physical Properties}

	Please refer to Atomic Structure.

	\subsection{Period 3 and Oxygen}

	\subsubsection{Oxides}

	\begin{center} \begin{tabular}{l|l|l|l|l|l|l}
		& & & & & &  \\
		\ch{Na2O} & \ch{MgO} & \ch{Al2O3} & \ch{SiO2} & \underline{\ch{P2O6}} & \underline{\ch{SO2}} & \underline{\ch{Cl2O}} \\
		 &  &  &  & \ch{P4O10} & \ch{SO3} & \underline{\ch{Cl2O7}} \\
		& & & & & & \\
	\end{tabular} \end{center}

	\emph{Note: Properties of underlined compounds are not in the A level syllabus.} \\

	Oxides of \ch{Na}, \ch{Mg} and \ch{Al} are ionic compounds, \ch{Si} is macromolecular and \ch{P}, \ch{S} and \ch{Cl} are simple covalent molecules. All are solid at room temperature save for \ch{SO3} which is an orange liquid. \\

	\subsubsection{Period 3 and Oxygen}

	\ch{Na} and \ch{Mg} react vigorously and burns with a orange and white flame respectively to form their oxides. \\
	\ch{Al} initially reacts vigorously but slows down due to the formation of an impervious layer of \ch{Al2O3}. \\
	\ch{Si} reacts slowly with oxygen to form \ch{SiO2}. \\
	\ch{P} reacts vigorously and burns  with a yellow flame to form its oxide. \\
	\ch{S} reacts slowly and burns with a blue flame to form its oxide. \\

	\subsubsection{Oxides and Water}

	Most oxides except for \ch{Al2O3} and \ch{SiO2} which have very strong lattice energies and covalent bonds are able to dissolve in water. \ch{MgO} is sparingly soluble. \\

	\ch{NaO + H2O -> 2 NaOH (aq)} \\
	\ch{MgO + H2O <-> Mg(OH)2 (s) <-> Mg(OH)2 (aq)} \\
	Oxides of \ch{Na} and \ch{Mg} are basic as they readily accept \ch{H+} in water and hence release \ch{OH-} in aqueous solution. \ch{MgO} is significantly less basic because it does not dissolve well. \\

	\ch{P4O10 + 6 H2O -> 4 H3PO4 (aq)} \\
	\ch{SO3 + H2O -> H2SO4 (aq)} \\
	Oxides of \ch{P} and \ch{S} are highly acidic because their bonding with many \ch{O} render them electron deficient and prone to nucleophilic attack by \ch{OH-}, causing the release of \ch{H+} into aqueous solution.

	\subsubsection{Acid-Base Properties}

	Across Period 3, oxides of elements progress from basic to amphoteric to acidic. \\

	\ch{Na2O} and \ch{MgO} act as bases whereas \ch{P4O10} and \ch{SO3} act as acids. \\

	\ch{Al2O3 + 6 HCl -> 2 AlCl3 + 3 H2O} \\
	\ch{Al2O3 + 2 NaOH + 3 H2O -> 2 Na+ [Al(OH)4]-} \\
	\ch{Al(OH)3 + 3 HCl -> AlCl3 + 3 H2O} \\
	\ch{Al(OH)3 + NaOH -> Na+ [Al(OH)4]-} \\
	Oxides and hydroxides of \ch{Al} are amphoteric and react with both acids and bases. \\

	\ch{SiO2 + 2 NaOH (conc) -> Na2SiO3 + H2O} \\
	\ch{SiO2} is classified as an acidic oxide because of its ability to react with concentrated bases. \\

	\subsection{Period 3 and Chlorine}

	\subsubsection{Chlorides}

	\begin{center} \begin{tabular}{l|l|l|l|l|l}
		& & & & & \\
		\ch{NaCl} & \ch{MgCl2} & \ch{AlCl3} & \ch{SiCl4} & \ch{PCl5} & \underline{\ch{S2Cl2}} \\
		 &  &  &  & \underline{\ch{PCl3}} & \\
		& & & & &  \\
	\end{tabular} \end{center}

	\emph{Note: Properties of underlined compounds are not in the A level syllabus.} \\

	Chlorides of Period 3 elements demonstrate the property of maximum oxidation state, where the theoretical limit on the number of covalent bonds an atom can form is equal to the number of valence electrons in its unbonded atom because electrons can be promoted to low-lying 3d orbitals and form as many bonds as it can supply electrons. \\

	Chlorides of \ch{Na} and \ch{Mg} are ionic compounds while the rest are simple covalent molecules. \ch{NaCl} and \ch{MgCl} are solids at room temperature, \ch{AlCl3} is solid but can sublime and dimerize into \ch{Al2Cl6} at higher temperatures, \ch{SiCl4} is liquid and \ch{PCl5} is gaseous.

	\subsubsection{Period 3 and Chlorine}

	\ch{Na}, \ch{Mg}, \ch{Al} and \ch{P} react vigorously and burns with an orange, white, white and yellow flame respectively to form their chlorides.

	\subsubsection{Chlorides and Water}

	\ch{NaCl} is a strong electrolyte and disassociates readily in water to form a neutral solution. \\

	\ch{[Mg(H2O)6]^{2+} + H2O-> [Mg(OH)(H2O)5]+ + H3O+} \\
	\ch{MgCl2} disassociates readily in water and exhibits slight hydrolysis. \\

	\ch{[Al(H2O)6]^{3+} + H2O -> [Al(OH)(H2O)5]^{2+} + H3O+} \\
	\ch{AlCl3} disassociates readily in water and exhibits moderate hydrolysis. \\

	\ch{AlCl3 + 3 H2O -> Al(OH)3 + 3 HCl (g)} \\
	\ch{SiCl4 + 2 H2O -> SiO2 + 4 HCl (g)} \\
	\ch{PCl5 + H2O -> POCl3 + 2 HCl (g)} \\
	In conditions with limited amount of water, \ch{AlCl3}, \ch{SiCl4} and \ch{PCl5} or when cold water is added to \ch{PCl5}, fumes of \ch{HCl} are released. \\

	\ch{POCl3 + 3 H2O -> H3PO4 + 3 HCl} \\
	In excess water, \ch{POCl3} can complete its reaction. \\

	\ch{SiCl4 + 2 H2O -> SiO2 + 4 HCl (aq)} \\
	\ch{PCl5 + 4 H2O -> H3PO4 + 5 HCl} \\
	In excess water, \ch{SiCl4} and \ch{PCl5} react to form acidic solutions. \\

	\subsection{Period 3 and Water}

	\ch{Na} reacts vigorously with water to form \ch{NaOH}. \\
	\ch{Mg} reacts vigorously with steam to form \ch{MgO} and reacts slowly with water to form \ch{Mg(OH)2}. \\
	\ch{Al} reacts vigorously with steam to form \ch{Al2O3} but slows down due to its impervious oxide layer. \\
	\ch{Cl2} reacts with water to form \ch{HCl} solution. \\


	% old table
		% \begin{tabular}{llllllll}
		%                & Na                                                                                                  & Mg                                                                         & Al                                                                                            & Si                                                         & P                                                                                                                & S                                                                                                                     & Cl  \\
		% + O2           & \begin{tabular}[c]{@{}l@{}}\ch\{NaO2\}\\ Vigorous, orange flame\\ May form \ch\{N2O2\}\end{tabular} & \begin{tabular}[c]{@{}l@{}}\ch\{MgO\}\\ Vigorous, white flame\end{tabular} & \begin{tabular}[c]{@{}l@{}}\ch\{Al2O3\}\\ Initially vigorous\\ Oxide layer forms\end{tabular} & \begin{tabular}[c]{@{}l@{}}\ch\{SiO2\}\\ Slow\end{tabular} & \begin{tabular}[c]{@{}l@{}}\ch\{P4O6 -\textgreater P4O10\}\\ Vigorous, yellow flame\\ Mixture forms\end{tabular} & \begin{tabular}[c]{@{}l@{}}\ch\{SO2 -\textgreater SO3\}\\ Slow\\ \ch\{SO3\} forms in specific conditions\end{tabular} & nil \\
		% O-ide          & Alkaline                                                                                            & Alkaline                                                                   & Amphoteric                                                                                    & Acidic                                                     & Acidic                                                                                                           & Acidic                                                                                                                &     \\
		% O-ide + H2O    & \begin{tabular}[c]{@{}l@{}}\ch\{NaOH\}\\ Vigorous\end{tabular}                                      & Sparsely soluble                                                           & Insoluble                                                                                     & Insoluble                                                  & \ch\{H3PO4\}                                                                                                     & \ch\{H2SO4\}                                                                                                          &     \\
		% O-ide + H+/OH- &                                                                                                     &                                                                            &                                                                                               &                                                            &                                                                                                                  &                                                                                                                       &     \\
		% + Cl2          & \ch\{NaCl\}                                                                                         & \ch\{MgCl2\}                                                               & \ch\{AlCl3\}                                                                                  & \ch\{SiCl4\}                                               & \ch\{PCl5\}                                                                                                      & nil                                                                                                                   & nil \\
		% Cl-ide         &                                                                                                     &                                                                            &                                                                                               &                                                            &                                                                                                                  &                                                                                                                       &     \\
		% Cl-ide + H2O   & Neutral                                                                                             &                                                                            & Hydrolysis to acidic soln                                                                     &                                                            & Produces \ch\{HCl\}                                                                                              &                                                                                                                       &     \\
		% + H2O          &                                                                                                     &                                                                            &                                                                                               &                                                            &                                                                                                                  &                                                                                                                       &    
		% \end{tabular}


\end{document}