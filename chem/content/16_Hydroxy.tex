\documentclass[../main]{subfiles}

\begin{document}

\section{Hydroxy Compounds}

	\subsection{Structure of Hydroxy Compounds}

	Hydroxy compounds, also known as alcohols, are organic molecules with a \ch{-OH} group. \\

	\subsection{Physical Properties of Hydroxy Compounds}

	Due to the presence of \ch{-OH} groups and the preceding hydrocarbon chain, Hydroxy Compounds are able to form strong Hydrogen bonds with polar solvents like water as well as within other Hydroxy molecules. As a result, Hydroxy Compounds are soluble in both organic and polar solvents and can be used as an emulsifier and also have relatively high melting and boiling points, which also increases as the number of Hydroxy groups increases.

	\subsection{Formation of Hydroxy Compounds}

	\noindent \textbf{Alkenes to Alcohols}

	Refer to `Electrophilic Addition' in Alkenes. \\

	\noindent \textbf{Halogenoalkanes to Alcohols}

	Refer to `Nucleophilic Substitution' in Halogenoalkanes. \\

	\noindent \textbf{Reduction of Carboxylic Acid / Aldehyde / Ketone to Alcohols}

	Refer to `Reduction' in Carbonyl Compounds and Carboxylic Acids. \\

	\subsection{Reaction of Hydroxy Compounds}

	\subsubsection{Reduction}

	Alcohols are able to donate \ch{H+} in organic solutions and can hence act as weak acids, therefore reducing metals like \ch{Na}. However, in aqueous solution \ch{H2O} acts as a stronger acid and hence alcohols do not act as an acid. Across different alkyl chains, the more substituted chain is the weaker acid because the electron-donating \ch{C} will destabilize \ch{O-}. \\

	Reagents: \ch{Na} \\

	\ch{R-OH + Na -> R-O- Na+ + 1/2 H2}

	\subsubsection{Elimination}

	\noindent \textbf{Dehydrohalogenation}

	Reagents: Concentrated \ch{H2SO4} \\
	Conditions: Heat at \SI{170}{\celsius}\\

	As with other reactions involving elimination of \ch{H2O}, the presence of a \(\beta\)-\ch{H} is required for successful elimination. \\

	\subsubsection{Nucleophilic Substitution}

	In the presence of other nucleophiles, an \ch{R-OH} is a suitable leaving group and can make way for nucleophiles such as \ch{X}. \\

	After being treated with \ch{Na} in anhydrous conditions, an alkoxide \ch{R-O-} is obtained which can then act as a nucleophile, allowing it to take place in reactions such as ether synthesis and acylation. \\

	\noindent \textbf{Halogenation}

	Reagents: \ch{HCl (aq)} OR \ch{Br2}, \ch{H2SO4} OR \ch{I2}, {H3PO4} \\
	Conditions:  Heat \\

	Reagents: \ch{PCl3} OR \ch{PBr3} OR red \ch{P}, \ch{I2} (1:3 Ratio) \\
	Conditions:  Anhydrous, room temperature \\

	Reagents: \ch{PCl5} (1:1 Ratio) \\
	Conditions:  Room temperature \\
	Observation: White fumes formed \ch{(HCl)} \\

	Reagents: \ch{SOCl2} (1:1 Ratio) \\
	Conditions:  Room temperature \\
	Observation: White fumes formed \ch{(HCl)} \\

	\noindent \textbf{Ether Synthesis}

	Reagents: \ch{R-Cl} \\
	Conditions:  Heat \\

	\noindent \textbf{Ester Synthesis}

	Reagents: \ch{RCOOH} \\
	Conditions:  Concentrated \ch{H2SO4}, \SI{60}{\celsius} \\

	Reagents: \ch{RCOCl} \\
	Conditions:  Room temperature \\
	Observation: White fumes formed \ch{(HCl)} \\

	\subsubsection{Oxidation}

	\noindent \textbf{Oxidation of Primary Alcohol to Aldehyde}

	Reagents: \ch{K2Cr2O7} \\
	Conditions: With dilute \ch{H2SO4}, heat, immediate distillation \\

	\noindent \textbf{Oxidation of Primary Alcohol to Carboxylic Acid and Secondary Alcohol to Ketone}

	Reagents: \ch{K2Cr2O7} OR \ch{KMnO4} \\
	Conditions: Dilute \ch{H2SO4}, heat \\

	\subsubsection{Iodoform Formation}

	Reagents: \ch{Na}, \ch{I2} \\
	Observation: Yellow ppt \ch{(CHI3)} \\

	\chfg{
		\subscheme[-90]{
			\subscheme{\chemfig{R-C(-[:270]H)(-[:90]CH3)-OH} \+ \ch{Na} \arrow \chemfig{R-C(-[:90]CH3)=O} \+ \ch{HI}} 
			\arrow{0}[,0.2]
			\subscheme{\chemfig{R-C(-[:90]CH3)=O} \+ \ch{3 I2} \arrow \chemfig{R-C(-[:90]CI3)=O} \+ \ch{3 HI}}
			\arrow{0}[,0.2]
			\subscheme{\chemfig{R-C(-[:90]CI3)=O} \+ \ch{OH-} \arrow \chemfig{R-C(-[:90]\ch{O-})=O} \+ \ch{CHI3}}
		}
	} \\

	The Iodoform reaction requires the presence of a \ch{CH3} and \ch{H} group next to \ch{COH} in the alcohol reactant. \\

	The Iodoform reaction results in the formation of a \ch{COO-} compound and also the precipitation of \ch{CHI3}. \\

	\subsection{Structure of Phenols}

	A Phenol group is a benzene ring with an \ch{OH} group bonded to a carbon in the ring. \\

	The \ch{Ph-OH} group is electron donating and hence an activating group, making them very reactive. Phenols are 2,4-directing. \\

	Due to the presence of the \(\pi\) \ch{e-} cloud in the benzene ring, \ch{Ph-OH} can deprotonate to \ch{Ph-O-} and the negative charge on \ch{O-} can be dispersed across the ring, hence phenols can act as acids. The acidity of a phenol is strengthened by attaching electron-withdrawing groups on the benzene ring (such as \ch{NO2}) or can be weakened by attaching electron-donating groups on the benzene ring (such as \ch{CH3}).

	\subsection{Physical Properties of Phenols}

	Pure phenols are colorless crystalline solids at room temperature. Phenols melts at \SI{42}{\celsius} and boil at \SI{217}{\celsius} due to their ability to form hydrogen bonds between molecules. However, the extent of these hydrogen bonds can be diminished if the species of phenol is able to exhibit intramolecular \ch{H} bonds which then reduces the strength of intermolecular bonds.

	Phenol compounds are sparsely soluble in water due to the presence of a hydrophobic benzene ring, but are soluble in \ch{NaOH} solution. \\

	\subsection{Reactions of Phenols}

	\subsubsection{Reduction}

	Reagents: \ch{Na} \\

	\ch{Ph-OH + Na -> Ph-O- Na+ + 1/2 H2}

	\subsubsection{Acid-base}

	Reagents: \ch{NaOH} \\

	\ch{Ph-OH + Na -> R-O- Na+ + H2O}

	\subsubsection{Electrophilic Substitution}

	Due to the presence of the electron-donating \ch{OH} group, the electron density in a phenol ring is increased, leaving it more reactive to electrophilic substitution reactions. Compared to reactions of arenes, reactions of phenols typically do not require heating or the presence of a catalyst. \\

	\noindent \textbf{Halogenation}

	Reagents: \ch{X2} (\ch{X} \(\in\) \ch{Cl,Br}) \\
	Conditions:  Aqueous, room temperature\\

	Halogenation of phenols in a aqueous substrate forms a polysubstituted product. A tribromophenol is an insoluble white precipitate. \\

	Reagents: \ch{X2} (\ch{X} \(\in\) \ch{Cl,Br}) \\
	Conditions: \ch{CCl4}, room temperature\\

	Halogenation of phenols in a aqueous substrate forms a monosubstituted product. \\

	\noindent \textbf{Nitration}

	Reagents: Concentrated \ch{HNO3} \\
	Conditions: Room temperature\\

	Nitration of phenols in a concentrated solution forms a polysubstituted product. \\

	Reagents: Dilute \ch{HNO3} \\
	Conditions: Room temperature\\

	Nitration of phenols in a dilute solution forms a monosubstituted product. \\

	\subsubsection{Acylation}

	Reagents: \ch{R-OCl} \\
	Conditions: \ch{Na}, room temperature \\
	Observation: White fumes formed \ch{(HCl)} \\

	\ch{Na} is added to form a carboxylate salt which can then readily react with acyl halides to form an ester. \ch{Ph-OH} is not a strong enough nucleophile to undergo esterification with carboxylic acids. \\

	\subsubsection{Complexation}

	Reagents: Neutral \ch{FeCl3} \\
	Observation: Violet coloration \\

	\chfg{
		\ch{Fe^{3+}} \+ \chemfig{!\rightbenz([:0]-OH)} \arrow
	} \\
	\chfg{
		\chembelow{\chemleft[ \subscheme{ \ch{Fe} \arrow{0}[,0] \chemleft( \chemfig{!\rightbenz([:0]-O)} \chemright{)_6} } \chemright{]^{3-}}}{violet~coloration} \+ \ch{H+}
	} \\

	\vspace{12pt}

	Phenols form complexes with \ch{Fe3+} and obtain a violet color. Phenols with different groups attached to them may have varying colors.

	
\end{document}