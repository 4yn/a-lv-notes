\documentclass[../main]{subfiles}

\begin{document}

\section{Hydroxy Compounds}

	\subsection{Structure of Hydroxy Compounds}

	Hydroxy compounds, also known as alcohols, are organic molecules with a \ch{-OH} group.

	\subsection{Physical Properties of Hydroxy Compounds}

	Due to the presence of \ch{-OH} groups and the preceding hydrocarbon chain, Hydroxy Compounds are able to form strong Hydrogen bonds with polar solvents like water as well as within other Hydroxy molecules. As a result, Hydroxy Compounds are soluble in both organic and polar solvents and can be used as an emulsifier and also have relatively high melting and boiling points, which also increases as the number of Hydroxy groups increases.

	\subsection{Formation of Hydroxy Compounds}

	\subsection{Reaction of Arenes}

	Arenes are typically reactive due to the presence of its electron-dense \ch{\(\pi\) e-} cloud. However, benzene rings tend to preserve their aromaticity even after reactions due to its unique stability. Arenes will typically undergo electrophillic substitution reactions. \\

	Side chains of benzene rings are typically more reactive than their counterparts without benzene rings as the \ch{\(\pi\) e-} cloud of the benzene ring can disassociate across \ch{C} atoms in the side chain, making them more reactive. Otherwise, side chains can undergo any reactions as if they lacked its benzene ring substituent.

	\subsubsection{Reduction}

	Reagents: \ch{H2}\\
	Conditions: \ch{Ni} + High T + High P 

	\subsubsection{Electrophillic Substitution}

	\noindent \textbf{Nitration}

	Reagents: Concentrated \ch{HNO3} \\
	Conditions: Concentrated \ch{H2SO4}, \SI{55}{\celsius} in benzene, \SI{30}{\celsius} in methylbenzene \\

	Generation of Electrophile: \ch{HNO3 + H2SO4 <=> H2NO3+ + HSO4-} 

	\chfg{
		\chemfig{!\upbenz(-[:90]H)(-[@{a1}:270,,,,draw=none])} \arrow{0}[,0] \+ \chemfig{@{a2}\ch{NO3+}} \arrow \chemfig{**[120,420]6(----@{a3}([:60]-H)([:120]-\ch{NO3})(-[:270,,,,draw=none]+)--)}
		\chemmove[shorten <=4pt,shorten >=2pt]{\draw (a1)..controls +(90:5mm) and +(135:10mm) ..(a2); }
	}

	\chfg{
		\chemfig{**[120,420]6(----(-[@{a1}:60]@{a2}H)([:120]-\ch{NO3})(-[@{a3}:270,,,,draw=none]+)--)} \arrow{0}[,0] \+ \chemfig{@{a4}\lewis{3:,\ch{O-}}-S(=[:90]O)(=[:270]O)-OH} \arrow \chemfig{!\upbenz(-[:90]\ch{NO3})} \arrow{0}[,0] \+ \ch{H2SO4}
		\chemmove[shorten <=2pt,shorten >=2pt]{\draw (a1)..controls +(330:4mm) and +(30:4mm) ..(a3); }
		\chemmove[shorten <=6pt,shorten >=2pt]{\draw (a4)..controls +(135:5mm) and +(300:5mm) ..(a2); }
	}

	\noindent \textbf{Halogenation}

	Reagents: \ch{X2} (\ch{X} \(\in\) \ch{Cl,Br,I}) \\
	Conditions: \ch{?} \\

	Electrophile is generated by homolytic fission of \ch{X-X} bond due to \ch{e-} concentration in benzene ring. \\

	\noindent \textbf{Friedel-Craft Alkylation}

	Reagents: \ch{R-Cl} \\
	Conditions: \ch{AlCl3} \\

	Generation of Electrophile: \ch{R-Cl + AlCl3 -> R+ + AlCl4-} \\

	\noindent \textbf{Friedel-Craft Acylation}

	Reagents: \ch{R-COCl} \\
	Conditions: \ch{AlCl3} \\

	Generation of Electrophile: \ch{R-COCl + AlCl3 -> RCO+ + AlCl4-} \\

	\subsubsection{Oxidation}

	Though the benzene ring itself is unreactive, any alkyl chains bonded to a benzene ring can undergo oxidation, replacing the alkyl group with a \ch{COOH} group. However, the reaction requires that the benzyllic carbon has at least 1 \ch{H} or 1 \ch{O} atom bonded to it, otherwise a case such as a tert-butyl benzene will not be oxidised due to steric effect of \ch{CH3} groups attached to the benzyllic \ch{C}.

	Reagents: \ch{KMnO4} \\
	Conditions: Warm \\

	\subsection{Activating and Deactivating Groups}

	Functional groups bonded to a benzene ring will distort the electron cloud of the benzene ring through two mechanisms. 
	Groups bonded to benzene rings can donate or withdraw electrons by `Resonance Effect' which arises when there is lone pairs (increasing reactivity) or \(\pi\) bonds (decreasing reactivity) at functional groups adjacent to the benzene. Groups can also affect reactivity through `Inductive Effect' which arises due to electronegativity of attached groups and may supply or withdraw \ch{e-} to the aromatic \(\pi\) \ch{e-} cloud. \\

	The functional groups presently on a benzene ring can determine the locations of where further electrophillic substitution reactions attach their functional groups onto. `Meta'-directing groups prefer addition to the third carbon in the benzene ring whereas `Ortho/Para'-directing groups prefer addition to the second or fourth carbon in the benzene ring. \\

	When there is more than one group on a benzene ring, follow the locations as determined by the strongest activating group on the benzene.
	
\end{document}