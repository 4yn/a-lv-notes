\documentclass[../main]{subfiles}

\begin{document}

\section{Periodicity II}

	\subsection{Group 2}

	Group 2 comprises the alkali earth metals, specifically \ch{Be}, \ch{Mg}, \ch{Ca}, \ch{Sr} and \ch{Ba}.

	\subsubsection{Reactivities}

	The reactivity of group 2 elements increase down the group. \\

	As the number of electronic shells increases, the radius of valence \usup{ns}{2} electrons increases and thus the attractive strength decreases. Ionization energy thus decreases down the group, reducing the first two ionization energies and thus making lower elements more easily lose their valence electrons.

	\subsubsection{Stability of Carbonates}

	The thermal stability of group 2 carbonates decreases down the group. \\

	Down the group, the ionic radius of elements increases, decreasing the charge density of element's +2 charge. There is hence a decreasing extent of distortion of the electron cloud of \ch{CO3^{2-}} and a decreasing extent of weakening of covalent bonds in \ch{CO3^{2-}}. As a result, more heat energy is required to break the covalent bonds in \ch{CO3^{2-}}, increasing the thermal stability down the group.

	\subsection{Group 17}

	Group 17 comprises the halogen gases, specifically \ch{F}, \ch{Cl}, \ch{Br}, \ch{I} and \ch{At}.

	\subsubsection{Bond Dissociation Energy}

	The BDE of diatomic group 17 elements decreases down the group, with the exception of \ch{F} which has a lower BDE than \ch{Cl}. \\

	Down the group, the number of electron shells increases, increasing the size of the valence orbitals. As the orbitals become more diffuse, there is a decreased effectiveness of overlap between orbitals, hence reducing the bond strength. \\

	The \ch{F-F} bond is weaker than the \ch{Cl-Cl} bond as the small size of \ch{F} causes inter-orbital repulsion of lone pairs and bonded pairs. \\

	\subsubsection{Oxidizing Power}

	The oxidizing strength of group 17 elements decreases down the group. \\

	As the number of electronic shells increases, the radius of valence orbitals increases, therefore decreasing the attraction experienced by valence electrons. Atoms are hence less able to attract electrons to form anions, decreasing their ability to be reduced and hence decreasing their oxidizing strength. \\

	\ch{F} and \ch{Cl} have positive \usup{E}{\plimsoll}, indicating that they are able to oxidize water in aqueous solution to form \ch{HF} and \ch{HCl}. \ch{Br} and \ch{I} on the other hand have negative \usup{E}{\plimsoll} and thus will not oxidize water in standard conditions. \\

	A more oxidizing halogen can displace a less oxidizing halogen from a compound. A halogen further up in the group is able to oxidize a halogen lower in the group. \\

	\noindent \textbf{Reactions with Thiosulfate}

	\ch{Cl2} and \ch{Br2} is able to oxidize \ch{S2O32-} to \ch{SO4}, but \ch{I2} can only oxidize \ch{S2O3^{2-}} to \ch{S4O6^{2-}}.

	\subsubsection{Hydrogen Halides}

	The thermal stability of hydrogen halides decreases down the group. \\

	Down the group, the larger number of electron shells increases the radius of valence electrons, therefore resulting in less effective overlap of covalent bonds. Additionally, there is a decrease in electronegativity difference down the group, reducing bond polarity and hence bond strength. As a result the \ch{H-X} bond strength decreases down the group, decreasing the amount of thermal energy needed to break covalent bonds and thus the thermal stability decreases down the group.

	\subsubsection{Silver Halides and \ch{NH3}}

	Halides form precipitates when added to solutions containing \ch{Ag+}. \\

	In order to differentiate between different silver halides, the concentration of \ch{OH-} is altered in order to bring the ionic product of the various silver halides below their \usub{K}{sp}. \\

	\begin{center}
		\ch{Ag+ + X- <=> AgX} \\
		\ch{Ag+ + 2 NH3 <=> [Ag(NH3)2]+} 
	\end{center}

	After a silver halide precipitate is formed, \ch{NH3 (aq)} is added to solution which will form \ch{[Ag(NH3)2]+} complex. As the formation of \ch{[Ag(NH3)2]+} has a very large equilibrium constant, the concentration of \ch{Ag+} will drop sharply, which then shifts the first equilibrium far to the left and possibly dissolving more \ch{AgX}. \\

	As \usub{K}{sp} of silver halides decreases from \ch{Cl} to \ch{I}, \ch{AgCl} will be soluble when \ch{NH3 (aq)} is added while \ch{AgBr} will be soluble once \ch{NH3 (conc)} is added. \\

	Exposing silver halides to sunlight also turns white \ch{AgCl} gray and turns cream \ch{AgBr} yellow.

	\subsubsection{Solubility of \ch{I2}}

	\ch{I2} is sparingly soluble in water, but very soluble once a slight amount of \ch{KI} is present.

	\begin{center}
		\ch{I2 (s) <=> I2 (aq)} \\
		\ch{I2 (aq) + I- (aq) (pale~yellow) <=> I3- (aq) (brown)} 
	\end{center}

	Once a small amount of \ch{KI} is added, the second equilibrium is established which lies far to the right, allowing for the formation of the triiodide \ch{I3-} complex. This decreases the concentration of \ch{I2 (aq)}, where then the first equilibrium now lies far to the right as well, allowing for more solution of solid \ch{I2}.

\end{document}