\documentclass[../main]{subfiles}

\begin{document}

\section{Chemical Bonding I}

	\subsection{Chemical Bonding}

	\scidef{Chemical Bonds}{Chemical Bonds are binding forces of attraction between atoms, ions or molecules which result in a lower energy arrangement, involving the rearrangement of outer electrons of particles.}

	The ``Octet rule'' states that atoms tend to rearrange electrons until they have 8 electrons in a valence shell, but this ``rule'' has many exceptions. Usually it is only applied to noble gases other than helium and particles in the first and second periods of the periodic table. \\

	Molecules with less than 8 electrons in their valence shell are described as electron deficient and are able to  accept electrons to achieve an octet structure. Molecules with an odd number of electrons are described as radicals and readily form bonds with other radicals or other particles.

	\subsection{Covalent Bonds}

	\scidef{Covalent Bonds}{Covalent Bonds are the electrostatic forces of attraction between two positively charged nuclei and the shared pair of electrons.}

	Note that the definition states ``positively charged nuclei'' rather than ``atom'', the atom itself is not necessarily charged. \\

	Covalent bonds are generally formed by two atoms sharing electrons among each other, in order for an atom to gain electrons in its orbit to achieve the energetically stable octet / noble gas configuration. \\

	Dative covalent bonds occur when a shared pair of electrons in a covalent bond are provided by only one of the bonding atoms, drawn in structural formula as a arrow pointing from the donor to the recipient atom rather than a solid line. Dative covalent bonds are received by atoms when they have vacant, low-lying orbitals to accept electrons.

	\subsubsection{Sigma and Pi Bonds}

	Sigma \(\sigma\) bonds involve the overlap of two electron orbitals head-on. Pi \(\pi\) bonds involve the overlap of two electron orbitals side-to-side. Single covalent bonds have one \(\sigma\) bond, while double or triple bonds have one \(\sigma\) and one or two \(\pi\) bonds. \(\sigma\) bonds are stronger than \(\pi\) bonds since head-on electron orbital overlap is stronger than side-to-side overlap. This is used to explain the fact that the strength of a double bond is less than twice that of a single bond (mathematically speaking, \( \sigma + \pi < 2\sigma \)).

	\subsubsection{Factors Affecting Strength of Covalent Bonds}

	\scidef{Strength of a Covalent Bond}{The Strength of a Covalent Bond is measured by its Bond Energy / Bond Enthalpy, which is the average energy absorbed when one mole of a bond is broken in a gaseous state.}

	A larger number of bonds between atoms result in stronger covalent bonds. More electrons are shared between atoms, creating a higher density of electrons in the inter-nuclear region, hence there is increased electrostatic attraction between bond pairs and nuclei, hence increasing bond strength. This is used when comparing bonds of different\\

	A larger atomic radius results in weaker covalent bonds. A larger electron cloud size among bonded atoms means that bonded electrons are more diffuse and spread out, hence there is a decrease in the ``effectiveness of overlap'' of larger atoms as compared to smaller atoms, hence there is a decrease in electrostatic attraction between bond pairs and nuclei and a reduction in bond strength. This is typically used to compare between bonds involving atoms with significantly different atomic radius and atoms of the same group. \\

	A larger difference in electronegativity results in stronger covalent bonds. Differences in electronegativity in bonded atoms result in the formation of partial charges among the bonded atoms, where the more electronegative atom attracts bonded electrons strongly, hence there is stronger electrostatic attraction due to the presence of opposite partial charges, hence increasing bond strength. This is typically used to compare between bonds involving atoms with significantly different electronegativities and in cases where bonds involve Fluorine.

	\subsubsection{Implications of Stronger Covalent Bonds}

	Stronger covalent bonds reduce the distance between two nuclei, reducing the bond length. \\

	Strongly covalently bonded atoms are less likely to react.

	\subsection{Intermediate Bond Types}

	\subsubsection{Covalent Bonds with Ionic Character}

	Some covalent bonds involve atoms with different degrees of electronegativity. With this difference in electronegativity, electrons are more strongly attracted to one nucleus than the other, making the electron pair(s) in a covalent bond unequally shared, giving rise to a polar covalent bond. \\

	\scidef{Dipole Moment}{Dipole moment is the quantification of the degree of polarity of a bond, calculated as a product of the charge charge and distance between charges. The greater the dipole moment, the more polar the bond.}

	Dipole moment is drawn as an arrow with a perpendicular line at its other end, pointing towards the more electronegative atom. Dipole moment of a molecule is the vector sum of all dipole moments in its bonds, and a molecule with a non-zero dipole moment is considered a polar molecule.

	\subsubsection{Ionic Bonds with Covalent Character}

	Cations attract the electron cloud of the anion and distorts the electron cloud towards the cation, causing covalent character in an ionic bond. The degree of covalent character in an ionic bond is dependent on the ability of the cation to polarize the anion by having a dense positive charge (high charge and small radius) and the ability of the the anion to be polarized by having a large electron cloud. \\

	Typically ionic compounds which have a large degree of covalency result in covalent bonds being formed. This gives rise tho the case where \ch{Al F3} is ionic but \ch{Al Cl3} since Al has a high charge density and Cl has a larger electron cloud size than F. Therefore, \ch{Al F3} and \ch{Al2 O3} are ionic but \ch{Al Cl3} is covalent.

	\subsection{Intermolecular Forces of Attraction}

	Simple Covalent substances exist as simple, small molecules. Physical state changes (melting, boiling) involve the manipulation of Intermolecular Forces of Attraction (IMF) rather than breaking or forming covalent bonds.

	\subsubsection{Instantaneous Dipole-Induced Dipole}

	Instantaneous dipole-induced dipole (id-id) interactions occur in all simple covalent molecules since all molecules have an electron cloud. At a point in time the random motion of electron orbitals may cause asymmetrical distribution of electrons to create an instantaneous dipole, which then induces a dipole in a neighboring particle, creating synchronized motion between two particles and hence a attraction between them. id-id interactions are short lived because instantaneous dipoles do not last as long. The strength of id-id interactions is hence considered weak. \\

	The larger the number of electrons in a molecule and the larger the surface area of a molecule (straight chain hydrocarbons vs branched hydrocarbons), the stronger the id-id interaction. \\

	\subsubsection{Permanent Dipole-Permanent Dipole}

	Permanent dipole-permanent dipole (pd-pd) interactions occur in all polar covalent molecules. Polar molecules align themselves such that their positive dipole is in line with the negative dipole of other molecules and vice versa, where the electrostatic attraction between dipoles gives rise to pd-pd interactions. \\

	The larger the dipole moment of an electron, the stronger its pd-pd interactions. \\

	In most cases where the number of electrons between molecules is similar, the molecule with a dipole has a higher melting or boiling point because it has pd-pd ON TOP OF the id-id which nonpolar molecules have. However, id-id interactions have a higher upper bound of strength and if the electron cloud size is large enough, id-id interactions can be stronger than pd-pd interactions of a smaller molecule.

	\subsubsection{Hydrogen Bond}

	Hydrogen bonds are a special category of pd-pd interaction which is exceptionally strong due to the small but highly electronegative hydrogen atom. F, O or N bonded to H is highly polar and causes electron density to be highly withdrawn from the H atom, making the H atom have a very high positive dipole, which then readily attracts a lone pair from another F, O or N from an adjacent molecule. F, O or N are required due to their high electronegativity which is able to induce a positive dipole on H and their small size which can then bring a lone pair close to the H atom for hydrogen bonds to form. \\

	Hydrogen bonds are the strongest IMF. When comparing strength of H-bonding between different molecules which are capable of forming H-bonds, highlight the number of hydrogen bonds it can form on average per molecule, which is limited by the number of H bonded to F, O and N or the number of lone pairs in F, O and N, whichever is lower, and identify which molecule has more extensive H-bonding. Strength of a H-bond also depends on the dipole moment of a bond between H and some atom hence \ch{HF} has higher melting/boiling points than \ch{NH3}. \\

	\ch{H2O}, when frozen, creates a highly-ordered tetrahedral lattice among molecules due to hydrogen bonding, where each oxygen atom is bonded to 4 hydrogen atoms, causing water to expand when frozen. \\

	H-bonding in organic molecules with the carboxylic acid group \ch{COOH} form dimers in gaseous state due to the presence of \ch{O-H} bonds. \\

	Intramolecular H-bonding can occur in molecules where its H and F/O/N molecules which could be used for hydrogen bonding instead bond within a molecule instead of between molecules, reducing sites available for H-bonding and reducing the overall IMF.

	\subsubsection{Solubility}

	For a solute to dissolve in a solvent, the intermolecular forces in the solute and solvent should be similar. This is because the energy released in the formation of IMF between solvent and solute needs to be large enough to overcome IMF between solvent and IMF between solute, and is hence ensured by having similar types of IMF. \\

	Simple molecules with the same type of intermolecular bonds mix well. If the solute-solute interaction and the solvent-solvent interaction is the same as the solvent-solute interaction, dissolution will be favorable because there is sufficient energy released in the formation of IMF to break preexisting IMF. If not, dissolution will be unfavorable because one of the interactions are weaker than the other and energy released in formation of IMF is not sufficient to overcome preexisting IMF. \\

	As an exception to this rule, ionic solids (pd-pd) also tend to dissolve in water (H-bonds) because of the strong ion-dipole interactions with polar molecules and compensate for overcoming strong ionic bonds in the solid and H-bonds among water molecules. \\

	Another addition to this rule would be the case where a solute reacts with solvent to create products which are able to form favorable interactions with the solvent (to follow the first or second rule).

	\subsection{Ionic Bonds}

	\scidef{Ionic Bonds}{Ionic Bonds are the electrostatic forces of attraction between two oppositely charged ions.}

	Ions are usually formed through the transfer of electrons from one (usually metallic) atom to another (usually non-metallic) atom. Ions in a solid are held in fixed and orderly arrangements.

	\scidef{Coordination Number}{The Coordination number is the number of nearest neighbors to an atom}

	Ionic solids generally have high melting and boiling points above 500 \degree C and are all solids at room temperature. Ionic solids are generally soluble in polar solvents, conduct electricity in molten or aqueous states and are hard and brittle.

	\subsubsection{Factors Affecting Strength of Ionic Bond}

	\scidef{Lattice Energy}{Lattice Energy is the energy released when one mole of ionic crystalline solid is formed from its constituent gaseous ions.}

	Ionic bonds are generally strong. The strength of an ionic bond, or its lattice energy is related to the charges on an ionic compound's constituent ions and their radii:

	\scieqn{Lattice Energy}{The Lattice Energy LE of an ionic compound is dependent on its cationic charge \(q_+\), anionic charge \(q_-\) and interionic distance \(r_+ + r_-\):}{| \text{LE} | \propto | \frac{q_+ \times q_-}{r_+ + r_-} | }

	\subsection{Giant Covalent Molecules}

	A giant covalent molecule is made of atoms held together in an extensive network by covalent bonds, such as graphite and quartz.

	\subsubsection{Diamond}

	Diamond contains molecules of carbon which are covalently bonded to 4 other carbon atoms in a tetrahedral arrangement. Strong  covalent bonds between carbon atoms mean that the molecule has a high melting and boiling point and is very strong. Diamond also has no unbonded electrons and hence is an insulator.

	\subsubsection{Graphite}

	Graphite is carbon in a layered structure, made of planes of bonded hexagonal rings of carbon atoms. Each carbon atom is singly bonded with other carbon atoms at a 120 \degree angle, leaving one electron in a p orbital not involved in bonding, hence forming an extended \(\pi\) electron cloud above and below a layer of carbon atoms. This electron cloud contains delocalised electrons which conduct electricity. \\

	Strong covalent bonds between carbon atoms mean that the boiling point of graphite is very high. However, layers of graphite are held together by weak IMF which then allow graphite to be malleable and soft as they can glide over each other.

	\subsubsection{Quartz}

	Quartz \ch{SiO2} contains silicon atoms covalently bonded to 4 oxygen atoms in a tetrahedral shape while oxygen atoms are bonded to 2 silicon atoms. Due to strong covalent bonds and its rigit 3-dimensional structure, quartz is hard and insoluble as well has having a high melting and boiling point.

	\subsection{Metallic Bonds}

	Metals are composed of a rigid lattice of positive ions surrounded by a ``sea of electrons''. \\

	Metals have high electrical conductivity in solid and liquid states due to the availability of free electrons as mobile charge carriers. Metals are also good conductors of heat as mobile electrons are fast-moving and mobile. Metals are also malleable and ductile as non-directional metallic bonds allow layers of metals to glide over each other without breaking metallic bonds. Metals are closely packed and have high densities. Metals generally have high melting and boiling points.

	\subsection{Factors Affecting Strength of Metallic Bond}

	Metallic bonds are strong and non-directional, with each nucleus attracting electrons in its surroundings. \\

	A higher number of valence electrons in a atom result in stronger metallic bonds.\\

	A smaller cationic size of a metal atom results in stronger metallic bonds. A metallic cation of smaller radius will have a higher charge density and hence have a stronger attraction to delocalised electrons.\\

	\subsection{Geometry of Molecules}

	\subsubsection{Dot and Cross}

	Dot and cross representations aim to show the distribution of electrons in a particle, especially highlighting which electrons originate from which atom. \\

	Ionic Dot and Cross diagrams are drawn using one formula unit of the ionic compound. Ions are drawn enclosed within square brackets with their charge written in superscript to the right. Multiple atoms in a ionic compound are represented with a coefficient to the left of the dot and cross diagram. Typically, the metallic ion has no more electrons in its valence shell hence when required to draw valence electrons only the metallic atom typically has no electrons surrounding it. Be sure to show that the non-metallic atom has received electrons from an external source by drawing a suitable number of electrons with a different sign surrounding the nonmetal atom. \\

	Covalent Dot and Cross diagrams are drawn such that the most number of atoms have a octet configuration, other than the \ch{H} atom which only has two electrons. Make the most electronegative atoms the central atoms, give each other atom a single covalent bond between central atom and then provide electrons to the external nuclei to allow them to achieve octet structure. Afterwards, calculate the total number of electrons involved in bonding and insert the remaining electrons into the central atom. Rearrange the bonds to ensure octet structure throughout, making some bonds dative, double, triple as suitable.\\

	For Dot and Cross diagrams of charged molecules, electrons are gained by the most electronegative atom and electrons are lost by the least electronegative atom, and overall charge must also be displayed as a square bracket with superscript. \\

	\textbf{NOT TAUGHT: Formal Charge} is a calculated quantity of how stable a covalently bonded atom is in its current state of bonding, calculated as the number of valence electrons an atom has minus the number of lone electrons and minus the number of bonds pairs. Minimizing this number when drawing a covalently bonded molecule will ensure that it is a energetically feasible configuration.

	\subsubsection{VESPR}

	Covalent bonds are directional and hence have predictable shapes. Valence Shell Electron Pair Repulsion (VESPR) Theory is used to predict the shape of covalently bonded molecules, stating that electron pairs repel each other and are arranged such that repulsion is minimized. Lone pair-lone pair repulsion is stronger than lone pair-bond pair, which is stronger than bond pair-bond pair. Single, unpaired electrons have a weaker repulsion than all of these. Lone pairs exert larger repulsion than bond pairs because they are only attracted by one nucleus. \\

	Shapes of covalently bonded molecules are derived by counting the number of areas of electron density (a lone electron/double bond/triple bond counts as one region) to derive electron-pair geometry. The number of lone electron pairs and bond electron pairs are then used to infer the molecular geometry of a molecule. \\

	Specific bond angles to remember would be that tetrahedral molecules with 4 bond pairs or 4 lone pairs at 109.5\degree, 3 bond pairs at about 107\degree and 2 bond pairs at about 105\degree (Water at 104.5\degree). An decreasing electronegativity of the central atom also results in an increased deviation of bond angle since the bond pairs are further from the central nucleus and exert less repulsion. \\

	Three dimensional bonds are drawn with triangles instead of lines. One vertice resides at the central atom while two reside at the external atom. Shaded triangles mean that a bond extends out of the paper, while a triangle of vertical lines mean that a bond extends into the paper. \\

	Please refer to page 10 of the lecture notes for the full VESPR table.

	\subsection{Interpretation of Physical Properties}

	High melting and boiling points are an indicator of strong interactions between particles: IMF in simple covalent molecules, covalent bonds in giant covalent molecules, ionic bonds in ionic substances and metallic bonds in metals. \\

	Conduction of electricity indicates the presence of mobile charge carriers like electrons or ions in a substance.

	\subsection{Exam Technique}

	\scidef{Bulk Property}{A Bulk Property is a physical property which is constant no matter the size or amount of substance in a system.}

	\subsubsection{In Terms of Structure and Bonding}

	When prompted with ``in terms of structure and bonding'', be sure to identify the type of molecule (simple covalent, giant covalent, ionic, metallic) and its type of bond (covalent + IMF, covalent, ionic, metallic). \\

	For questions regarding simple covalent molecules, identify the molecular shape and whether the molecule is polar using VESPR and other concepts.

	\subsubsection{Melting / Boiling Points}

	[Insert property here], hence the [high/low] electrostatic forces of attraction of [type of bond or whatever] mean that a [large/small] amount of energy is required to overcome these electrostatic forces of attraction, hence there is a [high/low] melting and boiling point.

\end{document}	