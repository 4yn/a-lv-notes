\documentclass[../main]{subfiles}

\begin{document}

\section{Equilibrium I}

\subsection{Equilibrium Properties}

\subsection{Le Chateliers' Principle}

\scidef{Le Chateliers' Principle}{Le Chateliers' Principle (LCP) states that when a system in equilibrium is disturbed, it will attempt to counteract the change by favoring the reaction which reverses the disturbance.}

LCP is a convenient explanation which amalgamates the observed reactions to disturbances in equilibrium systems. Disturbances in the form of adding or subtracting reactants, increasing or decreasing pressure and volume all change \usub{Q}{C} that just happen to follow this rule. Disturbances in the form of temperature changes affect \usub{K}{C} also happen to follow this rule since \usub{K}{C} is effectively one rate constant divided by another, and rate constants are dependent on temperature. \\

In exams, just write the following standard answer: \\

\end{document}