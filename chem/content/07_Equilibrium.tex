\documentclass[../main]{subfiles}

\begin{document}

\section{Equilibrium I}

	\subsection{Dynamic Equilibrium}

	\begin{center} \ch{aA + bB <=> cC + dD} \end{center}

	\scidef{Reversible Reaction}{A Reversible Reaction is a reaction which does not proceed to completion and can proceed both in forward and reverse, where eventually a mixture of reactants and products are acquired.}

	\scidef{Dynamic Equilibrium}{A Dynamic Equilibrium is the state in a reversible reaction when the macroscopic properties (concentration, pressure) appear to remain constant, implying that the rate of forward reaction is equal to the rate of reverse reaction which leads to zero net change.}

	\subsection{Equilibrium Properties}

	Consider a reaction as above. From Kinetics, we understand the rate of reactions of the forward reaction has a rate equation of \(\text{rate} = k_\text{fwd}\ch{[A]^a[B]^b}\) while the reverse reaction has a rate equation of \(\text{rate} = k_\text{rev}\ch{[C]^c[D]^d}\). \\

	In a dynamic equilibrium, the rate of the forward and reverse reactions are equal. Through manipulating the above equations, we obtain a special quantity \(K_C\), known as the equilibrium constant, for values of concentrations once equilibrium has been reached.

	\begin{equation*} \begin{gathered}
	k_\text{fwd}\ch{[A]^a[B]^b} = k_\text{rev}\ch{[C]^c[D]^d} \\
	K_C = \frac{k_\text{fwd}}{k_\text{rev}} = \frac{\ch{[C]^c[D]^d}}{\ch{[A]^a[B]^b}}
	\end{gathered} \end{equation*}

	Calculating the value of \(K_C\) when using values of concentrations not at equilibrium will instead obtain the value of \(Q_C\), the reaction quotient. Reaction will hence continue so that \(Q_C \rightarrow K_C\) and finally \(Q_C = K_C\). \\

	\scidef{Position of Equilibrium}{The Position of Equilibrium is a description of the relative composition of products and reactants in a reaction mixture.}

	In a gaseous reaction environment, the concentration of gaseous reactants will be proportional to the partial pressures of the reactants, hence an alternative equilibirum constant \(K_P\) can be used, where

	\[ K_P = \frac{p_A^a p_B^b}{p_C^c p_D^d} \]

	From the above derivation, the equilibrium constant could be expressed as the ratio between two rate constants. From the Arrhenius equation, \(K_C\) can be inferred to be dependent on temperature. \(K_C\) is not affected by catalysts, changes in total pressure or concentrations. \\

	A large/small value of \(K_C\) suggests the position of equilibrium of a reaction lies to the right/left, implying there is complete/no reaction.

	\subsubsection{Manipulating Equilibrium Constants}

	The equilibrium constant of a forward reaction is the reciprocal of the equilibrium constant of the reverse reaction.

	\[ K_\text{forward} = K_\text{reverse}^{-1} \]

	For a multi-step reaction, the overall equilibrium constant is the product of all the reactions.

	\[ K_\text{total} = K_1 \times K_2 \times K_3 ... \]

	When the stoichiometric coefficient of a reaction is multiplied by a product n, the equilibrium constant is raised to the same power n.

	\[ K_\text{new} = K_\text{old}^n\]

	\subsection{Gibbs Free Energy and Equilibria}

	The \(\Delta G^\plimsoll\) of a reaction determines if a reaction is spontaneous. The \(\Delta G\) value, defined as the instantaneous change in \(\Delta G^\plimsoll\) for a certain composition of a reaction mixture, can be seen to be a description of the instantaneous change in \(\Delta G^\plimsoll\). Combine this with the knowledge that energetic systems tend towards the lowest energy level and we conclude that equilibrium is reached when \(\Delta G = 0\). \\

	The graph of the \(\Delta G^\plimsoll\) of a system against progress of reaction is a U-shaped curve, because \(\Delta S\) increases as product and reactant is mixed and hence leading to a U-shaped dip due to the \(- T \Delta S\) term is superimposed on the straight line of the \(\Delta H\) term. \\

	From the graphs, we observe that the position of equilibrium is solely dependent on \(\Delta G^\plimsoll\), leading to the equation: 

	\[ \Delta G^\plimsoll = R T \ln(K_C) \]

	\subsection{Homogeneous and Heterogeneous Equilibria}

	Equilibrium systems can be classified according to what physical state its reactants are in. \\

	\scidef{Homogeneous Equilibrium}{A Homogeneous Equilibrium is one whose reactants and products are all in the same physical state.}

	\scidef{Heterogeneous Equilibrium}{A Heterogeneous Equilibrium is one whose reactants and products are in different states.}

	Pure reactants and products which are liquid (not aqueous) and solid form have a fixed concentration at a fixed temperature, and hence can be ignored when calculating \(K_C\). Water, since it is present in large quantities in an aqueous system, can also be ignored in calculation of \(K_C\) should the reaction involve water since its concentration is approximately constant.

	\subsection{ICE Tables}

	When given information about the reactants and products in an equilibrium system, the quantities of each substance can be calculated through the use of an Initial-Change-Equilibrium table. The header of each column corresponds to a balanced reversible reaction (with state symbols for safety) and each row is labeled as ``Initial Concentration/Pressure/Amount'' with the corresponding units.\\

	Quantities such as the  concentration, amount or partial pressures of reactants can be used to fill in the ICE table because all of these quantities are stoichiometrically related. However, when calculating or using \(K_C\) or \(K_P\) the volumes of the system need to be accounted for in the case where the two sides of a reaction have different amounts of products. \\

	The total pressure may also be a useful quantity when constructing ICE tables.

	\scidef{Degree of Dissociation}{The Degree of Disassociation \(\alpha\) is the fraction of a reactant that is dissassociated at a particular temperature, where \[\alpha = \frac{\text{Amount of reactant dissassociated}}{\text{Initial amount of reactant}}\]}

	\subsection{Le Chateliers' Principle}

	\scidef{Le Chateliers' Principle}{Le Chateliers' Principle (informally abbreviated LCP) states that when a system in equilibrium is disturbed, it will attempt to counteract the change by favoring the reaction which reverses the disturbance.}

	LCP is a convenient explanation which amalgamates the observed reactions to disturbances in equilibrium systems. Disturbances in the form of adding or subtracting reactants, increasing or decreasing pressure or volume all change \usub{Q}{C} that happen to follow this rule. Disturbances in the form of temperature changes affect \usub{K}{C} also happen to follow this rule since \usub{K}{C} is effectively a ratio between rate constants and rate constants are dependent on temperature. \\

	In exams, just write the following standard answer: \\

	[Change in System], equilibrium system counteracts by favoring [forward/reverse] reaction which [exo/endothermic, more/less gaseous particles, more reactant/product etc], [product/reactant increases/decreases]. 

	\subsection{Industrial Processes}

	Industrial processes involving obtaining products from an equilibrium reaction are tasked with obtaining a product cost-effectively, therefore maximizing yield, maximizing speed of reaction and minimizing cost (in equipment needed to maintain the equilibrium environment) . One such process would be the Haber Process to obtain Ammonia.

	\begin{center} \ch{N2 + 3 H2 <=> 2 NH3} \(\quad \Delta H^\plimsoll = \SI{92}{\kJ\per\mol}\) \end{center}

	The Haber Process is conducted at \SI{450}{\celsius} as a compromise to increase the speed of reaction by supplying a large amount of heat and also to use a moderate temperature so as to have a higher yield of \ch{NH3} at equilibrium. \\

	The Haber Process is conducted at \SI{200}{\atm} to shift the equilibrium to the right by favoring the reaction which decreases gaseous molecules but also to moderate the cost of maintaining a industrial setup at high pressures. \\

	The Haber Process also involves the addition of iron catalyst and aluminum oxide promoter to speed up reaction, a continous removal of ammonia gas to shift equilibrium to the right and increase yield and finally also introduces reactants in stoichiometric quantities to minimize excess.

	\begin{center} \ch{2 SO2 + O2 <=> 2 SO3} \(\quad \Delta H^\plimsoll = \SI{-197}{\kJ\per\mol}\) \end{center}

	Another industrial process is the Contact Process used to form Sulfur Trioxide, where the reaction takes place at \SI{450}{\celsius}, \SI{1}{\atm} and with the presence of Vanadium Oxide \ch{V2O5(s)} catalyst to obtain 99\% yield.

\end{document}