\documentclass[../main]{subfiles}

\begin{document}

\section{Equilibrium I}

	\subsection{Dynamic Equilibrium}

	\begin{center} \ch{aA + bB <=> cC + dD} \end{center}

	\scidef{Reversible Reaction}{A Reversible Reaction is a reaction which does not proceed to completion and can proceed both in forward and reverse, where eventually a mixture of reactants and products are acquired.}

	\scidef{Dynamic Equilibrium}{A Dynamic Equilibrium is the state in a reversible reaction when the macroscopic properties (concentration, pressure) appear to remain constant, implying that the rate of forward reaction is equal to the rate of reverse reaction which leads to zero net change.}

	\subsection{Equilibrium Properties}

	Consider a reaction as above. From Kinetics, we understand the rate of reactions of the forward reaction has a rate equation of \(\text{rate} = k_\text{fwd}\ch{[A]^a[B]^b}\) while the reverse reaction has a rate equation of \(\text{rate} = k_\text{rev}\ch{[C]^c[D]^d}\). \\

	In a dynamic equilibrium, the rate of the forward and reverse reactions are equal. Through manipulating the above equations, we obtain a special quantity \(K_C\), known as the equilibrium constant, for values of concentrations once equilibrium has been reached.

	\begin{equation*} \begin{gathered}
	k_\text{fwd}\ch{[A]^a[B]^b} = k_\text{rev}\ch{[C]^c[D]^d} \\
	K_C = \frac{k_\text{fwd}}{k_\text{rev}} = \frac{\ch{[C]^c[D]^d}}{\ch{[A]^a[B]^b}}
	\end{gathered} \end{equation*}

	Calculating the value of \(K_C\) when using values of concentrations not at equilibrium will instead obtain the value of \(Q_C\), the reaction quotient. Reaction will hence continue so that \(Q_C \rightarrow K_C\) and finally \(Q_C = K_C\). \\

	\scidef{Position of Equilibrium}{The Position of Equilibrium is a description of the relative composition of products and reactants in a reaction mixture.}

	In a gaseous reaction environment, the concentration of gaseous reactants will be proportional to the partial pressures of the reactants, hence an alternative equilibirum constant \(K_P\) can be used, where

	\[ K_P = \frac{p_A^a p_B^b}{p_C^c p_D^d} \]

	From the above derivation, the equilibrium constant could be expressed as the ratio between two rate constants. From the Arrhenius equation, \(K_C\) can be inferred to be dependent on temperature. \(K_C\) is not affected by catalysts, changes in total pressure or concentrations. \\

	A large/small value of \(K_C\) suggests the position of equilibrium of a reaction lies to the right/left, implying there is complete/no reaction.

	\subsubsection{Manipulating Equilibrium Constants}

	The equilibrium constant of a forward reaction is the reciprocal of the equilibrium constant of the reverse reaction.

	\[ K_\text{forward} = K_\text{reverse}^{-1} \]

	For a multi-step reaction, the overall equilibrium constant is the product of all the reactions.

	\[ K_\text{total} = K_1 \times K_2 \times K_3 ... \]

	When the stoichiometric coefficient of a reaction is multiplied by a product n, the equilibrium constant is raised to the same power n.

	\[ K_\text{new} = K_\text{old}^n\]

	\subsection{Gibbs Free Energy and Equilibrium}

	\subsection{Le Chateliers' Principle}

	\scidef{Le Chateliers' Principle}{Le Chateliers' Principle (informally abbreviated LCP) states that when a system in equilibrium is disturbed, it will attempt to counteract the change by favoring the reaction which reverses the disturbance.}

	LCP is a convenient explanation which amalgamates the observed reactions to disturbances in equilibrium systems. Disturbances in the form of adding or subtracting reactants, increasing or decreasing pressure and volume all change \usub{Q}{C} that just happen to follow this rule. Disturbances in the form of temperature changes affect \usub{K}{C} also happen to follow this rule since \usub{K}{C} is effectively one rate constant divided by another, and rate constants are dependent on temperature. \\

	In exams, just write the following standard answer: \\

	\subsection{The Haber Process}

\end{document}