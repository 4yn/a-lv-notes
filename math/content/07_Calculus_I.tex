\documentclass[../main]{subfiles}

\begin{document}

\section{Calculus I}

\subsection{Differentiation}

	\subsubsection{Standard Differentiation Forms}

	\begin{equation*} \begin{aligned}
	\frac{d}{dx} f(x) &= f'(x) \\
	\frac{d}{dx} f(x)^n &= f'(x) \quad n \quad f(x)^{n-1}  \\
	\frac{d}{dx} ln(f(x)) &= \frac{f(x)}{f'(x)} \\
	\frac{d}{dx} e^{f(x)} &= f'(x) \quad e^{f(x)}  \\
	\frac{d}{dx} k^{f(x)} &= \frac{d}{dx} e^{f(x)ln(k)} \\
	&= k^{f(x)}ln(k) f'(x)
	\end{aligned} \end{equation*}

	\subsubsection{Standard Differentiation Methods}

	Chain Rule: \( \frac{dy}{dx} = \frac{du}{dx} \frac{dy}{du} \) \\
	Product Rule: \( \frac{d}{dx} f(x)g(x) = f(x)g'(x) + g(x)f'(x) \) \\
	Quotient Rule: \( \frac{d}{dx} \frac{f(x)}{g(x)} = \frac{g(x)f'(x) - f(x)g'(x)}{g(x)^2} \) 

	\subsubsection{Implicit Differentiation}

	\begin{equation*} \begin{aligned}
		f(x) &= g(y) \\
		\frac{d}{dx} f(x) &= \frac{d}{dx} g(y) \\
		&= \frac{d}{dx} \frac{dy}{dy} g(y) \\
		&= \frac{dy}{dx} \frac{d}{dy} g(y) \\
		\frac{dy}{dx} &= \frac{d}{dx} f(x)/ \frac{d}{dy} g(y) \\
		&= \frac{f'(x)}{g'(y)}
	\end{aligned} \end{equation*}

	Take differential of both sides and differentiate \(g(y)\) in terms of \(y\) and \(\frac{dy}{dx}\), then isolate \(\frac{dy}{dx}\) to obtain a suitable expression.

	\subsubsection{Trigonometric Derivatives}

	\begin{equation*} \begin{aligned}
		\frac{d}{dx} \sin(x) &= \cos(x) \\
		\frac{d}{dx} \cos(x) &= -\sin(x) \\
		\frac{d}{dx} \tan(x) &= \sec^2(x) \\
		\frac{d}{dx} \sec(x) &= \tan(x)\sec(x) \\
		\frac{d}{dx} \csc(x) &= -\csc(x)\cot(x) \\
		\frac{d}{dx} \sin^{-1}(x) &= \frac{1}{\sqrt{1-x^2}} \\
		\frac{d}{dx} \cos^{-1}(x) &= -\frac{1}{\sqrt{1-x^2}}  \\
		\frac{d}{dx} \tan^{-1}(x) &= \frac{1}{1+x^2}  \\
	\end{aligned} \end{equation*}

	\subsubsection{Parametric Differentiation}

	\begin{equation*} \begin{gathered}
		x = f(t) \quad y = g(t) \\
		\frac{dx}{dt} = f'(t) \quad \frac{dy}{dt} = g'(t)  \\
		\frac{dy}{dx} = \frac{dy}{dt} / \frac{dx}{dt} = g'(t) / f'(t) 
	\end{gathered} \end{equation*}

\subsection{Applications of Differentiation}

	\subsubsection{Minima and Maxima}

	Minimum points occur when a graph changes from increasing to decreasing gradient. Maximum points occur when a graph changes from increasing to decreasing gradient. \\

	To find minima and maxima, find where \(\frac{dy}{dx} = 0 \). To test whether a point is a minimum or a maximum, use the first derivative test or second derivative test.

	\subsubsection{First Order Derivative Test}

	\subsubsection{Second Order Derivative Test}

	Find \(x\) where \(\frac{dy}{dx} = 0\). Evaluate \(\frac{d^2y}{dx^2}\) at this value of \(x\). A positive value indicates a minimum point while a negative value indicates a maximum point.


	\subsubsection{Concavity and Points of Inflection}

	A range \([u,v]\) of a graph \(f(x)\) is concave down if all points within this range is equal to or above a line passing through \(f(u)\) and \(f(v)\). A range is concave up if all points within this range is equal to or above this line. Strictly concave graphs do not have the property of ``or equal to''. \\

	The concavity of a graph can also be related to gradient. A range \([u,v]\) of a graph \(f(x)\) is concave down if its gradient is  non-increasing across this interval. A range is concave if its gradient is non-decreasing across this interval. \\

	Points of inflection occur when there is a change from concavity. Therefore, to find points of inflection, set the derivative of the gradient to zero, hence set the second derivative to zero and solve: \(\frac{d^2y}{dx^2} = 0\).

\subsection{Graphing Techniques}

	\subsubsection{Graphs with Asymptotes}

	Consider graphs of the form \(y = \frac{f(x)}{g(x)}\). \\

	Observe that if there are points where \(g(k) \rightarrow 0\), there will be a vertical asymptote at the line \(x=k\). \\

	Also notice that if \(f(x)\) and \(g(x)\) are polynomials, partial fractions can be used to simplify the graph if the order of \(f(x)\) is larger than the order of \(g(x)\). \\

	Consider \(f(x) = ax^2 + bx + c\) and \(g(x) = dx + e\). \(y\) can then be simplified using partial fractions to obtain \(y\) in the form \(y = px + r + \frac{s}{dx+e}\). Such a graph will have a vertical asymptote at \(dx+e=0\) and have oblique asymptotes tending towards \(y = px + r\). \\

	Note that if \(a=0\) and hence \(f(x)\) is an order 1 polynomial, the oblique asymptote will become a horizontal asymptote. Also note that \(\lim_{y\to\pm\infty} y = px+r\) does not mean that \(y \neq px + r\) at any \(x\). There can be interceptions of the curve with the asymptote. \\

	To find range of values of \(y\), either locate turning points of the graph using differentiation or equate \(y\) to some constant and solve where discriminant is more than or equal to zero.

	\subsubsection{Conic Sections}

	Conic Sections are a special category of graphs because they can be derived through the intersection of a plane with a three-dimensional biconic function. \\

	Circles have equations of the form \( \frac{(x-a)^2}{r^2} + \frac{(y-b)^2}{r^2} = 1 \), where coordinates \((a,b)\) indicate the center of the circle and \(r\) is the radius of said circle. \\

	Ellipses have equations of the form  \( \frac{(x-h)^2}{a^2} + \frac{(y-k)^2}{b^2} = 1 \), where coordinates \((h,k)\) indicate the center of the ellipse, \(a\) is the largest distance from the center to any point in the x axis and \(b\) is the largest distance from the center to any point in the y axis. \\

	Hyperbola have equations of the form \( \frac{(x-h)^2}{a^2} - \frac{(y-k)^2}{b^2} = 1 \) or \( \frac{(y-k)^2}{a^2} - \frac{(x-h)^2}{b^2} = 1 \), where coordinates \((h,k)\) indicate the center of the curves, and its asymptotes are given by the equation \( y - k = \frac{b}{a} (x - h) \). \\

\subsection{Mclaurin Series}
	
	Any equation of a \(n\)-degree polynomial can be recovered from the values of its first \(n\) derivatives at \(x=0\).

	\begin{equation*} \begin{gathered}
		f(x) = a + bx + cx^2 + dx^3 \ldots \Rightarrow f(0) = a \\
		f'(x) = b + 2cx + 3dx^2 \ldots \Rightarrow f'(0) = b \\
		f''(x) = 2c + 6dx \ldots \Rightarrow f''(0) = 2c \\
		f'''(x) = 6d \ldots \Rightarrow f'''(0) = 6d \\
		\cdots \\
		f(x) = f(0) + f'(x) + \frac{f''(x)}{2!} +  \frac{f'''(x)}{3 !} \ldots \\
	\end{gathered} \end{equation*}

	By assuming that all equations can be approximated by a polynomial, approximations in a polynomial form can be given to non-polynomial functions. Values of these polynomials can be evaluated and used if the polynomial is convergent. I.e. for an arbitary function \(f(x)\) :

	\[ f(x) \approx f(0) + \frac{f'(0)}{1!} + \frac{f''(0)}{2!} + \frac{f'''(0)}{3!} \ldots \]

\subsection{Graph Transformatons}

	\subsubsection{Linear Transformations}

	Four linear transformations of graphs need to be known:

	\begin{itemize}
		\item Scale along x axis: \\
			Scaling by factor a: \( f(x) \Rightarrow f(\frac{x}{a}) \) 
		\item Scale along y axis: \\
			Scaling by factor a: \( f(x) \Rightarrow f(x)a \) 
		\item Translate along x axis: \\
			Shifting left a units: \( f(x) \Rightarrow f(x+a) \) \\
			Shifting right a units: \( f(x) \Rightarrow f(x-a) \)
		\item Translate along y axis:  \\
			Shifting up a units: \( f(x) \Rightarrow f(x)+a \) \\
			Shifting down a units: \( f(x) \Rightarrow f(x)-a \)
	\end{itemize}

	\subsubsection{Inverse Graphs}

	Transformation of a graph \(f(x)\) to a graph \(\frac{1}{f(x)}\) holds the properties:

	\begin{itemize}
		\item X intercepts \(\Rightarrow\) vertical asymptotes
		\item Vertical asymptotes \(\Rightarrow\) X intercepts with exclusion circle
		\item Horizontal asymptotes remain
		\item Exact coordinates for all marked points that are not on the x-axis 
	\end{itemize}

	\subsubsection{Derivative Graphs}

	Transformation of a graph \(f(x)\) to a graph \(f'(x)\) holds the properties:

	\begin{itemize}
		\item Maximum, minimum and points of inflection \(\Rightarrow\) x-intercepts
		\item Vertical asymptotes remain in location
		\item Oblique asymptotes \(\Rightarrow\) Horizontal asymptotes
	\end{itemize}

\end{document}