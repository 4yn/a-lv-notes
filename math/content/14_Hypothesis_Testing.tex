\documentclass[../main]{subfiles}

\begin{document}

\section{Hypothesis Testing}

	\subsection{Hypotheses}

	A null hypothesis \(H_0 : \mu = k\) is a proposed hypothesis of what property a random variable should hold. \\

	An alternative hypothesis \(H_1\: \mu \neq k\) is a proposed hypothesis which contradicts a null hypothesis, and describes how a null hypothesis will be tested. \\

	The Level of Significance of a test is the probability of the result of the test to incorrectly reject the null hypothesis, i.e. the chance the result is a false positive. Levels of Significance can range from \(0.000001\) in physical sciences to \(0.05\) in general sciences to \(0.3\) in social sciences. \\

	\subsection{Z-Test}

	A Z-test is a statistical test for the validity of a null hypothesis which follows a normal distribution. \\

	A null hypothesis is first established with its mean \(\mu\) and variance \(\sigma^2 \text{OR} s^2\), either from known data or using an unbiased estimate of a sample. \\

	A sample mean with \(n\) samples and mean \(\bar{x}\) is taken. This sample mean is then compared to the distribution of the null hypothesis with distribution \(X \sim N(\mu,\frac{\sigma^2}{n}\). \\

	The probability of the sample mean lying at the extremities of the null hypothesis is then calculated and compared to the level of significance of the test. \\

	\subsubsection{Syntax for p-value Test}

	To be done post-ct2. \\

	\subsubsection{Syntax for z-value Test}

	To be done post-ct2. \\

\end{document}