\documentclass[../main]{subfiles}

\begin{document}

\section{APGP}

\subsection{Arithmetic Progression}

	\subsubsection{Arithmetic Sequence}
	An arithmetic progression is a sequence of numbers which have the same difference between consecutive elements. Sequences are defined by their initial term \(a\) and their constant difference \(d\)
	\begin{equation*} \begin{gathered}
		u_1 = a + (1-1) d = a\\
		u_2 = a + (2-1) d = a + d \\
		u_3 = a + (3-1) d = a + 2d \\
		...\\
		u_n = a + (n-1) d 
	\end{gathered} \end{equation*} 
	\subsubsection{Arithmetic Series}
	Series is defined as the sum of a certain number of consecutive elements in a arithmetic sequence.
	\begin{equation*} \begin{gathered}
		S_1 = u_1 \\
		S_2 = S_1 + u_2 = u_1 + u_2 \\
		S_3 = S_2 + u_3 = u_1 + u_2 + u_3 \\
		... \\
		S_n = u_1 + u_2 + ... + u_{n-1} + u_n
	\end{gathered} \end{equation*} 
	For a known arithmetic sequence, the term \(S_n\) can be derived from the equation
	\[ S_n = \frac{n}{2}(2a + (n-1)d) = \frac{n}{2}(u_1 + u_n) \]
	\subsubsection{Proving a AP}
	\subsubsection{Exponent and Log}

\subsection{Geometric Progression}
	
	\subsubsection{Geometric Sequence}
	\subsubsection{Geometric Series}
	\subsubsection{Proving a GP}
	\subsubsection{Convergence}

\end{document}