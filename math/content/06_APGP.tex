\documentclass[../main]{subfiles}

\begin{document}

\section{APGP}

\subsection{Arithmetic Progression}

	\subsubsection{Arithmetic Sequence}
	An arithmetic progression is a sequence of numbers which have the same difference between consecutive elements. Sequences are defined by their initial term \(a\) and their constant difference \(d\)
	\begin{equation*} \begin{gathered}
		u_1 = a = a + (1-1) d\\
		u_2 = u_1 + d = a + d = a + (2-1) d \\
		u_3 = u_2 + d = a + d + d = a + (3-1) d\\
		...\\
		u_n = a + (n-1) d 
	\end{gathered} \end{equation*} 
	\subsubsection{Arithmetic Series}
	An Arithmetic Series is defined as the sum of a certain number of consecutive elements in an arithmetic sequence.
	\begin{equation*} \begin{gathered}
		S_1 = u_1 \\
		S_2 = S_1 + u_2 = u_1 + u_2 \\
		S_3 = S_2 + u_3 = u_1 + u_2 + u_3 \\
		... \\
		S_n = u_1 + u_2 + ... + u_{n-1} + u_n
	\end{gathered} \end{equation*} 
	For an arithmetic sequence of known initial term and constant difference, the term \(S_n\) can be derived from the equation
	\[ S_n = \frac{n}{2}(2a + (n-1)d) = \frac{n}{2}(u_1 + u_n) \]
	\subsubsection{Proving an AP}
	To prove an AP, show that \(u_n - u_{n-1} = d\) for all \(n\geq 2\)

\subsection{Geometric Progression}
	
	\subsubsection{Geometric Sequence}
	A geometric progression is a sequence of numbers which have the same constant ratio between consecutive elements. Sequences are defined by their initial term \(a\) and their constant ratio \(r\)
	\begin{equation*} \begin{gathered}
		u_1 = a = ar^{1-1}\\
		u_2 = u_1r = ar^{2-1}\\
		u_3 = u_2r = ar^{3-1} \\
		...\\
		u_n = ar^{n-1}
	\end{gathered} \end{equation*} 
	\subsubsection{Geometric Series}
	A Geometric Series is defined as the sum of a certain number of consecutive elements in a geometric sequence. \\
	For a geometric sequence of known initial term and constant difference, the term \(S_n\) can be derived from the equation
	\[ S_n = \frac{a(r^n-1)}{r-1} = \frac{a(1-r^n)}{1-r} \]
	The first equation is typically used for \(r>1\) while the second is used for \(r<1\).
	\subsubsection{Convergence}
	For a \(r<1\), is can be proven that as \(n\) tends to infinity, the value of \(u_n\) tends to zero and the value of \(S_n\) converges to a certain value and the series of this geometric sequence is said to be convergent. The value to which a series converges to is given by:
	\[ S_{\infty} = \frac{a}{1-r} \]
	The equation can be derived from the general formula of a geometric series as the numerator term \(1-r^n\) can be reduced to \(1\) as \(r^n\) tends to zero.
	\subsubsection{Proving a GP}
	To prove a GP, show that \(\frac{u_n}{u_{n-1}} = r\) for all \(n\geq 2\)

\end{document}