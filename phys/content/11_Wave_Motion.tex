\documentclass[../main]{subfiles}

\begin{document}

\section{Wave Motion}

\scidef{Wave}

\subsection{Transverse and Longitudinal Waves}

\scidef{Transverse Wave}{Transverse Waves involve particles oscillating in a direction perpendicular to the direction of motion of the wave.}

\scidef{Longitudinal Wave}{Longitudinal Waves involve particles oscillating in a direction parallel to the direction of motion of the wave.}

\subsection{Intensity of Waves}

\scidef{Intensity}{Intensity is the measurement of distribution of energy over a area.}

\scieqn{Intensity}{For power \(P\) spread across surface area \(S\), or a proportionality constant \(k\) and amplitude of wave \(A\), intensity \(I\) is defined as}{
	I = \frac{P}{S} = kA^2
}

As wave energy dissipates across a larger surface area and its total power remains the same, its intensity decreases. Cylindrically radiating waves have \(S=2\pi r\) and follow a inverse rule while spherically radiating waves have \(S=4\pi r^2\) and follow a inverse square rule.

\subsection{Polarization}

\scieqn[gathered]{Polarization}{asd}{
	A\prime = A \cos{\theta} \\
	I\prime = I \cos^2{\theta}
}

\end{document}