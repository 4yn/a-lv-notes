\documentclass[../main]{subfiles}

\begin{document}

\section{Wave Motion}

	\scidef{Wave}{Waves are displacements in a system where oscillations in a medium are propagated and therefore energy is propagated without physically relocating the medium.}

	\subsection{Transverse and Longitudinal Waves}

	\scidef{Transverse Wave}{Transverse Waves involve particles oscillating in a direction perpendicular to the direction of energy transfer.}

	\scidef{Longitudinal Wave}{Longitudinal Waves involve particles oscillating in a direction parallel to the direction of motion of energy transfer.}

	\subsection{Intensity of Waves}

	\scidef{Intensity}{Intensity is the quantization of how much energy is transferred to a surface area by a wave per unit time, measured in \si{\W\per\square\m}}

	\scieqn{Intensity}{For power \(P\) spread across surface area \(S\), or a proportionality constant \(k\), frequency \(f\) and amplitude of wave \(A\), intensity \(I\) is defined as}{I = \frac{P}{S} = kf^2A^2}

	As wave energy dissipates across a larger surface area and its total power remains the same, its intensity decreases. Cylindrically radiating waves have \(S=2\pi r\) and follow a inverse rule while spherically radiating waves have \(S=4\pi r^2\) and follow a inverse square rule.

	\subsection{Polarization}

	\scieqn[gathered]{Polarization}{For a angle \(\theta\) between the original and new planes of polarized light, the amplitude of wave \(A\) and intensity of wave \(I\) are changed as:}{
		A\prime = A \cos{\theta} \\
		I\prime = I \cos^2{\theta}
	}

\end{document}