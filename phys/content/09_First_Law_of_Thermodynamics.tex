\documentclass[../main]{subfiles}

\begin{document}

\section{Thermodynamics}

	\subsection{Heat Capacity}

	\scidef{Heat Capacity}{The Heat Capacity of a system is the ratio of energy input to temperature change of a system.}

	\scieqn{Heat Capacity}{}{c = \frac{Q}{\Delta T}}

	\scidef{Specific Heat Capacity}{The Specific Heat Capacity of a material is the ratio of energy input to temperature change of a system per mass of that material.}

	\scieqn{Specific Heat Capacity}{}{c = \frac{Q}{m\Delta T}}

	\subsection{Kinetic Model of Matter}

	Matter primarily exists in three states: solid, liquid and gas. At ANY of these phases, the mean-square speed of these particles is still proportional to its temperature, but is NOT necessarily proportional to all of its internal energy. At the solid and liquid states, energy is stored within bonds between particles and is hence unable to be detected and measured in terms of temperature. \\

	Phase changes occur when energy is released or used to form or break these bonds respectively. At these points, the temperature remains the same as energy is contributed to overcoming chemical bonds rather than increasing temperature of the object.

	\subsection{Latent Heat}

	\scidef{Latent Heat of Fusion}{The Latent Heat of Fusion is the amount of heat per unit mass required to cause a phase change from solid to liquid or vice versa.}

	\scidef{Latent Heat of Fission}{The Latent Heat of Fission is the amount of heat per unit mass required to cause a phase change from liquid to gas or vice versa.}

	The latent heat of fusion is less than the latent heat of fission because the phase change between solid and liquid only involves the breakdown of a solid lattice whereas the phase change between liquid and gas involves the breakdown of complete bonds between particles, in addition to the fact that energy needs to be supplied for a gas to form and do work against atmospheric pressure. \\

	Evaporation is followed by a cooling effect because an evaporating particle may instantaneously have a above average amount of energy and be removed from the system, hence reducing the average energy of the remaining particles and leading to a drop in the average and hence a drop in temperature.

	\subsection{Calorimetry}

	In order to find the heat capacities of an object, static heating is used for solids whereas a method of continuous flow is used for fluids. \\

	For solids, an object can be insulated from its surroundings and a immersion heater can be introduced to the object. The heat capacity is calculated from the energy dissipated from the heater and the change in reading of a temperature probe. \\

	For fluids, the inability to insulate the system results in an alternative method which accounts for heat loss \(H\). \(H\) is assumed to be constant when a system is at a constant temperature, hence a sample of the fluid is manipulated to flow through a pipe with varying mass flow rates and heat energy supplied to obtain the same temperature at the beginning and end of the pipe (typically using platinum resistance thermometers). \\

	\scieqn{Specific Heat Capacity using Constant Flow}{For two experimental setups \(x\) and \(x'\) with respective power supplied \(P\) and mass flow rate \(m\); and identical time taken for sample \(t\) and terminal temperature change \(T_\text{out} - T_\text{in}\), the specific heat capacity of the fluid is given by the equation}{c = \frac{(P - P')t}{(m-m')(T_\text{out} - T_\text{in})}}

	\subsection{Energy of a Thermodynamic System}

	\scidef{First Law of Thermodynamics}{The First Law of Thermodynamics states that as a result of conservation of energy, the change in energy of a gaseous system through a process is equal to the sum of the work done on the gas and the heat supplied to it and that Internal Energy is a state function.}

	\scieqn{First Law of Thermodynamics}{For some work done on the gas \(WD\) and heat supplied to the gas \(\Delta Q\), the change in energy of a gaseous system \(\Delta U\) is given by the equation}{
		\Delta U = WD_\text{on} + \Delta Q
	}

	\scidef{Internal Energy \(U\)}{The Internal Energy \(U\) of a system is the total energy of a system, effectively the summation of microscopic kinetic energies and potential energies of all molecules in the gas.}

	\scieqn{Internal Energy \(U\)}{At a certain \(P\) and \(V\), or with known thermodynamic temperature \(T\) and total number of molecules \(N\), the internal energy of a system is given by the equation}{U = \frac{3}{2} P V = \frac{3}{2}NkT}

	\scidef{Work Done \(WD\)}{The Work Done \(WD\) of a system is the sum of energy used to compress or expand a system done by an external force.}

	\scieqn[aligned]{Work Done \(WD\)}{}{
		WD & = - \int P dV \\
		&= - P \Delta V \quad \text{(at a constant \(P\))}
	}

	The sign convention used by the A-levels is where the a positive \(WD\) indicates compression while a negative \(WD\) indicates expansion.

	\scidef{Heat Exchange \(Q\)}{The Heat Exchange \(Q\) of a thermodynamic change is the amount of heat provided to a system.}

	\subsection{\(P-V\) Graphs}

	\scidef{Thermodynamic Process}{A Thermodynamic Process is a process which involves some change in the Pressure, Volume and Temperature quantities of a system. Processes are represented by arrows on a \(P-V\) graph.}

	\scidef{Cyclic Process}{A Cyclic Process is a set of thermodynamic changes which start and end at the same point on a \(P-V\) graph. This implies that across one cycle, the total change in internal energy is \(0\) and the net area under the cycle is numerically equal to \(\Delta Q\) and \(-WD\).}

	\subsection{Thermodynamic Processes}

	\scidef{Isobaric Process}{An Isobaric Process is one where the pressure of a system remains constant. These processes are characterized by horizontal lines on a \(P-V\) graph.}

	\scieqn[gathered]{Isobaric Process}{}{\Delta P = 0 \\ WD = P \Delta V}

	\scidef{Isothermal Process}{An Isothermal Process is one where the temperature of a system remains constant. These processes are characterized by inverse (\(\frac{1}{x}\)) lines on a \(P-V\) graph.}

	\scieqn[gathered]{Isothermal Process}{}{\Delta T = 0 \\ \Delta PV = 0 \\ \Delta U = 0 \\ WD = - \Delta Q}

	\scidef{Isovolumetric / Isochoric Process}{An Isovolumetric / Isochoric Process is one where the volume of a system remains constant. These processes are characterized by vertical lines on a \(P-V\) graph.}

	\scieqn[gathered]{Isovolumetric / Isochoric Process}{}{\Delta V = 0 \\ WD = 0 \\ \Delta U = \Delta Q}

	\scidef{Adiabatic Process}{An Adiabatic Process is one where no heat is exchanged with the external system. These processes are characterized by lines which traverse isotherms on a \(P-V\) graph.}

	\scieqn[gathered]{Adiabatic Process}{}{\Delta Q = 0 \\ \Delta U = WD}

\end{document}