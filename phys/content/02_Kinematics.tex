\documentclass[../main]{subfiles}

\begin{document}

\section{Kinematics}

	\scidef{Distance}{Distance x is the length of a path followed by an object, measured in \si{m}.}
	\scidef{Displacement}{Displacement s is the distance moved in a specified direction from a reference point, measured in \si{m}. It is the vector equivalent of distance.}
	\scidef{Speed}{Speed v is the instantaneous speed of an object, defined as the rate of change of distance traveled with respect to time, measured in \si{m.s^{-1}}. Average speed refers to the distance traveled over a significantly large time taken.}
	\scidef{Velocity}{Velocity v is the instantaneous velocity of an object, defined as the rate of change of displacement with respect to time, measured in \si{m.s^{-1}}. Average velocity refers to the change in displacement over a significantly large time taken. It is the vector equivalent of speed.}
	\scidef{Acceleration}{Acceleration a is the instantaneous change in velocity of an object, defined as the rate of change of velocity with respect to time, measured in \si{m.s^{-2}}. Average acceleration refers to the change in velocity over a significantly large time taken.}

	Note that when faced with a kinematics graph (s/v/a against t), the gradient of the graph (differential) and the area under the graph (integral) obtain special meanings.

	\subsection{Equations of Motion}

	For a situation involving uniform acceleration and motion in a straight line, the following equations hold:

	\begin{equation*} \begin{gathered}
		\text{Final velocity from initial velocity and acceleration} \\	v = u+at \\
		\text{Displacement from average velocity} \\ s = \frac{1}{2}(u+v)t \\
		\text{Displacement from initial velocity and acceleration} \\	s = ut + \frac{1/2}at^2 \\
		\text{Final velocity from displacement,} \\ \text{initial velocity and acceleration} \\	v^2 = u^2 + 2as \\
	\end{gathered} \end{equation*}


	For the condition of objects in freefall, acceleration is equal to which takes the value of 9.81 \si{m.s^{-2}}.For the conditions of objects in projectile motion with the assumption of no air resistance, acceleration in the vertical dimension behaves as if the object is in freefall, and acceleration in the horizontal direction is equal to zero.

	\subsection{Air Resistance}

	When objects move through air, it experiences viscous drag or air resistance. Air resistance acts opposite to the direction of velocity and is proportional to the velocity, or at higher velocities is proportional to the square of the velocity. The terminal velocity is the velocity at which the air resistance is equal to accelerative forces on an object, hence the resultant acceleration is equal to zero. \\

	For an object projected upwards in freefall, the time of flight upwards will be smaller than the flight downwards. \\

	On the way up, air resistance acts against upward motion and hence acts downwards in line with gravity, creating a larger resultant force downwards and a larger acceleration which retards its vertical motion, hence the velocity decreases at a faster rate and it takes less time to travel to the peak of the trajectory than if there had been no air resistance. \\

	On the way down, air resistance acts against downward motion and hence acts upward and against gravity, reducing the resultant force downwards and a lower acceleration which accelerates the object downward, hence the velocity increases at a slower rate and it takes more time to travel the same distance downward had there been no air resistance.


\end{document}