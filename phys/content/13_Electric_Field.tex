\documentclass[../main]{subfiles}

\begin{document}

\section{Electric Field}

	\scieqn[gathered]{Electric Force Equations}{For a charge \(q_1\), the radius from it \(r\), and the permittivity of free space \(\epsilon _0\), the force experienced \(F\) by a secondary charge \(q_2\), the strength of electric field \(g\), the potential energy \(U\) of a secondary charge \(q_2\) and the potential \(V\) are given by the equations}{
		F = \frac{q_1q_2}{4\pi\epsilon _0 r^2} \\
		E = \frac{q_1}{4\pi\epsilon _0 r^2} \\
		U = \frac{q_1q_2}{4\pi\epsilon _0 r} \\
		V = \frac{q_1}{4\pi\epsilon _0 r} \\
	}

	\subsection{Electric Charge}

	\scidef{Coulomb \si{\coulomb}}{The Coulomb \si{\coulomb} is the SI unit for electric charge. One coulomb is the amount of current needed for }

	\scidef{Electric Charge}{}

	\subsection{Electric Force}

	\scidef{Coulomb's Law}{Coulomb's Law states that the force between two charges is proportional to the product of both charges and inversely proportional to the square of the distance between them.}

	An electric charge in the proximity of another electric charge experiences electric force.

	\subsection{Electric Field}

	\scidef{Electric Field}{The Electric Field is a space where a charge experiences an electric force, with units \si{\V \per \m}.}

	\scidef{Electric Field Strength \(E\)}{The Electric Field Strength \(E\) at a point in space is the force per unit positive charge at a point.}

	Electric field lines in a diagram indicate the direction of acceleration of a positive test charge at a specific point. Electric field lines will never cross each other. A higher density of field lines in a unit area indicates a stronger electric field. \\

	Electric field lines originate from positive charges and terminate at negative charges. Field lines will also always be perpendicular to any conducting metal body since any other electric field will induce movement in the charges in the metal. \\

	For a hollow sphere with charge distributed at its surface, the net electric field inside the sphere is \(0\).

	\subsection{Electric Potential}

	\scidef{Electric Potential \(V\) }{The Electric Potential \(V\) at a point is the work done per unit positive charge by an external force in bringing a small test charge from infinity to a point, with units \si{\V}.}

	\scidef{Potential Difference \(\Delta V\) }{The Potential Difference \(\Delta V\) is the change in electric potential between two points.}

	Equipotential lines are lines in a diagram where all points have the same potential. Equipotential lines are always perpendicular to electric field lines.

	\subsection{Parallel Charged Plates}

	For two long parallel charged plates separated by distance \(d\)with potential difference \(V\) between them, the electric field between these two plates are approximately constant across the plates. As a result, the electric field between them is given by the equation \( E = \frac{V}{d} \).

\end{document}