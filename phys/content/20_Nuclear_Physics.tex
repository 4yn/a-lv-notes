\documentclass[../main]{subfiles}

\begin{document}

\section{Nuclear Physics}

	\subsection{The Nucleus}

	\scidef{Nucleus}{The Nucleus is a dense region at the center of an atom which contains its protons and neutrons.}

	\scidef{Nucleon}{A Nucleon refers to particles found in the nucleus, which can be either protons or neutrons.}

	\scidef{Mass Number \(A\)}{The Mass Number \(A\), also known as Nucleon Number, of a nucleus is the total number of protons and neutrons in a nucleus.}

	\scidef{Proton Number \(Z\)}{The Proton Number \(Z\) of a nucleus is the total number of protons in a nucleus.}

	\scidef{Nuclide}{A Nuclide is a specific type of nucleus with a specific mass number and a specific proton number.}

	\scidef{Isotope}{Isotopes are a group of nuclides with same proton number but different mass number.}

	\subsubsection{Nuclear Mass}

	\scidef{Atomic Mass Unit \(u\)}{1 Atomic Mass Unit , written \SI{1}{\atomicmassunit}, is equal to \(\frac{1}{12}\) the mass of a \(^{12}_6C\) atom.}

	A proton has a mass of \SI{1.007277}{\atomicmassunit}, a neutron has a mass of \SI{1.008665}{\atomicmassunit} and an electron has a mass of \SI{0.000549}{\atomicmassunit}. \\

	These values are not a perfect `1' due to binding energy mass defect.

	\subsubsection{Rutherford Scattering Experiment}

	In the Rutherford Scattering Experiment, a gold foil was bombarded with helium nuclei and the directions of which helium nuclei were deflected were recorded. A large majority of helium nuclei would seem to pass through the gold foil while a few helium nuclei (1 in 8000) would be deflected to the side. \\

	Deflections of helium nuclei to the side are caused by electric repulsion between the positively charged helium nuclei and the positively charged gold nuclei. The closer the helium nuclei approaches the gold nuclei, the more repulsion is experienced and the exit path of the helium nuclei is deflected more backward. \\

	A large majority of helium nuclei passed through undeflected as they travel through the large amount of empty space between gold nuclei, whereas only a few come near gold nuclei and are deflected. This is indicative of the extremely small size of nuclei relative to their atomic radii. \\

	\subsection{Mass-Energy Equivalence}

	Einstein's postulates on relativity and quantized light lead to the conclusion that mass and energy are interchangeable quantities. \\

	\scieqn{Einstein's Mass-Energy Equivalence}{An amount of energy \(E\) is energetically equivalent to a mass \(m\) and vice versa by the equation}{E = mc^2}

	\subsection{Binding Energy and Mass Defect}

	When nuclei are formed from protons and neutrons, large amounts of energy are released due to the strong nuclear force. This release of energy is accompanied with a reduction in observed mass of the entire atom. 

	\scidef{Mass Defect}{The Mass Defect of a nucleus is the difference between the sums of individual nucleons and the total mass of the final nucleus.}

	\scieqn{Mass Defect}{For a nucleus with mass \(m\), number of protons \(A\) and number of neutrons \(N\), its mass defect \(\Delta m\) is given by the equation}{\Delta m = A m_p + N m_n - m}

	\scidef{Nuclear Binding Energy}{The Nuclear Binding Energy of a nucleus is the energy required to separate its nucleons to infinity. This is numerically equivalent to the energy of its mass defect.}

	\subsection{Nuclear Reactions}

	\scidef{Binding Energy per Nucleon}{The Binding Energy per Nucleon is the energy required to separate its nucleons to infinity divided by the number of nucleons in a nucleus.}

	A larger binding energy per nucleon indicates that a nucleus is more stable. As a result, reactions of multiple nuclei to form a nucleus with larger binding energy per nucleon are energetically feasible processes, and can be used to obtain energy. \\

	The highest binding energy per nucleon resides between Fe and Ni nuclei. \\

	\scidef{Nuclear Reaction}{A Nuclear Reaction is a process where two nuclei or subatomic particles collide to form new species of nuclides.}

	\scidef{Fission}

	\scidef{Nuclear Fission}{Nuclear Fission is the splitting of a heavy nucleus to two lighter nuclei of approximately the same mass.}

	\scidef{Fusion}

	\subsection{Radioactivity}

	\subsubsection{Radioactive Decay}

	\subsection{Activity and Half-Life}

\end{document}