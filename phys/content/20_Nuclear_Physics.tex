\documentclass[../main]{subfiles}

\begin{document}

\section{Nuclear Physics}

	\subsection{The Nucleus}

	\scidef{Nucleus}{The Nucleus is a dense region at the center of an atom which contains its protons and neutrons.}

	\scidef{Nucleon}{A Nucleon refers to particles found in the nucleus, which can be either protons or neutrons.}

	\scidef{Mass Number \(A\)}{The Mass Number \(A\), also known as Nucleon Number, of a nucleus is the total number of protons and neutrons in a nucleus.}

	\scidef{Proton Number \(Z\)}{The Proton Number \(Z\) of a nucleus is the total number of protons in a nucleus.}

	\scidef{Nuclide}{A Nuclide is a specific type of nucleus with a specific mass number and a specific proton number.}

	\scidef{Isotope}{Isotopes are a group of nuclides with same proton number but different mass number.}

	\subsubsection{Nuclear Mass}

	\scidef{Atomic Mass Unit \(u\)}{1 Atomic Mass Unit , written \SI{1}{\atomicmassunit}, is equal to \(\frac{1}{12}\) the mass of a \(^{12}_6C\) atom.}

	A proton has a mass of \SI{1.007277}{\atomicmassunit}, a neutron has a mass of \SI{1.008665}{\atomicmassunit} and an electron has a mass of \SI{0.000549}{\atomicmassunit}. \\

	These values are not a perfect `1' due to binding energy mass defect.

	\subsubsection{Rutherford Scattering Experiment}

	In the Rutherford Scattering Experiment, a gold foil was bombarded with helium nuclei and the directions of which helium nuclei were deflected were recorded. A large majority of helium nuclei would seem to pass through the gold foil while a few helium nuclei (1 in 8000) would be deflected to the side. \\

	Deflections of helium nuclei to the side are caused by electric repulsion between the positively charged helium nuclei and the positively charged gold nuclei. The closer the helium nuclei approaches the gold nuclei, the more repulsion is experienced and the exit path of the helium nuclei is deflected more backward. \\

	A large majority of helium nuclei passed through undeflected as they travel through the large amount of empty space between gold nuclei, whereas only a few come near gold nuclei and are deflected. This is indicative of the extremely small size of nuclei relative to their atomic radii. \\

	\subsection{Mass-Energy Equivalence}

	Einstein's postulates on relativity and quantized light lead to the conclusion that mass and energy are interchangeable quantities. \\

	\scieqn{Einstein's Mass-Energy Equivalence}{An amount of energy \(E\) is energetically equivalent to a mass \(m\) and vice versa by the equation}{E = mc^2}

	\subsection{Binding Energy and Mass Defect}

	When nuclei are formed from protons and neutrons, large amounts of energy are released due to the strong nuclear force. This release of energy is accompanied with a reduction in observed mass of the entire atom. 

	\scidef{Mass Defect}{The Mass Defect of a nucleus is the difference between the sums of individual nucleons and the total mass of the final nucleus.}

	\scieqn{Mass Defect}{For a nucleus with mass \(m\), number of protons \(A\) and number of neutrons \(N\), its mass defect \(\Delta m\) is given by the equation}{\Delta m = A m_p + N m_n - m}

	\scidef{Nuclear Binding Energy}{The Nuclear Binding Energy of a nucleus is the energy required to separate its nucleons to infinity. This is numerically equivalent to the energy of its mass defect.}

	\subsection{Nuclear Reactions}

	\scidef{Binding Energy per Nucleon}{The Binding Energy per Nucleon is the energy required to separate its nucleons to infinity divided by the number of nucleons in a nucleus.}

	A larger binding energy per nucleon indicates that a nucleus is more stable. As a result, reactions of multiple nuclei to form a nucleus with larger binding energy per nucleon are energetically feasible processes, and can be used to obtain energy. \\

	The highest binding energy per nucleon resides between Fe and Ni nuclei. \\

	\scidef{Nuclear Reaction}{A Nuclear Reaction is a process where two nuclei or other subatomic particles collide to form new species of nuclides.}

	\scidef{Fission}

	\scidef{Nuclear Fission}{Nuclear Fission is the splitting of a heavy nucleus to two lighter nuclei of approximately the same mass.}

	Splitting heavy parent nuclei releases a large amount of energy as the daughter nuclei are typically more stable. \\

	Nuclear Fission is typically triggered through bombardment of the heavy nuclei with a high-speed neutron and hence require energy input to begin. However, since many fission reactions also produce high-energy neutrons, each reaction can cause additional reactions, increasing the rate of reaction in an exponential fashion in a `chain reaction' which will release large amounts of energy. \\

	Such methods of releasing energy are harnessed for energy production (nuclear energy) and military usage (atomic bombs). \\

	\scidef{Fusion}

	\scidef{Nuclear Fusion}{Nuclear Fusion is the combination of two light nuclei to form a nucleus of greater mass.}

	Combining nuclei of very low mass such has Hydrogen and Helium produce even more energy per nucleon than nuclear fission. \\

	The sun and other stars obtain its energy through the fusion of Hydrogen nuclei, in a reaction scheme known as the p-p cycle:

	\begin{equation*} \begin{aligned}
		\text{1. Deuterium Formation} & : \quad ^1_1\text{H} + ^1_1\text{H} \rightarrow ^2_1\text{H} + ^2_1\text{e} + \nu \\
		\text{2. Helium-3 Formation} & : \quad ^2_1\text{H} + ^1_1\text{H} \rightarrow ^3_2\text{He} + \gamma\\
		\text{3. Helium-4 Formation} & : \quad ^3_2\text{He} + ^3_2\text{He} \rightarrow ^4_2\text{He} + 2^1_1\text{H} \\
		\text{4. Electron Annihilation} & : \quad ^0_1\text{e} + ^0_{-1}\text{e} \rightarrow 2\gamma \\
			&	\\
		\text{Overall Reaction} & : \quad ^1_1\text{H} + ^0_{-1}\text{e} \rightarrow ^4_2\text{He}
	\end{aligned} \end{equation*}

	\scidef{Neutrino \(\nu\)}{A Neutrino \(\nu\) is a neutral, very small particle of negligible mass (many orders of magnitude less than electrons).}

	Neutrinos are added to the nuclear reactions to account for various inconsistencies in conservation of momentum and nucleon properties such as spin and lepton number. \\

	Though nuclear fusion produces extreme amounts of energy, has a large reserve of source fuel (hydrogen in water) and is much less hazardous than handling heavy nuclei, nuclear fusion reactions require temperatures up to \SI{1E8}{\K} to occur at reasonable rates and is hence unfeasible to be currently used as means of energy production. \\

	\subsection{Radioactivity}

	\scidef{Radioactivity}{Radioactivity is the random spontaneous decay of nucleus to a more stable nucleus, typically involving the emission of \(\alpha\), \(\beta\) and/or \(\gamma\) particles.}

	\scidef{Spontaneity}{A process is spontaneous if it is not affected by external factors such as temperature or pressure.}

	\scidef{Randomness}{A process is random if it cannot be predicted.}

	As for radioactivity, whether a nucleus decays within a time interval cannot be predicted. Nuclides of a single species are assumed to have a constant probability of decaying in a fixed period of time. \\

	\subsubsection{Radioactive Decay}

	\scidef{\(\alpha\)-decay}{\(\alpha\)-decay is the emission of a \(^4_2\text{He}\)helium nucleus (no electrons).}

	\scidef{\(\beta\)-decay}{\(\beta\)-decay is the emission of \(^0_-1\text{e}\)electrons.).}

	\scidef{\(\gamma\)-decay}{\(\gamma\)-decay is the emission of high-energy photons.}

	\scidef{Ionizing Power}{The Ionizing Power of a radioactive decay is the ability of the decay particle to remove electrons from other external atoms. Ionizing power decreases from \(\alpha\) to \(\gamma\) decay.}

	\scidef{Penetrating Power}{The Penetrating Power of a radioactive decay is the ability of the decay particle to pass through material before being absorbed. Penetrating power increases from \(\alpha\) to \(\gamma\) decay.}

	Which form of decay occurred in a reaction can be inferred through inspection of the atomic mass numbers and charges of specific nuclei. \\

	\subsection{Activity and Half-Life}

	\scidef{Law of Radioactive Decay}{The Law of Radioactive Decay states that the rate of decay of a source of radioactive nuclei per unit time is proportional to the total number of nuclei present.}

	\scieqn{Law of Radioactive Decay}{For a sample of \(N\) radioactive nuclei with decay constant \(N\), the rate of decay \(\frac{dN}{dt}\) is given by the equation}{\frac{dN}{dt} = - \lambda N}

	\scidef{Decay Constant \(\lambda\)}{The Decay Constant \(\lambda\) of a nuclide is the probability that a nucleus will decay within a specific duration and has units \si{\becquerel}}

	\scidef{Activity}{The Activity of a radioactive source is the number of nuclear decays that occur per unit time in a source.}

	\scidef{Becquerel}{\SI{1}{\becquerel} is the progression of one radioactive decay reaction per second.}

	\scidef{Half-life}{The Half-life of a radioactive source is the time taken for its population to decrease to half of its previous value.}

	\scieqn{Half-life}{A radioactive nuclei with decay constant \(\lambda\) and half-life \(t_{1/2}\) are related by the equation}{t_{1/2} = \frac{\ln(2)}{\lambda}}

	\scieqn[gathered]{Equations for Radioactive Decay}{For a radioactive source of decay constant \(\lambda\) and initial population \(N_0\); its population \(N\), activity \(A\) at time \(t\) are given by the equations}{
		N = N_0 e^{-\lambda t} = N_0 \frac{1}{2}^{-t/t_{1/2}} \\
		A = -\frac{dN}{dt} = \lambda N = A_0 e^{-\lambda t} = A_0 \frac{1}{2}^{-t/t_{1/2}}
	}

	\subsubsection{Background Radiation}

	\scidef{Background Radiation}{Background Radiation is the ambient radiation in environment which people are exposed to.}

	The largest natural source of background radiation is the presence of airborne Radon \(_{86}\text{Rn}\), among other sources such as cosmic rays and radiation from \(^{14}\text{C}\). The largest artificial source of background radiation comes from medical imaging equipment such as X-ray and CT scan machines. \\

	When measuring the radiation given off by a sample of radioactive matter, the background radiation must be first measured and then used to treat the data to account for extra counts from background radiation.

	\subsubsection{Carbon Dating}

	\(^{14}\text{C}\) is a radioactive isotope of Carbon which has a half-life of 5730 years. Natural carbon samples contain \SI{1.30E-12}{} part C-14 due to nuclear reactions in the upper atmosphere. \\

	Living organisms will maintain this ratio of C-14 to C-12 as carbon is constantly in exchange with the environment. However, once a organism dies its C-14 begins to decay without being replenished. By measuring the radioactivity of a sample of dead matter, the ratio of C-14 to C-12 can be established and hence the duration between its death and now can be calculated. \\

\end{document}