\documentclass[../main]{subfiles}

\begin{document}

\section{Work Energy and Power}

	\subsection{Work}

	\scidef{Work}{Work \(WD\) is defined as the product of a force and the displacement in the direction of the force, measured in Joules \si{\J} or \si{\kg\m\square\per\s\square}}

	Negative work done is a sign of a dissipative force.

	\scieqn{Work Done on a System}{For a force \(F\), displacement \(s\) and angle between Force and Displacement \(\theta\), work done \(W\) is given by the equation}{W = F s \cos(\theta)}

	\scieqn{Work Done by a Gas}{For a contained gas of changing volume \(V\) and external pressure \(p\), work done \(W\)is given by the equation}{W = p \Delta V}

	Amount of work done can be measured as the integral of the Force-Distance graph for normal motion, the integral of the Pressure-Volume graph for work done by a gas and the integral of the Force-Extension graph for work done by a spring.

	\subsection{Energy}

	\scidef{Energy}{Energy is the quantification of an object's capacity to do work,measured in \si{\J} or \si{\kg\m\square\per\s\square} }

	\scieqn{Kinetic Energy}{For a object of mass \(m\) and speed \(v\), the amount of Kinetic Energy \(E_k\) contained is given by the equation}{E_k = \frac{1}{2}mv^2}

	Note that an object whose velocity changes has an energy change of \(\frac{1}{2} m ( v^2 - u^2)\) rather than \(\frac{1}{2} m ( v - u)^2\).

	\scieqn{Gravitational Potential Energy}{For a object of mass \(m\), gravitational acceleration \(g\) and (relative) height \(h\), the amount of Gravitational Potential Energy \(E_p\) contained is given by the equation}{E_p = mgh}

	\scieqn{Elastic Potential Energy}{For a spring of proportionality constant \(k\) and extension \(x\), the amount of Elastic Potential Energy \(U_E\) contained is given by the equation}{U_E = \frac{1}{2}kx^2}

	Given the Energy-Distance graph of a object experiencing a field force, the gradient of the graph gives the force at a certain distance.

	\scidef{Principle of Conservation of Energy}{The Principle of Conservation of Energy states that energy cannot be destroyed or created, only converted and transferred.}

	The sum of all kinetic and potential energy at any point in time is constant, even in the presence of a dissipative force so long as no work is done on a system. Dissipative forces are also lessened in a system with more uniform motion.

	\scieqn{Efficiency}{To obtain the efficiency \(\eta\) of a system, use the equation}{\eta = \frac{\text{useful energy output}}{\text{total energy input}} \times 100 \percent }

	\scidef{Power}{Power is the quantification of work done with respect to time, defined as the rate which work is done with respect to time or the amount of energy transferred with respect to time, measured in \si{\W} or \si{\kg\m\square\per\s}}

	For questions involving mass flow rates, calculate values in terms of their algebraic quantities and then cancel out the time quantity at the end to find the rate.

\end{document}