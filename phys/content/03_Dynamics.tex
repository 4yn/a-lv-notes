\documentclass[../main]{subfiles}

\begin{document}

\section{Dynamics}

	\scidef{Force}{Force F is an action that causes a change in the physical shape or state of a body and is defined as the product of mass and acceleration with units \si{\N} or \si{\kg\m\per\s\square}}

	Multiple forces acting upon a body can be added together in a vector sum to find the resultant force. 

	\subsection{Newton's Laws of Motion}

	\scidef{Newton's First Law of Motion}{Newton's First Law of Motion states that a body continues in its state of rest or motion in a straight line unless acted upon by an external force.}
	\scidef{Newton's Second Law of Motion}{Newton's Second Law of Motion states that the change in momentum of a body is proportional to the resultant force acting on it and occurs in the direction of said resultant force.}
	\scidef{Newton's Third Law of Motion}{Newton's Third Law of Motion states that for a force acting from a first body on a second body, there is an equal and opposite force acting from the second body on the first body.}

	\subsection{Equilibrium}

	\scidef{Inertia}{Inertia is the tendency of a body to maintain its current motion or lack thereof unless acted upon a force.}

	\scidef{Equilibrium}{When a body experiences forces which do not change its state}

	For a object to be in equilibrium, the resultant force on the object must have zero magnitude and the resultant torque on the object about any axis must also have zero magnitude.

	\subsection{Momentum}

	\scidef{Momentum}{Momentum is defined as the product of the mass of an object and its velocity with units \si{\kg\m\per\s}.}

	The total momentum of a system is equivalent to the vector sum of its component objects' momenta. Forces can be simplified into the change of momentum over time.

	\subsection{Action Reaction Pairs}

	\scidef{Action Reaction Pairs}{Action Reaction Paris are pairs of forces which arise due to Newton's Third Law of Motion which then are of the same type (Normal Contact with Normal Contact, Friction with Friction, Electric with Electric) and act upon different bodies, in addition to properties described in the law which states that the forces are equal in magnitude but opposite in direction.}

	\subsection{Impulse}

	\scidef{Impulse}{Impulse is defined as the product of a force acting on an object and the time which the force is exerted, alternatively the amount of momentum that is transfered, with units \si{\N\s} or \si{\kg\m\per\s}.}

	Change in momentum can also be measured by finding the area under a Force-Time graph.

	\subsection{Drawing Forces}

	Free body diagrams are rudimentary drawings which illustrate the location, direction and magnitude of multiple forces acting upon objects. When drawing diagrams, label with full names of forces unless the short forms are already defined (or define them yourself in a section of the question paper).

	\begin{description}
		\item[Weight W] is drawn from the center of mass downward.
		\item[Normal Contact Force N] is drawn from the point of contact between two bodies. For two contacting surfaces, the force is drawn perpendicular to the surface.
		\item[Frictional Force f] is drawn on the surface which friction acts on.
		\item[Tension T / Compression ] is drawn along the wire, spring or strut. Tension is drawn inward while compression is drawn outward.
		\item[Upthrust U] is drawn upwards from the center of mass which is below water level.
		\item[Viscous Force \usub{F}{v}] is drawn from the center of the surface furthest from the direction of motion and is opposite to the direction of motion.
		\item[Lift L] is drawn perpendicular to the axis of wings.
		\item[Resultant \usub{F}{net}] is drawn disconnected from the body and is drawn with two arrows in the direction of motion.
	\end{description}

	\subsection{Collisions}

	Extremely quick collisions between objects involve high values of F such that assessment is more feasible by examining changes in momentum rather than forces exerted.

	\scidef{Law of Conservation of Momentum}{The Law of Conservation of Momentum states that the total momentum of a system remains the same when no external force is applied.}

	Conservation of Momentum provides the equation:

	\scieqn{Conservation of Momentum}{For mass \(m\) and initial and final velocities \(u\) and \(v\)}{m_1u_1 + m_2u_2 = m_1v_1 + m_2v_2 }

	\scidef{Law of Conservation of Energy}{The Law of Conservation of Energy states that energy can neither be created nor destroyed, hence the total energy of a closed system remains the same.}

	Conservation of Energy provides the equation:

	\scieqn{Conservation of Kinetic Energy}{For mass \(m\) and initial and final velocities \(u\) and \(v\)}{ m_1u_1^2 + m_2u_2^2 = m_1v_1^2 + m_2v_2^2 }

	\scidef{Elastic}{Elastic collisions maintain the property of conservation of momentum as well as conservation of kinetic energy.}

	Combining the equations of conservation of momentum and conservation of energy, we obtain this in the case of an elastic collision:

	\scieqn{Elastic Collision: Constant Relative Speed}{For initial and final velocities \(u\) and \(v\)}{ u_1 - u_2 = v_2 - v_1 }

	\scidef{Completely Inelastic}{Completely Inelastic collisions maintain the property of conservation of momentum but involve the conversion of kinetic energy to other forms of energy. Particles stick to each other after collision.}

\end{document}