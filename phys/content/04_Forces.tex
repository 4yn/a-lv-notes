\documentclass[../main]{subfiles}

\begin{document}

\section{Forces}

	\subsection{Elastic Force}

	\scidef{Hooke's Law}{Hooke's Law states that the extension of a spring is proportional to the applied force if the limit of proportionality is not exceeded.}

	For some distance of extension \(x\), some proportionality constant \(k\) and some force \(F\), a spring is related to the equation:

	\scieqn{Elastic Force}{ F = kx}

	And the energy stored in a spring is given by the equation:

	\scieqn{Elastic Energy}{ E = \frac{1}{2}kx^2 }

	\subsection{Frictional Force}

	\scidef{Friction}{Friction is a force which exists between two surfaces in contact with each other and resists motion between these two surfaces.}

	For a given normal contact force \(N\) between two surfaces with a frictional constant \(\mu\), the frictional force \(f\) is given by the equation:

	\scieqn{Frictional Force}{f = \mu N}

	\subsection{Upthrust}

	\scidef{Fluid}{A Fluid is a substance which can flow, including most liquids and gases.}

	\scidef{Density}{Density \(\rho\)  a substance is its mass per unit volume, with units \si{kg m^{-3}}}

	\scidef{Pressure}{Pressure p is the force per unit area exerted at right angles to a surface by some object, with units \si{N m^{-3}} or \si{kg m^{-2} s^{-2}}}

	Pressure p in a fluid is calculated using the height \(h\)of a fluid above the level considered, its density \(\rho\) and the gravitational acceleration at a point. Do note that when considering pressure in a liquid at sea level, the pressure due to atmosphere needs to be accounted for.

	\scieqn{Pressure in Fluid}{p = h \rho g}

	\subsection{Viscous Force}

	\subsection{Calculating Equilibrium}

	\subsection{Calculating Center of Mass}

\end{document}