\documentclass[../main]{subfiles}

\begin{document}

\section{Forces}

	\subsection{Elastic Force}

	\scidef{Hooke's Law}{Hooke's Law states that the extension of a spring is proportional to the applied force if the limit of proportionality is not exceeded.}

	\scieqn{Elastic Force}{For some distance of extension \(x\), some proportionality constant \(k\) and some force \(F\), the values are related by the equation}{F = k x}

	And the energy stored in a spring is given by the equation:

	\scieqn{Elastic Energy}{For some distance of extension \(x\) and some proportionality constant \(k\), the energy stored in a spring \(E\) is given by the equation}{ E = \frac{1}{2}kx^2 }

	\subsection{Frictional Force}

	\scidef{Friction}{Friction is a force which exists between two surfaces in contact with each other and resists motion between these two surfaces.}

	Friction is drawn along the line of contact between two objects. Note that friction between a wheel and a surface is in the direction of motion.

	\scieqn{Frictional Force}{For a given normal contact force \(N\) between two surfaces with a frictional constant \(\mu\) the frictional force \(f\) is given by the equation}{f = \mu N}

	\subsection{Upthrust}

	\scidef{Fluid}{A Fluid is a substance which can flow, including most liquids and gases.}

	\scidef{Density}{Density \(\rho\)  a substance is its mass per unit volume, with units \si{kg m^{-3}}}

	\scidef{Pressure}{Pressure p is the force per unit area exerted at right angles to a surface by some object, with units \si{N m^{-3}} or \si{kg m^{-2} s^{-2}}}

	Note that when considering pressure in a liquid at sea level, the pressure due to atmosphere needs to be accounted for.

	\scieqn{Pressure in Fluid}{For the height \(h\)of a fluid above the level considered, its density \(\rho\) and the gravitational acceleration \(g\) at a point, pressure \(p\) is given by the equation}{p = h \rho g}

	\scidef{Upthrust}{Upthrust is a vertically upward force exerted on a body by a fluid when it is fully or partially submerged in a fluid due to difference in fluid pressure at different heights.}

	\scidef{Archimedes Principle}{The Archimedes Principle states that the upthrust on a submerged object is equal to the weight of liquid displaced by said object.}

	Questions on upthrust usually involve a calculation of density, mass or volume of liquids or solids in a system. In the case of floating objects, utilize the equation \(U = W\) to find the weight or upthrust experienced by an object.

	\subsection{Viscous Force}

	\scidef{Viscous Force}{Viscous Force is the force experienced by a body moving through a fluid when it receives normal contact force from the particles of the fluid after it imparts momentum onto fluid, written \usub{F}{v}.}

	The magnitude of the viscous force depends on the shape of the body and viscosity of the fluid, as well as the speed of the body which has a proportional relationship to the force at low velocity and a squared relationship with the force at higher velocities. \\

	\usub{F}{v} is zero when a body is at rest. When a body affected by viscous force experiences a constant force or acceleration, the body speeds up at a decreasing rate as the resultant force is smaller due to the increasing viscous force. Terminal velocity is reached when the viscous force is equal to the applied force, resulting in equilibrium.

	\scidef{Terminal Velocity}{Terminal Velocity is the speed at which a the viscous force experienced by a body causes further acceleration to be prevented.}

	\subsection{Calculating Equilibrium}

	A body in equilibrium must have both translational and rotational equilibrium. As such, it needs to have zero resultant force as well as zero torque about any axis. \\

	Translational equilibrium is obtained when all forces acting on a body are added using vector addition and have zero resultant magnitude. Forces can be resolved into their dimensional components and summed together. \\

	\scidef{Principle of Moments}{For any body in rotational equilibrium, the sum of all clockwise moments about an axis is equal to the sum of all anticlockwise moments.}

	\scidef{Moment}{A Moment is a physical value which involves the multiplication of a perpendicular distance from an arbitrary axis with another physical quantity existent at a point.}

	\scidef{Torque}{The Torque of a force about an arbitrary axis is defined as the product of the force and the perpendicular distance from the point to the line of action of the force, with units \si{N m} or \si{kg m^2 s^{-2}}.}

	\scidef{Couple}{A Couple is a pair of equal and opposite parallel forces whose lines of action do not meet.}

	\scieqn{Torque of A Couple}{For the magnitude of one force in the couple \(F\) and the perpendicular distance between the two forces \(d\), total torque \(\tau\) is given by the equation}{\tau = F d}

	A couple has the special quality that it has zero resultant force but still has a torque. A couple will continue to rotate until the lines of action of the two forces coincide and have zero perpendicular distance.

	\subsection{Calculating Center of Mass}

	\scidef{Center of Gravity}{The Center of Gravity of an object is the point where gravitational attraction on the body appears to act.}

	CG is calculated by finding the point where when used as a pivot results in rotational equilibrium. This means that the point is vertically in line with the center of gravity. CG of an irregularly shaped object can be obtained by twice pivoting the body and drawing a line vertically down from the pivot, where the point where the lines intersect would be the center of gravity.

\end{document}