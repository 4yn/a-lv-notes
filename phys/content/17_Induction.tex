\documentclass[../main]{subfiles}

\begin{document}

\section{Electromagnetic Induction}

	\subsection{Flux and Flux Linkage}

	\scidef{Magnetic Flux}{The Magnetic Flux of a surface is the product of the magnetic field strength and the area of the surface perpendicular to the magnetic field, with units \si{\weber} or \si{\tesla\per\meter\squared} }

	\scieqn{Magnetic Flux}{The Magnetic Flux \(\Phi\) of a surface with area \(A\), magnetic field strength \(B\) and angle between normal of surface and magnetic field \(\theta\) is given by the equation}{\Phi = A B \cos \theta}

	\scidef{Magnetic Flux Linkage}{The Magnetic Flux Linkage of a coil is the product of the flux through the coil and the number of turns on the coil.}

	\scieqn{Magnetic Flux Linkage}{The Magnetic Flux Linkage \(\lambda\) of a coil with \(N\) turns and magnetic flux through cross section \(\Phi\) is given by the equation}{\lambda = N \Phi}

	\scidef{Weber}{One Weber is the amount of magnetic flux through an area of \SI{1}{\meter\squared} with magnetic field strength perpendicular to the area of \SI{1}{\tesla}}

	\subsection{Induced Electromotive Force}

	\scidef{Faraday's Law of Electromagnetic Induction}{Faraday's Law of Electromagnetic Induction states that an induced electromotive force in a system is proportional to this rate of change of magnetic flux linkage is formed.}

	\scieqn{Faraday's Law of Electromagnetic Induction}{The induced electromotive force \(E\) of a system with changing flux linkage \(\Phi\) over time \(t\) is given by the equation}{E = \frac{d E}{d t}}

	\scidef{Lenz's Law}{Lenz's Law states that the direction of an induced current will act to oppose the change in magnetic flux linkage.}

	\subsection{Moving Metals and EMF}

	\subsubsection{Moving Charged Rod}

	A charged metallic rod of length \(l\) is placed in a region with magnetic field \(B\) perpendicular to its length and moved at speed \(v\) in a direction perpendicular to its length and the magnetic field. As an area of \(lvt\) in a unit of time, magnetic flux linkage thus changes and a potential difference between the two terminals of the rod \( E = B l v\) is induced.

	\subsubsection{Rotating Metal Disc}

	A metallic disc with area \(A\) is placd horizontally in a magnetic field \(B\) normal to it is rotated at a frequency \(f\). As an amount of magnetic flux \(\Phi = A B\) is swept each rotation, there is a potential difference between the center of the disc and its edge equal to \(E = A B f\). \\

\end{document}