\documentclass[../main]{subfiles}

\begin{document}

\section{Oscillations}

\subsection{Simple Harmonic Motion}

\scidef{Simple Harmonic Motion}{Simple Harmonic Motion (SHM) occurs when a body oscillates about a point where its acceleration is proportional to its displacement from said point and directed towards said point.}

\subsection{SHM Equations}

\scieqn[gathered]{SHM Equation}{For }{
	a = -kx \\
	x = x_o \cos{\omega t} \\
	v = \frac{dx}{dt} = - x_o \omega \sin{\omega t} \\
	a = \frac{dv}{dt} = - x_o \omega^2 \cos{\omega t} \\
	  = - \omega^2 x \\
	k = \omega^2
}

\subsubsection{Graphs of SHM}

Bodies undergoing SHM have sinusoidal graphs of displacement, velocity and acceleration due to the trigonometric functions in their equations.

\subsection{Cases of SHM}

\subsubsection{Horizontal Oscillations}

\subsubsection{Vertical Oscillations}

\subsection{Damping}

\scidef{Damping}{Damping occurs when there is an external force acting on the object undergoing SHM, usually proportional to velocity (such as air resistance).}

\scidef{Light Damping}{Light Damping occurs when a damping force causes the amplitude of SHM to decrease exponentially over time.}

\scidef{Critical Damping}{Critical Damping occurs when a damping force causes a body undergoing SHM to return to equilibrium position and stop oscillating within the shortest possible time.}

\scidef{Heavy Damping}{Heavy Damping occurs when a damping force causes a body undergoing SHM to return to equilibrium position and stop oscillating over a period of time longer than if the system were critically damped.}

\subsubsection{Forced Oscillations}

\scidef{Forced Oscillations}{Forced Oscillations arise when a system has to receive external force in order to have consistent motion.}

\subsection{Resonance}

\scidef{Resonance}{Resonance occurs when a system responds to a driving force with a maximum amplitude. This implies the maximal transfer of energy between driving and driven systems, therefore implying that the driving frequency is equal to the natural frequency of the driven system.}

\scidef{Natural Frequency}{The Natural Frequency, also known as the Resonant Frequency or Resonance Frequency, is the frequency at which a system is able to receive energy at a maximum.}

RI's tutorial solutions have used the terms ``Resonance Frequency'', ``System at Resonance'' and ``Resonance Occurs''. Use these if phrasing is vague and if you worry about presentation / phrasing marks.

\end{document}