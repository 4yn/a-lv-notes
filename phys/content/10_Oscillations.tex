\documentclass[../main]{subfiles}

\begin{document}

\section{Oscillations}

\subsection{Simple Harmonic Motion}

\scidef{Simple Harmonic Motion}{Simple Harmonic Motion (SHM) occurs when a body oscillates about a point where its acceleration is proportional to its displacement from said point and directed towards said point.}

\subsection{SHM Equations}

\scieqn[gathered]{SHM Equation}{For }{
	a = -kx \\
	x = x_o \cos{\omega t} \\
	v = \frac{dx}{dt} = - x_o \omega \sin{\omega t} \\
	a = \frac{dv}{dt} = - x_o \omega^2 \cos{\omega t} \\
	  = - \omega^2 x \\
	k = \omega^2
}

\subsubsection{Graphs of SHM}

Bodies undergoing SHM have sinusoidal graphs of displacement, velocity and acceleration due to the trigonometric functions in their equations.

\subsection{Cases of SHM}

\subsubsection{Horizontal Oscillations}

\subsubsection{Vertical Oscillations}

\subsection{Damping}

\subsection{Resonance}

\end{document}