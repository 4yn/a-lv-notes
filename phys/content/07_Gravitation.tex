\documentclass[../main]{subfiles}

\begin{document}

\section{Gravitation}

	\scidef{Gravitation}{Gravitation is defined as the attractive force between to masses.}

	\scieqn[gathered]{Gravitational Equations}{For the mass of two planetary objects \(m_1\) and \(m_2\), the radius between them \(r\), and the gravitational constant \(G\), the force experienced by each of the planets \(F\), the strength of gravitational field or acceleration due to gravity \(g\), the potential energy of a each planet \(U\) and the potential gradient of a planet \(\phi\) are given by the equations}{
		F = G \frac{m_1m_2}{r^2} \\
		g = G \frac{m_1}{r^2} \\
		U = - G \frac{m_1m_2}{r} \\
		\phi = - G \frac{m_1}{r} \\
	}

	\subsection{Gravitational Force}

	\scidef{Newton's Law of Gravitation}{Newton's Law of Gravitation states that every point mass attracts every single other point mass along a line intersecting both points, which is proportional to the product of the two masses and inversely proportional to the square of the distance between the two masses.}

	Gravitational force is a field force, which does not require contact between two objects to have effect. Gravitational force is also a case of an inverse-square law. \\

	Gravitational Force is the basis which other gravitational quantities are calculated on.

	\subsection{Gravitational Field}

	Gravitational Fields are drawn as a set of field lines which demonstrate the direction and magnitude of acceleration at a certain point. Lines are drawn pointing towards masses and more dense field lines indicate stronger fields.

	\scidef{Gravitational Field Strength}{The Gravitational Field Strength \(g\)at a point is the gravitational force experienced per unit mass at a point, with units \si{\m\per\s\square}.}

	Gravitational field varies in the case of hollow or solid masses as well as whether the object is inside or outside the mass. Hollow masses have zero gravitational field inside a mass. Solid masses have fields which vary linearly inside of them and fields which follow inverse square law outside of them. \\

	When calculating gravitational field strength which involves some element of circular motion (planetary rotation, object in orbit), do consider that observed acceleration is a composite of acceleration due to gravity and centripetal acceleration.

	\subsection{Gravitational Potential}

	\scidef{Gravitational Potential Energy}{The Gravitational Potential Energy \(U\) of a mass at a point is the work done by an external force in bringing a test mass from infinity to that point, with units \si{\J}.}

	Negative total energy indicates that a mass is bounded to the gravitational field of said mass.

	\scidef{Gravitational Potential}{Gravitational Potential \(\phi\) at a point is the work done per unit mass by a external force in bringing a test mass from infinity to a point, with units \si{\J\per\kg}.}

	Gravitational Potential (Energy) is negative because the direction of displacement is opposite that of the external force. 

	\subsection{Escape Velocity}

	The escape velocity of a body is the velocity required such that said body is able to travel to infinity away from a planetary mass. Escape velocity is independent of the mass of the escaping object. Escape from Earth requires a speed of about \SI{11.2e3}{\m\per\s}.

	\scieqn{Escape Velocity}{For gravitational acceleration \(g\) and radius from the original point \(R\), the escape velocity \(v\) is given by the equation}{v = \sqrt{2gR}}

	\subsection{Planetary Orbit}

	\scidef{Kepler's Third Law}{Kepler's Third Law states that the square of the period of a planetary orbit is proportional to the cube of its radius. Mathematically, it is presented as \[T^2 \propto R^3\]}

	Given the radius and period of one planet's orbit, you can then infer mathematically the period or radius of another planet when given one of the two variables.

	\subsubsection{Geostationary Orbit}

	A geostationary satellite maintains the same position relative to a point on a planet's surface.

	\begin{itemize}
		\item Orbital period is same as the planet's rotation period (24 hours for Earth)
		\item Plane of orbit is the same as the planet's equator
		\item Direction of orbit is same as the planet's direction of rotation (eastward for Earth)
	\end{itemize}

	Geostationary orbits allow for uninterrupted surveillance of one point on a planet and is easier to communicate with, and has a high field of view due to its large orbital radius. However, geostationary orbits face a significant loss in signal strength due to the large radius, as well as poorer quality in terms of imaging satellites as well as significant latency in signal.

\end{document}