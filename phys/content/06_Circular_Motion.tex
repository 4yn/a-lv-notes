\documentclass[../main]{subfiles}

\begin{document}

\section{Circular Motion}

	\subsection{Kinematics of Circular Motion}

	\scidef{Angular Displacement}{Angular Displacement \(\theta\) is the angle an object makes with reference to a line, measured in radians.}

	\scidef{Radian}{A Radian is the angle subtended by an arc of length equal to its radius.}

	\scidef{Angular Velocity}{Angular Velocity \(\omega\) is defined as the rate of change of angular displacement with respect to time, measured in \si{\radian\per\s}}

	\scidef{Period}{A period of a system \(T\) is the time taken for a system to complete one cycle of motion, measured in \si{\s}}

	Period, with regard to circular motion, is the time taken for one complete revolution to finish.

	\scieqn{Linear Velocity}{For some angular velocity \(\omega\) and some radius \(r\), the linear velocity of an object in uniform circular motion is given by the equation}{v = r \omega}

	\subsection{Dynamics of Circular Motion}

	A object in uniform circular motion orbits an object at a constant radius, linear velocity and angular velocity, but with a changing direction. \\

	Velocity is constantly changing since direction is changing despite linear velocity remaining the same, hence there is a force acting on the body. \\
	Linear velocity is constant, hence the force acting on the body is perpendicular to the direction of motion in order to keep the linear velocity unchanged. As such, no work is done on the force as well. \\

	\scieqn{Centripetal Acceleration}{For a radius \(r\), linear velocity \(v\) and angular velocity \(\omega\), the centripetal acceleration \(a\) is given by the equation}{a = r \omega^2 = v \omega = \frac{v^2}{r}}

	\scidef{Centripetal Force}{Centripetal Force \usub{F}{c} is a name given to any force which allows a body to undergo circular motion, typically calculated as the product of centripetal acceleration and the mass of an object, measured in \si{\N}.}

	\subsection{Special Cases}

	\subsubsection{Racecar on Inclined Track}

	Horizontal component of normal force of ground on the car provides centripetal force. When the car goes above or below the speed needed to maintain this centripetal force, friction between the car tires and the track contributes additional force towards maintaining circular motion. Alternatively, the car will experience horizontal acceleration in the event it is no longer able to maintain circular motion and ``slide'' up and down the incline.

	\subsubsection{Vertical Circular Motion}

	Apparent weight is given by the normal contact force acting on an object. Since centripetal force is the vector sum of gravity and normal force, at the bottom of a loop where centripetal force acts upwards but gravity acts downwards, the normal contact force is largest so as to act against gravity to obtain the necessary centripetal force, hence apparent weight is highest at the bottom.

	Objects falling at the top of circular motion / objects (not) in contact with other objects can be explained by saying that ``contact'' is caused by having a normal force between two objects, and at the top of a loop and at low linear velocity the weight of an object is higher or lower than its necessary centripetal force, hence there is (no) normal force between object and another object. \\

	Alternatively, argue that objects in circular motion tend to move tangentially to the path of circular motion and at sufficient speeds press against other objects rather than fall due to gravity.

	\subsubsection{Vertical Circular Motion with Uniform Speed}

	Note that due to the presence of gravitational acceleration, any additional force required to act upon an object to keep it in circular motion changes at different stages of the rotation in order to keep total centripetal acceleration constant. \\

	Also note that work is done in the process of maintaining uniform speed as \usub{E}{k} is constant but \usub{E}{p} changes.

	\subsubsection{Vertical Circular Motion with Non-uniform Speed}

	In order to maintain circular motion throughout a vertical loop without any external work being done, the object must have enough energy at the bottom of the loop such that it can reach the top of the loop and still have the necessary linear velocity given the centripetal acceleration provided by gravity. Velocity at the bottom is typically in the form \(v_\text{bottom} = \sqrt{5gr}\) while velocity at the top is in the form \(v_\text{top} = \sqrt{gr}\), but its derivation is required to be shown. \\

\end{document}